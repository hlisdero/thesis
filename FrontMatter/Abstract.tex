\clearpage
\pdfbookmark{Abstract}{abstract}
\chapter*{\centering Abstract}

\section*{Detección de Deadlocks en Rust en tiempo de compilación mediante Redes de Petri}

En la presente tesis de grado se presenta una herramienta de análisis estático
para detección de \textit{deadlocks} y señales perdidas en el lenguaje de programación Rust.
Se realiza una traducción en tiempo de compilación del código fuente a una red de Petri.
Se obtiene entonces la red de Petri como salida en uno o más de los siguientes formatos:
DOT, Petri Net Markup Language o \acrshort{LoLA}.
Posteriormente se utiliza el verificador de modelos \acrshort{LoLA}
para probar de forma exhaustiva la ausencia de \textit{deadlocks} y de señales perdidas.
La herramienta está publicada como \textit{plugin} para el gestor de paquetes \textit{cargo}
y la totalidad del código fuente se encuentra disponible
en GitHub\footnote{\url{https://github.com/hlisdero/granite2}}\footnote{\url{https://github.com/hlisdero/netcrab}}.
La herramienta demuestra de forma práctica la posibilidad
de extender el compilador de Rust con un pase adicional para
detectar más clases de errores en tiempo de compilación.

\section*{Compile-time Deadlock Detection in Rust using Petri Nets}

This undergraduate thesis presents a static analysis tool
for the detection of deadlocks and missed signals in the Rust programming language.
A compile-time translation of the source code into a Petri net is performed.
The Petri net is then obtained as output in one or more of the following formats:
DOT, Petri Net Markup Language, or \acrshort{LoLA}.
Subsequently, the \acrshort{LoLA} model checker is used
to exhaustively prove the absence of deadlocks and missed signals.
The tool is published as a plugin for the package manager \textit{cargo} and
the entirety of the source code is available on GitHub\footnotemark[1]\footnotemark[2].
The tool demonstrates in a practical way the possibility
to extend the Rust compiler with an additional pass
to detect more error classes at compile time.
