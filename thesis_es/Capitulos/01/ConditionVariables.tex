\section{Condition variables}
\label{sec:condition-variables}

Las variables de condición (\textit{condition variables}) son una primitiva de sincronización en la programación concurrente
que permite a los hilos esperar eficientemente a que se cumpla una condición específica antes
de continuar. Fueron introducidas por primera vez por \cite{hoare1974monitors} como parte de un bloque
de construcción para el concepto de monitor desarrollado originalmente por \cite{hansen1973operating}.

Siguiendo la definición clásica, se pueden llamar dos operaciones principales sobre una variable de
condición:

\begin{itemize}
      \item \texttt{esperar (wait)}:
            Bloquea el hilo o proceso actual.
            En algunas implementaciones, el mutex asociado
            se libera como parte de la operación.
      \item \texttt{señal (signal)}:
            Despierta un hilo o proceso que espera en la condition variable. En algunas
            implementaciones, el mutex asociado es adquirido inmediatamente por el hilo o
            proceso sobre el que se aplica la operación.
\end{itemize}

Las condition variables suelen estar asociadas a un predicado booleano (una condición) y a
un mutex. El predicado booleano es la condición que esperan los hilos o procesos. Cuando se
establece en un valor determinado (verdadero o falso), el hilo o proceso puede continuar
ejecutándose. El mutex garantiza que sólo un hilo o proceso pueda acceder a la condition variable a la vez.

Las variables de condición no contienen un valor real accesible para el programador en su
interior. En su lugar, se implementan utilizando una estructura de datos de cola donde los hilos
o procesos se añaden a la cola cuando entran en el estado de espera. Cuando otro hilo o proceso
señala el estado, se selecciona un elemento de la cola para reanudar la ejecución. La política de
\textit{scheduling} específica puede variar en función de la implementación.

A lo largo de los años, se han desarrollado diversas implementaciones y optimizaciones para
las condition variables con el fin de mejorar el rendimiento y reducir la sobrecarga. Por
ejemplo, algunas implementaciones permiten despertar varios hilos a la vez (una operación
denominada \emph{broadcast}), mientras que otras utilizan una cola de prioridad para lograr que los
hilos de mayor prioridad se despierten primero.

Las condition variables forman parte de la biblioteca estándar POSIX para hilos (\texttt{pthreads}) \cite{nichols1996pthreads}
y en la actualidad se utilizan ampliamente en lenguajes y sistemas de programación
concurrentes. Se encuentran entre otros en:

\begin{itemize}
      \item UNIX\footnote{\url{https://man7.org/linux/man-pages/man3/pthread_cond_init.3p.html}},
      \item Rust\footnote{\url{https://doc.rust-lang.org/std/sync/struct.Condvar.html}}
      \item Python\footnote{\url{https://docs.python.org/3/library/threading.html}}
      \item Go\footnote{\url{https://pkg.go.dev/sync}}
      \item Java\footnote{
                  \url{https://docs.oracle.com/en/java/javase/20/docs/api/java.base/java/util/concurrent/locks/Condition.html}}
\end{itemize}

A pesar de su uso generalizado, las variables de condición pueden ser difíciles de utilizar
correctamente, y un uso incorrecto puede dar lugar a errores sutiles y difíciles de depurar, como
señales perdidas o despertares espurios. A continuación veremos estos errores en detalle.

\subsection{Señales perdidas}

Una señal perdida ocurre cuando un hilo o proceso que espera en una condition variable no
recibe una señal aunque haya sido emitida. Esto puede ocurrir debido a una condición de
carrera en la que la señal se emite antes de que el hilo entre en estado de espera, provocando
que se pierda la señal.

Para ilustrar el concepto de señal perdida, veremos un ejemplo. Supongamos que tenemos
dos hilos, T1 y T2, y una variable entera compartida llamada \texttt{flag}. T1 establece \texttt{flag} en \texttt{true}
y envía una señal a una variable de condición \texttt{cv} para despertar a T2 que está esperando en
cv para saber cuándo se ha marcado \texttt{flag}. T2 espera en \texttt{cv} hasta que recibe una señal de
T1. El listado \ref{lst:missed-signal-example} muestra el pseudocódigo correspondiente.

\begin{listing}[!htb]
      \begin{minted}{C++}
            // T1
            lock.acquire()
            flag = true
            cv.signal()   // Signal T2 to wake up
            lock.release()
            
            // T2
            lock.acquire()
            while (flag == false)   // Wait until flag has changed
                cv.wait(lock)
            lock.release()
      \end{minted}
      \caption{Pseudocódigo para un ejemplo de señal perdida.}
      \label{lst:missed-signal-example}
\end{listing}

Supongamos ahora que T1 activa \texttt{flag} y emite una señal a \texttt{cv} pero T2 aún no ha
entrado en estado de espera en \texttt{cv} debido a algún retraso en \textit{scheduling}. En este caso, la
señal emitida por T1 podría ser pasada por alto por T2, como se muestra en la siguiente
secuencia de acontecimientos:

\begin{enumerate}
      \item T1 adquiere el bloqueo y marca \texttt{flag} como \texttt{true}.
      \item T1 indica a \texttt{cv} que despierte a T2.
      \item T1 libera el lock.
      \item T2 adquiere el bloqueo y comprueba si \texttt{flag} ha cambiado. Como \texttt{flag} sigue siendo
            \texttt{flag}, T2 entra en estado de espera en \texttt{cv}.
      \item Debido a retrasos en el \textit{scheduling} o a otros factores, T2 no recibe la señal emitida por
            T1 y permanece atascado en el estado de espera para siempre.
\end{enumerate}

Este escenario ilustra el concepto de señal perdida, en el que un hilo que espera en una variable
de condición no recibe una señal aunque haya sido emitida. Para evitar que se pierdan
señales, es esencial asegurarse de que los hilos que esperan en variables de condición estén
correctamente sincronizados con los hilos que emiten señales y de que no existan condiciones
de carrera o problemas de sincronización que puedan hacer que se pierdan señales.

\subsection{Despertares espurios (\textit{spurious wakeups})}

Un despertar espurio (\textit{spurious wakeup}) ocurre cuando un hilo
que espera en una variable de condición se despierta sin recibir una señal o notificación de otro hilo.
Las razones son múltiples:
interrupciones del hardware o del sistema operativo, detalles internos de implementación de la
condition variable u otros factores impredecibles.

Reutilizando la situación descrita en la sección anterior y el pseudocódigo mostrado en el
Listado \ref{lst:missed-signal-example}, supongamos ahora que T1 pone el \texttt{flag} em \texttt{true} y emite una señal a \texttt{cv},
pero T2 se despierta sin recibir la señal emitida por T1.

Este es precisamente el despertar espurio. La siguiente secuencia de acontecimientos conduce a
este desafortunado resultado:

\begin{enumerate}
      \item T1 adquiere el bloqueo y marca \texttt{flag} como \texttt{true}.
      \item T1 indica a \texttt{cv} que despierte a T2.
      \item T1 libera el lock.
      \item T2 adquiere el lock y comprueba si \texttt{flag} es verdadero. Como \texttt{flag} sigue siendo
            \texttt{false}, T2 entra en estado de espera en cv.
      \item Debido a algún detalle de implementación interna de la condition variable o a otros
            factores impredecibles, T2 se despierta sin recibir la señal emitida por T1 y continúa
            ejecutando la siguiente sentencia de su código.
\end{enumerate}

Este ejemplo demuestra la idea de un despertar espurio en el que un hilo que espera en una
condition variable se despierta sin recibir una señal o notificación de otro hilo. Para evitar
los despertares espurios es inevitable utilizar un bucle para volver a comprobar la condición
después de despertarse de un estado de espera, como se muestra en el pseudocódigo para T2
(línea 9). Esto garantiza que el hilo no prosiga hasta que la condición que está esperando se
haya producido realmente. Si no existiera el bucle while, un despertar espurio haría que T2
continuara ejecutándose después de la llamada a \texttt{wait}, independientemente de si T1 emitió
una señal o no.