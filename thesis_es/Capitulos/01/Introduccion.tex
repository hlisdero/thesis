Para comprender plenamente el alcance y el contexto de este trabajo, es beneficioso
proporcionar algunos temas de fondo que sientan las bases de la investigación. Estos temas de
fondo sirven como bloques teóricos sobre los que se construye la traducción.

En primer lugar, se presenta la teoría de las redes de Petri tanto gráficamente como en términos
matemáticos. Para ilustrar el poder de modelado y la versatilidad de las redes de Petri, se
proporcionan al lector varios modelos diferentes a modo de ejemplo. Estos modelos muestran
la capacidad de las redes de Petri para capturar diversos aspectos de los sistemas concurrentes y
representarlos de forma visual e intuitiva. Más adelante, se introducen algunas propiedades
importantes y se explica el análisis de alcanzabilidad que realiza el verificador de modelos.

En segundo lugar, se analiza brevemente el lenguaje de programación Rust
y sus principales características.
Se incluye un puñado de ejemplos de aplicaciones notables de Rust en la industria.
Se reúnen pruebas convincentes del uso de lenguajes con un manejo seguro de la memoria para
argumentar que Rust proporciona una base excelente para ampliar la detección de clases de
errores en tiempo de compilación.

En tercer lugar, se ofrece información general sobre el problema de los bloqueos mutuos y las señales
perdidas cuando se utilizan \textit{condition variables}, así como una descripción de las estrategias
habituales utilizadas para resolver estos problemas.

Por último, se ofrece una visión general de la arquitectura de los compiladores y del concepto
de verificación de modelos. Señalaremos el potencial aún sin explorar que subyace a la
verificación formal para aumentar la seguridad y fiabilidad de los sistemas de software.
