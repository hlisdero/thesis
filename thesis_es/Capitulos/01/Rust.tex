\section{El lenguaje de programación Rust}

One of the most promising modern programming languages
for concurrent and memory-safe programming is Rust\footnote{\url{https://www.rust-lang.org/}}.
Rust is a multi-paradigm, general-purpose programming language that aims to provide developers
with a safe, concurrent, and efficient way to write low-level code.
It started as a project at Mozilla Research in 2009.
The first stable release, Rust 1.0, was announced on May 15, 2015.
For a brief history of Rust up to 2023, see \cite{thompson2023mit}.

Rust's memory model based on the concept of \emph{ownership} and its expressive type system prevent
a wide variety of error classes related to memory management and concurrent programming at compile-time:

\begin{itemize}
      \item Double free \cite[Chap. 4.1]{rust-book}
      \item Use after free \cite[Chap. 4.1]{rust-book}
      \item Dangling pointers \cite[Chap. 4.2]{rust-book}
      \item Data races \cite[Chap. 4.2]{rust-book}
            (with some important caveats explained in \cite[Chap. 8.1]{rustonomicon})
      \item Passing non-thread-safe variables \cite[Chap. 16.4]{rust-book}
\end{itemize}

The official compiler \emph{rustc}\footnote{\url{https://github.com/rust-lang/rust}}
takes care of controlling how the memory is used and allocating and deallocating objects.
If a violation of its strict rules is found, the program will simply not compile.

In this section, we will justify the choice of Rust
to study the detection of deadlocks and lost signals.
We will show how these problems can be studied separately,
knowing that other errors are already caught at compile time.
In other words, we will argue that
the stability and safety of the language provide a firm foundation
on which to build a tool that detects additional errors during compilation.

\subsection{Características principales}

Some of Rust's main features are:

\begin{itemize}
      \item Type system: Rust has a powerful type system
            that provides compile-time safety checks and prevents many common programming errors.
            It includes features such as type inference, generics, enums, and pattern matching.
            Every variable has a type but it is commonly inferred by the compiler.
      \item Performance: Rust's performance is comparable to C and C++,
            and it is often faster than many other popular programming languages
            such as Java, Go, Python, or Javascript.
            Rust's performance is achieved through a combination of features
            such as zero-cost abstractions, minimal runtime, and efficient memory management.
      \item Concurrency: Rust has built-in support for concurrency.
            It supports several concurrency paradigms such as shared state,
            message passing and asynchronous programming.
            It does not force the developer to implement concurrency in a specific manner.
      \item Ownership and borrowing: Rust uses a unique ownership model to manage memory,
            allowing for efficient memory allocation and deallocation
            without the risk of memory leaks or data races.
            Furthermore, it does not rely on a garbage collector, thus saving resources.
            The \emph{borrow checker} ensures that
            there is only one owner of a resource at any given time.
      \item Community-driven: Rust has a vibrant and growing community of developers
            who contribute to the language's development and ecosystem.
            Anyone can contribute to the language's development and suggest improvements.
            The documentation is also open-source and significant decisions are documented
            in the form of \acrfull{RFCs}\footnote{\url{https://rust-lang.github.io/rfcs/}}.
\end{itemize}

The release cycle of the official Rust compiler, \textit{rustc}, is remarkably fast.
A new stable version of the compiler is released every 6 weeks \cite[Appendix G]{rust-book}.
This is made possible by a complex automated testing system that compiles even all packages
available on \texttt{crates.io}\footnote{\url{https://crates.io/}}
using a program called \textit{crater}\footnote{\url{https://github.com/rust-lang/crater}}
to verify that compiling and running the tests with the new version of the compiler
does not break existing packages \cite{albini2019}.

\subsubsection{El verificador de préstamos (\textit{borrow checker})}

Rust's borrow checker is an essential component of its ownership model,
which is designed to ensure memory safety and prevent data races in concurrent code.
The borrow checker analyzes Rust code at compile-time
and enforces a set of rules to ensure that a program's memory is accessed safely and efficiently.

The core idea behind the borrow checker is that
each piece of memory in a Rust program has an owner.
The owner may change during execution but
there can only be one owner at any given time.
Memory values can also be \emph{borrowed}, that is,
used without swapping the owner,
similar to accessing the value through a pointer
or a reference in other programming languages.
When a value is borrowed, the borrower receives a reference to the value,
but the original owner retains ownership.
The borrow checker enforces rules to ensure that
a borrowed value is not modified while it is borrowed
and that the borrower releases the reference before the owner goes out of scope.

For clarity, we will now present some of the key rules enforced by the borrow checker:

\begin{itemize}
      \item No two mutable references to the same memory location can exist simultaneously.
            This prevents data races, where two threads
            try to modify the same memory location at the same time.
      \item Mutable references cannot exist at the same time
            as immutable references to the same memory location.
            This ensures that mutable and immutable references
            cannot be used simultaneously, preventing inconsistent reads and writes.
      \item References cannot outlive the value they reference.
            This ensures that references do not point to invalid memory locations,
            preventing null pointer dereferences and other memory errors.
      \item References cannot be used after their owner has been moved or destroyed.
            This ensures that references do not point to memory that has been deallocated,
            preventing use-after-free errors.
\end{itemize}

It can take some effort to write Rust code that satisfies these rules.
The borrow checker is usually singled out as
one aspect of the language that is confusing for newcomers.
However, this discipline pays off in terms of
increased memory safety and performance.
By ensuring that Rust programs follow these rules,
the borrow checker eliminates many common programming errors
that can lead to memory leaks, data races, and other bugs,
while also teaching good coding practices and patterns.

\subsubsection{Manejo de errores aplicado por el compilador}

Error handling is an essential aspect of programming and
is typically addressed in the design of programming languages.
The myriad of approaches can be summarized in two separate groups.

One group formed by languages such as C++, Java, or Python employs exceptions,
utilizing \texttt{try} and \texttt{catch} blocks to handle exceptional conditions.
When exceptions are thrown and not caught, the program terminates abruptly.

The other group is formed by languages like C or Go, among others,
where the convention is to communicate an error either through the return value of functions
or through a function parameter specifically dedicated to this purpose.
The disadvantage is that the compiler does not enforce error-checking on the programmer,
which may lead to error cases not being taken into account when adding new functionality.

Rust takes a different approach by promoting the notion that functions should ideally not fail
and that the function signature should reflect if the function may return an error.
Instead of exceptions or integer error codes,
Rust functions that may encounter errors return a
\Rustinline{std::result::Result}\footnote{\url{https://doc.rust-lang.org/std/result/}} type,
which can hold either the result of the computation
or a custom error type accompanied by a description of the error.
\emph{rustc} requires the programmer to write code for both cases
and the language provides mechanisms to facilitate error handling \cite[Chap. 9.2]{rust-book}.

In Rust, the focus lies on consistently managing the error case.
Errors can be propagated to higher-level function calls
until a consistent program state can be restored.
However, there may be situations where recovery from an error state is not feasible.
In such cases, the program can be instructed to panic,
resulting in an abrupt and ungraceful shutdown,
similar to an uncaught exception in other programming languages.
During a panic, program execution is aborted, and the stack is unwound \cite[Chap. 9.1]{rust-book}.
An error message containing details of the panic,
e.g., the error message itself and its location, is generated.
While panics can be caught by parent threads and
in specific cases when the programmer so desires\footnote{\url{https://doc.rust-lang.org/std/panic/fn.catch\_unwind.html}},
they typically lead to the termination of the current program.
This structured panic mechanism makes
the compiler aware of potential unrecoverable errors,
enabling the generation of appropriate code to handle such cases.

Rust also provides a type \Rustinline{std::option::Option}\footnote{\url{https://doc.rust-lang.org/std/option/}}
that represents either the presence of a value or the absence of it.
The compiler again enforces discipline on the programmer to always handle the \Rustinline{None} case.
Thus, Rust eliminates almost completely the need for a NULL pointer
as found in other languages like C, C++, Java, Python, or Go.

\subsection{Adopción}

In this subsection, we will briefly describe the trend
in the adoption of the Rust programming language.
This highlights the relevance of this work
as a contribution to a growing community of programmers
who emphasize the importance of safe and performant systems programming
for the next years in the software industry.

In the last few years, several major projects in the Open Source community and
at private companies have decided to incorporate Rust to reduce the number of bugs
related to memory management without sacrificing performance.
Among them, we can name a few representative examples:

\begin{itemize}
      \item The Android Open Source Project encourages
            the use of Rust for the SO components
            below the \acrfull{ART} \cite{stoep2021}.
      \item The Linux kernel, which introduces in version 6.1
            (released in December 2022) official tooling
            support for programming components in Rust
            \cite{corbet2022,desimone2022}.
      \item At Mozilla, the Oxidation project was created in 2015
            to increase the usage of Rust in Firefox and related projects.
            As of March 2023, the lines of code in Rust represent more than
            10\% of the total in Firefox Nightly \cite{mozilla-oxidation}.
      \item At Meta, the use of Rust as a development language server-side
            is approved and encouraged since July 2022 \cite{garcia2022}.
      \item At Cloudflare, a new HTTP proxy in Rust was built from scratch
            to overcome the architectural limitations of NGINX,
            reducing CPU usage by 70\% and memory usage by 67\% \cite{wu2022}.
      \item At Discord, reimplementing a crucial service written in Go
            in Rust provided great benefits in performance
            and solved a performance penalty due to the garbage collection in Go \cite{howarth2020}.
      \item At npm Inc., the company behind the npm registry, Rust allowed scaling CPU-bound services
            to more than 1.3 billion downloads per day \cite{rust-npm-case-study}.
\end{itemize}

In other cases, Rust has proved to be a great choice in existing C/C++ projects to rewrite modules
that process untrusted user input, for instance, parsers,
and reduce the number of security vulnerabilities due to memory issues \cite{chifflier2017writing}.

Moreover, the interest of the developer community in Rust is undeniable,
as it has been rated for 7 years in a row as the programming language most ``loved'' by programmers
in the Stack Overflow Developer Survey \cite{so-survey2022}.

\subsection{Importancia del uso seguro de la memoria}

In this subsection, compelling evidence supporting
the use of a memory-safe programming language is presented.
The goal is to highlight the importance of advancing research in the compile-time detection of errors
to prevent bugs that are subsequently difficult to correct in production systems.

Several empirical investigations have concluded that
around 70\% of the vulnerabilities found in large C/C++ projects are due to memory handling errors.
This high figure can be observed in projects such as:

\begin{itemize}
      \item Android Open Source Project \cite{memory-bugs-android},
      \item the Bluetooth and media components of Android \cite{memory-bugs-android-media-bluetooth},
      \item the Chromium Projects behind the Chrome web browser \cite{memory-bugs-chrome},
      \item the CSS component of Firefox \cite{memory-bugs-firefox},
      \item iOS and macOS \cite{memory-bugs-ios-macos},
      \item Microsoft products \cite{miller-security-microsoft2019, memory-bugs-microsoft},
      \item Ubuntu \cite{memory-bugs-ubuntu}
\end{itemize}

Numerous tools have set the goal to address these vulnerabilities
caused by improper memory allocation in already established codebases.
However, their use leads to a noticeable loss of performance and
not all vulnerabilities can be prevented \cite{szekeres2013sok}.
An example of a representative tool in this area, more precisely
a dynamic data racer detector for multithreaded programs in C,
can be found in \cite{savage1997eraser},
whose algorithm was later improved in \cite{jannesari2009helgrind+} and
integrated into the Helgrind tool, part of the well-known
Valgrind instrumentation framework\footnote{\url{https://valgrind.org/}}.

In \cite{jaeger2014mind}, the authors provide
a detailed survey of programming language features that
compromise the security of the resulting programs.
They discuss the intrinsic security characteristics of programming languages
and list recommendations for the education of developers or evaluators for secure software.
Type safety is mentioned as one of the key elements
for eliminating complete classes of bugs from the start.
Another noteworthy consideration is using a language where the specifications are
as complete, explicit, and formally defined as possible.
The concept of \acrfull{UB} should be included with caution and only sparingly.
Examples from the C/C++ specification illustrate the confusion
that follows from not following these principles.
The authors conclude that memory safety achieved
through garbage collection poses a threat to security
and that other mechanisms should be considered instead.

We must note that Rust itself, like any other piece of software,
is not exempt from security vulnerabilities.
Serious bugs have been discovered in the standard library in the past \cite{davidoff2018}.
Besides, code generation in Rust also includes mitigations
to exploits of various kinds \cite[Chap. 11]{rustc-book}.
However, this is far from the well-known issues in C and C++.