\section{El lenguaje de programación Rust}

Uno de los lenguajes de programación modernos más prometedores para la programación
concurrente y segura para la memoria es Rust\footnote{\url{https://www.rust-lang.org/}}.
Rust es un lenguaje de programación multiparadigma y de propósito general
cuyo objetivo es proporcionar a los desarrolladores una
forma segura, concurrente y eficiente de escribir código de bajo nivel. Comenzó como un
proyecto en Mozilla Research en 2009. La primera versión estable, Rust 1.0, se anunció el 15
de mayo de 2015. Para una breve historia de Rust hasta 2023, véase \cite{thompson2023mit}.

El modelo de memoria de Rust basado en el concepto de propiedad (\textit{ownership}) y su expresivo sistema de
tipos evitan una gran variedad de clases de errores relacionados con la gestión de memoria y la
programación concurrente en tiempo de compilación:

\begin{itemize}
      \item Double free \cite[Chap. 4.1]{rust-book}
      \item Use after free \cite[Chap. 4.1]{rust-book}
      \item Dangling pointers \cite[Chap. 4.2]{rust-book}
      \item Data races \cite[Chap. 4.2]{rust-book}
            (con algunas advertencias importantes expuestas en \cite[Chap. 8.1]{rustonomicon})
      \item Pasar variables no seguras entre hilos \cite[Chap. 16.4]{rust-book}
\end{itemize}

El compilador oficial \emph{rustc}\footnote{\url{https://github.com/rust-lang/rust}}
se encarga de controlar cómo se utiliza la memoria y de asignar y
desasignar objetos.
Si se encuentra una violación de sus reglas estrictas, el programa
simplemente no será compilado.

En esta sección, justificaremos la elección de Rust para estudiar la detección de bloqueos y
señales perdidas. Mostraremos cómo estos problemas pueden estudiarse por separado, sabiendo
que otros errores ya se detectan en tiempo de compilación. En otras palabras, argumentaremos
que la estabilidad y la seguridad del lenguaje proporcionan una base firme sobre la que
construir una herramienta que detecte errores adicionales durante la compilación.

\subsection{Características principales}

Algunas de las principales características de Rust son:

\begin{itemize}
      \item Sistema de tipos: Rust cuenta con un potente sistema de tipos que proporciona
            comprobaciones de seguridad en tiempo de compilación y evita muchos errores comunes
            de programación. Incluye características como la inferencia de tipos, los tipos genéricos,
            los enums y el \textit{pattern matching}. Cada variable tiene un tipo pero éste suele ser
            inferido por el compilador.
      \item Performance: La performance de Rust es comparable a la de C y C++ y a menudo es más
            rápido que muchos otros lenguajes de programación populares como Java, Go, Python o
            Javascript. La performance de Rust se debe a una combinación de
            características como las abstracciones de cero coste, un \textit{runtime} mínimo y
            una gestión eficiente de la memoria.
      \item Concurrencia: Rust tiene un buen soporte por defecto para la concurrencia. Soporta varios
            paradigmas de concurrencia como el estado compartido, el pasaje de mensajes y la
            programación asíncrona. No obliga al desarrollador a implementar la concurrencia de una
            manera específica.
      \item \textit{Ownership} y \textit{borrowing}: Rust utiliza un modelo de \textit{ownership} único para gestionar la
            memoria, lo que permite una asignación y desasignación de memoria eficientes sin riesgo
            de perder memoria o condiciones de carrera en el acceso a los datos.
            Además, no depende de un recolector de basura (\textit{garbage collector}),
            ahorrando recursos. El verificador de préstamos (\textit{borrow checker}) garantiza que sólo haya un
            propietario (\textit{owner}) de un recurso en un momento dado.
      \item Impulsado por la comunidad: Rust cuenta con una vibrante y creciente comunidad de
            desarrolladores que contribuyen al desarrollo y al ecosistema del lenguaje. Cualquiera
            puede contribuir al lenguaje y sugerir mejoras. La documentación también es de código abierto y las decisiones
            importantes se documentan en forma de \acrfull{RFCs}\footnote{\url{https://rust-lang.github.io/rfcs/}}.
\end{itemize}

El ciclo de publicación del compilador oficial de Rust, \textit{rustc}, es notablemente rápido.
Cada 6 semanas se publica una nueva versión estable del compilador \cite[Appendix G]{rust-book}.
Esto es posible gracias a un complejo sistema de pruebas automatizado que
compila incluso todos los paquetes disponibles en \texttt{crates.io}\footnote{\url{https://crates.io/}}
utilizando un programa llamado \textit{crater}\footnote{\url{https://github.com/rust-lang/crater}}
para verificar que la compilación y ejecución de las pruebas con la nueva versión del
compilador no rompe los paquetes existentes \cite{albini2019}.

\subsubsection{El verificador de préstamos (\textit{borrow checker})}

El verificador de préstamos (\textit{borrow checker}) de Rust es un componente esencial de su modelo de \textit{ownership},
diseñado para garantizar la seguridad de la memoria y evitar las carreras de datos (\textit{data races}) en el código
concurrente. El borrow checker analiza el código Rust en tiempo de compilación y
aplica un conjunto de reglas para garantizar que se accede a la memoria de un programa de
forma segura y eficiente.

La idea central detrás del borrow checker es que cada porción de memoria en un
programa Rust tiene un propietario. El propietario puede cambiar durante la ejecución, pero
sólo puede haber un propietario en un momento dado. Los valores de memoria también pueden
tomarse \emph{prestados} (\textit{borrowed}), es decir, utilizarse sin cambiar el propietario, de forma similar al acceso al
valor a través de un puntero o una referencia en otros lenguajes de programación.
Cuando se toma prestado un valor, el prestatario recibe una referencia al valor, pero el propietario original
conserva la propiedad. El verificador de préstamos aplica reglas para garantizar que un valor
prestado no se modifica mientras está prestado y que el prestatario libera la referencia antes de
que el propietario salga de \textit{scope}.

Para mayor claridad, presentaremos a continuación
algunas de las reglas centrales aplicadas por el borrow checker:

\begin{itemize}
      \item No pueden existir simultáneamente dos referencias mutables a la misma posición de
            memoria. Esto evita las carreras de datos en las que dos hilos intentan modificar la
            misma posición de memoria al mismo tiempo.
      \item Las referencias mutables no pueden existir al mismo tiempo que las referencias
            inmutables a la misma ubicación de memoria. Esto garantiza que las referencias mutables
            e inmutables no puedan utilizarse simultáneamente, evitando lecturas y escrituras
            incoherentes.
      \item Las referencias no pueden sobrevivir al valor al que hacen referencia. Esto garantiza que
            las referencias no apunten a ubicaciones de memoria no válidas, evitando desferencias de
            punteros nulos y otros errores de memoria.
      \item Las referencias no pueden utilizarse después de que su propietario haya sido desplazado o
            destruido. Esto asegura que las referencias no apunten a memoria que ha sido
            desasignada, evitando errores de uso después de liberar.
\end{itemize}

Puede requerir cierto esfuerzo escribir código Rust que satisfaga estas reglas.
El borrow checker suele señalarse como un aspecto del lenguaje que resulta confuso para los nuevos usuarios.
Sin embargo, esta disciplina adquirida paga sus frutos en términos de mayor seguridad de memoria y
performance durante la ejecución.
Al garantizar que los programas Rust siguen estas reglas, el verificador de
préstamos elimina muchos errores comunes de programación que pueden dar lugar a fugas de
memoria, \textit{data races} y otros bugs, al tiempo que enseña buenas prácticas y patrones de
programación.

\subsubsection{Manejo de errores aplicado por el compilador}

La gestión de errores es un aspecto esencial de la programación y suele abordarse en el diseño
de los lenguajes de programación. La miríada de enfoques puede resumirse en dos grupos
distintos.

Un grupo formado por lenguajes como C++, Java o Python emplea excepciones, utilizando
bloques \texttt{try} y \texttt{catch} para manejar condiciones excepcionales. Cuando se lanzan
excepciones y no se capturan, el programa termina abruptamente.

El otro grupo lo forman lenguajes como C o Go, entre otros, en los que la convención es
comunicar un error a través del valor de retorno de las funciones o mediante un parámetro de
función específicamente dedicado a este fin. La desventaja es que el compilador no impone al
programador la comprobación de errores, lo que puede hacer que no se tengan en cuenta los
casos de error al añadir nuevas funciones.

Rust adopta un enfoque diferente promoviendo la noción de que las funciones idealmente no
deberían fallar y que la firma de la función debería reflejar si la función puede devolver un
error. En lugar de excepciones o códigos de error con números enteros, las funciones Rust que pueden
terminar con errores devuelven un tipo
\Rustinline{std::result::Result}\footnote{\url{https://doc.rust-lang.org/std/result/}} que puede contener el resultado del
cálculo o un tipo de error personalizado acompañado de una descripción del error.
\emph{rustc} impone que el programador escriba código para ambos casos y
el lenguaje proporciona mecanismos para facilitar la gestión de errores \cite[Chap. 9.2]{rust-book}.

En Rust, el foco está puesto en la gestión coherente del caso de error. Los errores pueden
propagarse a las llamadas a funciones de nivel superior hasta que pueda restablecerse un estado
coherente del programa. Sin embargo, puede haber situaciones en las que la recuperación de un
estado de error no sea factible. En tales casos, se puede ordenar al programa que entre en
pánico, lo que resulta en un cierre abrupto y sin gracia (\textit{ungraceful}),
similar a una excepción no capturada en otros lenguajes de programación.
Durante un pánico, la ejecución del programa se aborta y la
pila se despliega (\textit{stack unwinding}) \cite[Chap. 9.1]{rust-book}.
Se genera un mensaje de error que
contiene detalles del pánico, por ejemplo, el propio mensaje de error y su ubicación.
Aunque los \textit{panic} pueden ser capturados por hilos padre (\textit{parent threads})
y en casos específicos
cuando el programador así lo desea\footnote{\url{https://doc.rust-lang.org/std/panic/fn.catch\_unwind.html}},
normalmente conducen a la terminación del programa actual. Este
mecanismo de pánico estructurado hace que el compilador sea consciente de los posibles
errores irrecuperables, lo que permite la generación del código adecuado para manejar estos
casos.

Rust también proporciona un tipo \Rustinline{std::option::Option}\footnote{\url{https://doc.rust-lang.org/std/option/}}
que representa tanto la presencia de un valor como su ausencia.
De nuevo, el compilador impone disciplina al programador para manejar
siempre el caso \Rustinline{None}.
De este modo, Rust elimina casi por completo la necesidad de un
puntero NULL como se encuentra en otros lenguajes como C, C++, Java, Python o Go.

\subsection{Adopción}

En esta subsección, describiremos brevemente la tendencia en la adopción del lenguaje de
programación Rust. Esto resalta la relevancia de este trabajo como contribución a una
comunidad creciente de programadores que hacen hincapié en la importancia de una
programación de sistemas segura y eficaz para los próximos años en la industria del software.

En los últimos años, varios proyectos importantes de la comunidad de código abierto y de
empresas privadas han decidido incorporar Rust para reducir el número de bugs relacionados
con la gestión de la memoria sin sacrificar performance. Entre ellos, podemos citar algunos
ejemplos representativos:

\begin{itemize}
      \item El Android Open Source Project fomenta el uso de Rust para los componentes del
            SO por debajo del \acrfull{ART} \cite{stoep2021}.
      \item El kernel Linux introduce en la versión 6.1 (publicada en diciembre de 2022)
            soporte oficial de herramientas para la programación de componentes en Rust
            \cite{corbet2022,desimone2022}.
      \item En Mozilla, el proyecto Oxidation se creó en 2015 para aumentar el uso de Rust en
            Firefox y proyectos relacionados. En marzo de 2023, las líneas de código en Rust
            representan más del 10\% del total en Firefox Nightly \cite{mozilla-oxidation}.
      \item En Meta, el uso de Rust como lenguaje de desarrollo del lado del servidor está aprobado
            y es alentado desde julio de 2022 \cite{garcia2022}.
      \item En Cloudflare, se construyó desde cero un nuevo proxy HTTP en Rust para superar las
            limitaciones architectónicas de NGINX, reduciendo el uso de CPU en un 70\% y el de
            memoria en un 67\% \cite{wu2022}.
      \item En Discord, la reimplementación en Rust de un servicio crucial escrito en Go
            proporcionó grandes beneficios en el rendimiento y resolvió una pérdida de
            rendimiento debida al \emph{garbage collection} en Go \cite{howarth2020}.
      \item En npm Inc, la empresa detrás del npm registry, Rust permitió escalar los servicios limitados por
            la cantidad de CPU disponible a más de 1.300 millones de descargas al día \cite{rust-npm-case-study}.
\end{itemize}

En otros casos, Rust ha demostrado ser una gran elección en proyectos C/C++ existentes para
reescribir módulos que procesan entradas de usuario no fiables, por ejemplo, analizadores
sintácticos, y reducir el número de vulnerabilidades de seguridad debidas a problemas de
memoria \cite{chifflier2017writing}.

Además, el interés de la comunidad de desarrolladores por Rust es innegable, ya que ha sido
calificado durante 7 años consecutivos como el lenguaje de programación más "querido" (\textit{loved}) por
los programadores en la encuesta Stack Overflow Developer Survey \cite{so-survey2022}.

\subsection{Importancia del uso seguro de la memoria}

En esta subsección, se presentan pruebas convincentes que apoyan el uso de un lenguaje de
programación que soporte un uso seguro de la memoria. El objetivo es resaltar la importancia de avanzar en la
investigación sobre la detección de errores en tiempo de compilación para evitar fallos que
posteriormente son difíciles de corregir en los sistemas en producción.

Varias investigaciones empíricas han concluido que alrededor del 70\% de las vulnerabilidades
encontradas en grandes proyectos C/C++ se deben a errores en el manejo de la memoria. Esta
cifra elevada puede observarse en proyectos como:

\begin{itemize}
      \item Android Open Source Project \cite{memory-bugs-android},
      \item los componentes Bluetooth y multimedia de Android \cite{memory-bugs-android-media-bluetooth},
      \item el Proyecto Chromium detrás del navegador web Chrome \cite{memory-bugs-chrome},
      \item el componente CSS de Firefox \cite{memory-bugs-firefox},
      \item iOS y macOS \cite{memory-bugs-ios-macos},
      \item productos de Microsoft \cite{miller-security-microsoft2019, memory-bugs-microsoft},
      \item Ubuntu \cite{memory-bugs-ubuntu}
\end{itemize}

Numerosas herramientas se han fijado el objetivo de abordar estas vulnerabilidades causadas
por una asignación inadecuada de memoria en bases de código ya establecidas.
Sin embargo, su uso conlleva una pérdida notable de rendimiento y no todas las vulnerabilidades pueden
evitarse \cite{szekeres2013sok}.
Un ejemplo de herramienta representativa en este ámbito, más
concretamente un detector dinámico de data races para programas multihilo en C, puede
encontrarse en \cite{savage1997eraser},
cuyo algoritmo se mejoró posteriormente en \cite{jannesari2009helgrind+}
y se integró en la herramienta Helgrind, parte del conocido framework de instrumentación
Valgrind\footnote{\url{https://valgrind.org/}}.

En \cite{jaeger2014mind}, los autores ofrecen un estudio detallado de las características de
los lenguajes de programación que comprometen la seguridad de los programas resultantes.
Hablan de las características de seguridad intrínsecas de los lenguajes de programación y
enumeran recomendaciones para la formación de desarrolladores o evaluadores de software
seguro. La seguridad de tipos se menciona como uno de los elementos clave para eliminar
clases completas de errores desde el principio. Otra consideración digna de mención es utilizar
un lenguaje en el que las especificaciones sean lo más completas, explícitas y formalmente
definidas posible. El concepto de \acrfull{UB} debe incluirse con
precaución y sólo con moderación. Algunos ejemplos de la especificación C/C++ ilustran la
confusión que se deriva de no seguir estos principios. Los autores concluyen que la
seguridad de la memoria conseguida mediante la recogida de basura (\textit{garbage collection})
supone una amenaza para la seguridad y que en su lugar deberían considerarse otros mecanismos.

Debemos tener en cuenta que el propio Rust, como cualquier otra producto de software, no está
exento de vulnerabilidades de seguridad. En el pasado se han descubierto bugs serios en la
biblioteca estándar \cite{davidoff2018}. Además, la generación de código en Rust también incluye
mitigaciones a exploits de diversa índole \cite[Chap. 11]{rustc-book}. Sin embargo, esto
dista mucho de los problemas ampliamente conocidos en C y C++.