\section{Correctitud de programas concurrentes}

En el área de la computación concurrente, uno de los principales retos es demostrar la
correctitud de un programa concurrente. A diferencia de un programa secuencial en el que para
cada entrada se obtiene siempre la misma salida, en un programa concurrente la salida puede
depender de cómo se hayan intercalado las instrucciones de los distintos procesos o hilos
durante la ejecución.

La correctitud de un programa concurrente se define entonces en términos de las propiedades
de la computación realizada y no sólo en términos del resultado obtenido.
En la literatura \cite{ben-ari2006,coulouris2012,tanenbaum2017}, se definen dos tipos de
propiedades de correctitud:

\begin{itemize}
    \item \textbf{Propiedades de seguridad (\emph{Safety properties}):}: La propiedad debe ser \emph{siempre} verdadera.
    \item \textbf{Propiedades de liveness}: La propiedad debe volverse \emph{eventualmente} verdadera.
\end{itemize}

Dos propiedades de seguridad deseables en un programa concurrente son:

\begin{itemize}
    \item \textbf{Exclusión mutua}: Dos procesos no deben acceder a los recursos compartidos al mismo tiempo.
    \item \textbf{Ausencia de bloqueo (\emph{deadlock})}: Un sistema en funcionamiento debe poder seguir realizando
          su tarea, es decir, progresando y produciendo trabajo útil.
\end{itemize}

Las primitivas de sincronización como los mutexes, los monitores
(propuestos por \cite{hansen1972structured,hansen1973operating}),
los semáforos (propuestos por \cite{Dijkstra2002}) y las variables de
condición (propuestas por \cite{hoare1974monitors}) suelen utilizarse para implementar el acceso
coordinado de hilos o procesos a recursos compartidos. Sin embargo, el uso correcto de estas
primitivas es difícil de conseguir en la práctica y puede introducir errores difíciles de detectar y
corregir. Actualmente, la mayoría de los lenguajes de propósito general, ya sean compilados o
interpretados, no permiten detectar estos errores en todos los casos.

Dada la creciente importancia de la programación concurrente debido a la proliferación de
sistemas de hardware multihilo y multihilo, minimizar la aparición de errores asociados a la
sincronización de hilos o procesos tiene una importancia innegable para la industria. El
funcionamiento libre de deadlocks es un requisito fundamental para muchos proyectos, como los
sistemas operativos \cite{ArpaciDusseau2018},
las aeronaves \cite{carreno2005safety,monzon2009deadlock}
y los vehículos autónomos \cite{Perronnet2019}.

En la próxima sección examinaremos más detenidamente las condiciones que provocan un
deadlock y las estrategias utilizadas para hacerles frente.
