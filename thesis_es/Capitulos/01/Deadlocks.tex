\section{Bloqueo mutuo (\textit{deadlocks})}

Los bloqueos o \textit{deadlocks} son un problema común en sistemas concurrentes, es decir, sistemas
en los que varios hilos o procesos se ejecutan simultáneamente y potencialmente comparten
recursos. Se han estudiado al menos desde \cite{dijkstra1964}
quien acuñó el término ``abrazo mortal'' en holandés el cual cayó eventualmente en desuso.

Un deadlock se produce cuando dos o más hilos o procesos están bloqueados y no pueden seguir
ejecutándose porque cada uno está esperando a que el otro libere un recurso que necesita.
Esto da lugar a una situación en la que ninguno de los hilos o procesos puede progresar y el sistema
queda efectivamente atascado. En \cite{holt1972some} puede encontrarse una definición alternativa
equivalente de los bloqueos en términos de los estados del programa.

Los deadlocks pueden ser un problema grave en los sistemas concurrentes, ya que
pueden provocar que el sistema deje de responder o incluso que se interrumpa la ejecución abruptamente.
Por lo tanto, sería ventajoso poder detectar y prevenir los dealocks.
Pueden producirse en cualquier sistema concurrente
en el que varios hilos o procesos compiten por recursos compartidos.
Algunos ejemplos de recursos compartidos que pueden provocar deadlocks son la memoria del
sistema, los dispositivos de entrada/salida, los \textit{locks} y otros tipos de primitivas de
sincronización.

Los bloqueos pueden ser difíciles de detectar y prevenir
porque dependen de la sincronización precisa de los eventos en el sistema.
Incluso en los casos en los que pueden detectarse
los deadlocks, resolverlos puede ser difícil, ya que puede requerir liberar recursos que
ya han sido adquiridos o deshacer transacciones completadas.
Para evitar los bloqueos, es importante gestionar cuidadosamente
los recursos compartidos en un sistema concurrente.
Esto puede implicar el uso de técnicas como algoritmos de asignación de recursos,
algoritmos de detección de bloqueos y otros tipos de primitivas de sincronización.
Gestionando cuidadosamente los recursos compartidos
es posible evitar que se produzcan bloqueos y
garantizar el buen funcionamiento de los sistemas concurrentes.

Para entender el concepto con más detalle,
considérese un ejemplo sencillo en el que dos
procesos, A y B, compiten por dos recursos, X e Y.
Inicialmente, el proceso A ha adquirido el
recurso X y está esperando adquirir el recurso Y,
mientras que el proceso B ha adquirido el
recurso Y y está esperando adquirir el recurso X.
En esta situación, ninguno de los dos procesos puede seguir ejecutándose
porque está esperando a que el otro proceso libere un
recurso que necesita.
Esto da lugar a un deadlock, ya que ninguno de los dos procesos
puede progresar.
La Fig. \ref{fig:state-graph-example} ilustra esta situación.
El ciclo que aparece en ella indica un deadlock
como se explicará en la siguiente sección.

\begin{figure}[!htb]
      \centering
      \includesvg[width=0.3\linewidth]{state-graph-example.svg}
      \caption{Ejemplo de un grafo de estados con un ciclo que indica un bloqueo mutuo.}
      \label{fig:state-graph-example}
\end{figure}
\subsection{Condiciones necesarias}
\label{sec:coffman-conditions}

Según el paper clásico sobre el tema \cite{coffman1971deadlocks}, deben darse las siguientes
condiciones para que se produzca un deadlock.
A veces se las denominan ``condiciones de Coffman''.

\begin{enumerate}
      \item \textbf{Exclusión mutua (\textit{Mutual Exclusion})}:
            Al menos un recurso del sistema debe mantenerse en un modo no
            compartible, lo que significa que sólo un hilo o proceso puede utilizarlo a la vez, por
            ejemplo, una variable detrás de un mutex.
      \item \textbf{Retener y esperar (\textit{Hold and Wait})}:
            Al menos un hilo o proceso del sistema debe estar reteniendo un
            recurso y esperando para adquirir recursos adicionales que en ese momento están siendo
            retenidos por otros hilos o procesos.
      \item \textbf{Sin apropiación (\textit{No Preemption})}:
            Los recursos no pueden apropiarse, lo que significa que un hilo o proceso
            que posea un recurso no puede ser obligado a liberarlo hasta que haya completado su
            tarea.
      \item \textbf{Espera circular (\textit{Circular Wait})}:
            Debe haber una cadena circular de dos o más hilos o procesos, en la
            que cada hilo o proceso esté esperando un recurso en poder del siguiente de la cadena.
            Esto suele visualizarse en un gráfico que representa el orden en que se adquieren los
            recursos.
\end{enumerate}

Usualmente, las tres primeras condiciones son características del sistema estudiado, es decir,
los protocolos utilizados para adquirir y liberar recursos, mientras que la cuarta puede
materializarse o no en función del intercalado de instrucciones durante la ejecución.

Cabe señalar que las condiciones de Coffman son en general necesarias pero no suficientes
para que se manifieste un bloqueo. En efecto, las condiciones son suficientes en el caso de
sistemas de recursos de una sola instancia (una unidad de cada recurso).
Pero sólo indican la posibilidad de un deadlock
en los sistemas en los que hay múltiples instancias indistinguibles del mismo recurso.

En el caso general, si no se cumple alguna de las condiciones, no puede producirse un bloqueo,
pero la presencia de las cuatro condiciones no garantiza necesariamente un bloqueo. No
obstante, las condiciones de Coffman son un marco útil para comprender y analizar las causas
de los bloqueos en los sistemas concurrentes y pueden ayudar a orientar el desarrollo de
estrategias para prevenir y resolver los bloqueos.

\subsection{Estrategias}
\label{sec:deadlock-strategies}

Existen varias estrategias para gestionar los bloqueos,
cada una de las cuales tiene sus puntos fuertes y débiles.
En la práctica, la estrategia más eficaz dependerá de los requisitos y
limitaciones específicos del sistema que se esté desarrollando.
Los diseñadores y desarrolladores deben considerar cuidadosamente
las compensaciones entre las distintas
estrategias y elegir el enfoque que mejor se adapte a sus necesidades.
Se remite a los lectores
interesados a \cite{coffman1971deadlocks,singhal1989deadlock}.

\subsubsection{Prevención}

Una forma de hacer frente a los bloqueos es evitar que se produzcan en primer lugar. La idea es
que los bloqueos se excluyan a priori. Con este objetivo en mente, debemos asegurarnos de que
en cada momento no se cumple al menos una de las condiciones necesarias desarrolladas en la
Sec. \ref{sec:coffman-conditions}.
Esto restringe los posibles protocolos en los que se pueden realizar solicitudes de
recursos. A continuación examinaremos cada condición por separado y desarrollaremos los
enfoques más comunes.

Si la primera condición debe ser falsa, entonces el programa debe permitir el acceso
compartido a todos los recursos. Los algoritmos de sincronización sin bloqueo pueden
utilizarse para este fin, ya que no implementan la exclusión mutua. Esto es difícil de conseguir
en la práctica para todos los tipos de recursos, ya que, por ejemplo, un archivo no puede ser
compartido por más de un hilo o proceso durante una actualización del contenido del archivo.

En cuanto a la segunda condición, un enfoque viable sería imponer que cada hilo o proceso
adquiera todos los recursos necesarios a la vez y que el hilo o proceso no pueda continuar hasta
que se le haya concedido acceso a todos ellos. Esta política de ``todo o nada'' provoca una
penalización significativa del rendimiento, dado que los recursos pueden asignarse a un hilo o
proceso específico pero pueden permanecer sin utilizar durante largos periodos. En términos
más sencillos, disminuye la concurrencia.

Si se deniega la condición de no apropiación, los recursos pueden recuperarse en determinadas
circunstancias, por ejemplo, utilizando algoritmos de asignación de recursos que garanticen que
los recursos nunca se retienen indefinidamente. Tras un tiempo de espera o cuando se cumple
una condición, el hilo o proceso libera el recurso o un proceso supervisor lo recupera a la
fuerza. Normalmente, esto funciona bien cuando el estado del recurso puede guardarse
fácilmente y restaurarse más tarde. Un ejemplo de ello es la asignación de núcleos de CPU en
un sistema operativo (\acrshort{OS}) moderno.
El \textit{scheduler} asigna un núcleo de procesador a una
tarea y puede cambiar a una tarea diferente o puede mover la tarea a un nuevo núcleo de
procesador en cualquier momento simplemente guardando el contenido de los registros
\cite[Chap. 6]{ArpaciDusseau2018}. Sin embargo, si no es posible preservar el
estado de los recursos, el adelantamiento puede implicar una pérdida del progreso realizado
hasta el momento, lo que no es aceptable en muchos escenarios.

Por último, si el grafo de estados de los recursos nunca forma un ciclo, entonces la cuarta
condición necesaria es falsa y se evitan los bloqueos. Para lograrlo, se podría introducir una
ordenación lineal de los tipos de recursos. En otras palabras, si a un proceso o hilo se le han
asignado recursos del tipo $r_i$, podrá requerir posteriormente sólo aquellos recursos de tipos que
sigan a $r_i$ en el ordenamiento. Esto implica utilizar primitivas de sincronización especiales que
permitan compartir recursos de forma controlada y aplicar reglas estrictas para la adquisición y
liberación de recursos. En estas condiciones, el grafo de estado será estrictamente un bosque
(un grafo acíclico), por lo que no es posible que se produzcan bloqueos.

En aplicaciones prácticas, una combinación de las estrategias anteriores puede resultar útil
cuando ninguna de ellas sea totalmente aplicable.

\subsubsection{Evasión (\textit{avoidance})}

La evitación es otra estrategia para hacer frente a los bloqueos, que consiste en detectar y evitar
dinámicamente los bloqueos potenciales \emph{antes} de que se produzcan. Para ello, el sistema
requiere un conocimiento global por adelantado sobre qué recursos solicitará un hilo o proceso
durante su vida. Tenga en cuenta que, en términos lingüísticos, ``evitar un bloqueo''
y ``prevenir un bloqueo'' pueden parecer similares, pero en el contexto de la gestión de bloqueos, son
conceptos distintos.

Uno de los algoritmos clásicos para evitar el bloqueo es el algoritmo de Banker \cite{dijkstra1964}.
Otro algoritmo relevante es el propuesto por \cite{habermann1969prevention}.

Lamentablemente, estas técnicas sólo son efectivas en escenarios muy específicos, como en un
sistema embebido en el que se conoce a priori
el conjunto completo de tareas a ejecutar y el \textit{locks} necesarios.
En consecuencia, la evitación de bloqueos no es una solución de uso
común aplicable a una amplia gama de situaciones.

\subsubsection{Detección y recuperación}

Otra estrategia para gestionar los bloqueos es detectarlos \emph{después} de que se produzcan y
recuperarse de ellos. Para un estudio de los algoritmos de detección de bloqueos en sistemas
distribuidos, véase \cite{singhal1989deadlock}. Presentaremos brevemente la idea general que subyace a
uno de ellos con fines ilustrativos.

El gráfico de asignación de recursos (\acrfull{RAG}) es un método comúnmente utilizado para detectar
bloqueos en sistemas concurrentes. Representa la relación entre hilos/procesos y recursos en el
sistema como un grafo dirigido. Cada proceso y recurso está representado por un nodo en el
grafo y se traza una arista dirigida desde un proceso a un recurso si el proceso está actualmente
ocupando ese recurso.
Esto es análogo al grafo de estado mostrado en la Fig. \ref{fig:state-graph-example} pero con los
hilos/procesos representados en el diagrama. El grafo de estado también puede aplicarse a la
detección de bloqueos \cite{coffman1971deadlocks}.

Para detectar bloqueos mediante el \acrshort{RAG} tenemos que buscar ciclos en el gráfico.
Si hay un ciclo en el gráfico, indica que un conjunto de procesos está esperando recursos que en ese
momento están en manos de otros procesos del ciclo. Por lo tanto, ningún proceso del ciclo
puede avanzar.

La parte de recuperación del proceso consiste en terminar uno de los hilos o procesos del ciclo.
Esto hace que se liberen los recursos y que los demás hilos o procesos puedan continuar.

Los sistemas de gestión de bases de datos (\acrfull{DBMS}) incorporan subsistemas para detectar y
resolver los bloqueos. Un detector de bloqueos se ejecuta a intervalos, generando un gráfico de
asignación regular, también llamado gráfico de transacción-espera (\acrfull{TWF}), y examinándolo en
busca de cualquier ciclo. Si se identifica un ciclo (deadlock), el sistema debe reiniciarse. Una
excelente visión general de la detección de bloqueos en sistemas de bases de datos distribuidos
es \cite{knapp1987deadlock}. El tema del control de la concurrencia y la recuperación de los bloqueos en
los \acrshort{DBMS} se trata ampliamente en \cite{bernstein1987concurrency}.

\subsubsection{Aceptar o ignorar por completo los deadlocks}

En algunos casos, puede ser admisible aceptar simplemente el riesgo de que se produzcan
deadlocks y gestionarlos a medida que vayan apareciendo. Este enfoque puede ser adecuado
en sistemas en los que el coste de prevenir o detectar los deadlocks sea demasiado
elevado, o en los que la frecuencia de los deadlocks sea lo suficientemente baja como
para que el impacto en el rendimiento del sistema sea mínimo, o en los que la pérdida de datos
que se produzca cada vez sea tolerable.

UNIX es un ejemplo de sistema operativo que sigue este principio \cite[p. 477]{shibu2016}. Otros
sistemas operativos populares también muestran este comportamiento. Por otro lado, un
sistema de misión crítica no puede permitirse fingir que su funcionamiento estará libre de bloqueos por
ningún motivo.