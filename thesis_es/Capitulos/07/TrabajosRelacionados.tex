En \cite{rawson2022petri} los autores proponen un modelo generalizado basado en redes
de Petri coloreadas e implementan un framework middleware de código abierto
en Rust\footnote{\url{https://github.com/MarshallRawson/nt-petri-net}}
para construir, diseñar, simular y analizar las redes de Petri resultantes.

Las redes de Petri coloreadas (\acrfull{CPN}) son un tipo de red de Petri que puede representar sistemas
más complejos que las redes de Petri tradicionales. En una CPN, las fichas tienen asociado un
valor específico, que puede representar diversos atributos o propiedades del sistema que se está
modelando. Esto permite un modelado más detallado y preciso de los sistemas del mundo real,
incluidos aquellos con estructuras de datos y comportamientos complejos. En la representación
visual, cada token tiene un color (análogo a un tipo en los lenguajes de programación) y las
transiciones esperan tokens de un determinado color (tipo) y pueden generar tokens del mismo
color o tokens de un color diferente. Como ejemplo breve, considere una transición con dos
lugares de entrada y uno de salida que representa la mezcla de colores primarios. Si los colores
de las fichas de entrada son rojo y azul, entonces el color de la ficha de salida es morado. Si los
colores de las fichas de entrada son amarillo y azul, entonces el color de la ficha de salida es
verde.

El modelo propuesto por los autores es un tipo de red de Petri aún más general, denominado
Redes de Petri de transición no determinista (\acrfull{NT-PN}) que permite que las transiciones se
disparen sin que todos sus lugares de entrada estén marcados con fichas, al tiempo que permite
que cada transición defina qué lugares de salida deben marcarse en función de la entrada. En
otras palabras, cada transición define reglas arbitrarias para que se produzca su disparo.
Explican brevemente cómo podría analizarse la red de Petri para resolver el número máximo de
hilos útiles para ejecutar la tarea modelada en ella. También mencionan el paso de modelado
como herramienta para comprobar si hay estados erróneos antes de desplegar un sistema
electrónico o informático.

En \cite{deboer2013petri} se presenta una traducción de un lenguaje formal a redes de Petri para
la detección de bloqueos en el contexto de objetos activos y futuros. El lenguaje formal
elegido es \acrfull{Creol}. Se trata de un lenguaje
orientado a objetos diseñado para especificar sistemas distribuidos.
En este trabajo, el programa está
formado por objetos activos que se comunican asíncronamente y en el que se utilizan futuros
para manejar los valores de retorno, que pueden recuperarse mediante una primitiva \texttt{get} que
retiene el lock (bloqueante) o una primitiva \texttt{claim} que libera el lock (no bloqueante).
Tras traducir el programa a una red de Petri, se aplica el análisis de alcanzabilidad para detectar
los bloqueos. Este artículo muestra que también es posible una traducción de estrategias de
comunicación asíncrona a redes de Petri con el objetivo de detectar los bloqueos.