\section{MIR function}

En la siguiente sección, profundizaremos en el proceso de traducción de una función \acrshort{MIR}.
Esta sección pretende ofrecer una comprensión exhaustiva de las técnicas de traducción aplicadas a
elementos específicos de la \acrshort{MIR}, a saber, los \acrfull{BB}, las sentencias y los
\textit{terminator statements}. Estos componentes se introdujeron anteriormente en la Sec. \ref{sec:mir-components}.

La implementación en el repositorio lleva el nombre
\Rustinline{MirFunction}\footnote{\url{https://github.com/hlisdero/cargo-check-deadlock/blob/main/src/translator/mir_function.rs}}.
Este tipo almacena el lugar de inicio y el lugar final de la función. Estos deben suministrarse a la función
\acrshort{MIR} porque también representan dónde tuvo lugar la llamada a la función y a dónde debe
volver. El lugar final es, en términos más sencillos, el lugar de retorno en la red de Petri.
La Fig. \ref{fig:simplest-function} ilustra esto.

El lugar de inicio de la función se solapa con el lugar que modela el primer bloque básico de la
función. Esto se ajusta más a la \acrshort{MIR}, ya que el código sólo vive dentro de los bloques básicos,
por lo que la llamada a la función comienza en el primer bloque básico (BB0).

La función \Rustinline{MirFunction} también almacena el ID que la identifica.
Esto es necesario para realizar llamadas a funciones desde esta función.
Asimismo, la función requiere un nombre que es
diferente para cada llamada a la función, por lo que recibe un nombre con un índice añadido,
lo que lo hace único en toda la red de Petri.

A continuación explicaremos cómo se expresa cada componente en el lenguaje de las redes de
Petri. Mediante una exploración detallada de las técnicas de traducción empleadas para los
bloques básicos, las sentencias y los terminadores, desarrollaremos un modelo formal que
capture con precisión el comportamiento de una función \acrshort{MIR} para la detección de bloqueos.

\subsection{Bloques básicos}

Un aspecto del proceso de traducción consiste en transformar los bloques básicos en redes de
Petri, que sirven como bloque de construcción fundamental para modelar el flujo de control
dentro de la función \acrshort{MIR}. Como se ve en la Fig. \ref{fig:mir-cfg-example},
un bloque básico en \acrshort{MIR} actúa como un contenedor
que alberga una secuencia de cero o más sentencias (\emph{statements}), así como una sentencia de
terminación obligatoria.

Como nodos de un grafo, la principal propiedad de los bloques básicos es su capacidad para
dirigir el flujo de control dentro de un programa. Cada bloque básico puede tener uno o más
bloques básicos que apunten a él, indicando los posibles caminos desde los que el flujo de
control puede alcanzarlo. Del mismo modo, un bloque básico puede apuntar a otro u otros
bloques básicos, señalando los posibles caminos que puede tomar el flujo de control tras
ejecutar el bloque básico actual. Cabe mencionar que los bloques básicos aislados sin
conexiones no tienen sentido, debido a que nunca se ejecutarían, es decir, son código muerto.

Este comportamiento de bifurcación permite un flujo de control dinámico dentro del programa,
ya que varios bloques básicos pueden continuar el flujo de control hacia el mismo bloque
básico de destino (por ejemplo, hacia un bloque que realice tareas de limpieza). A la inversa,
un bloque básico puede bifurcarse y determinar el siguiente bloque básico basándose en
condiciones específicas o en la lógica del programa,
por ejemplo, en un \Rustinline{if}, \Rustinline{while}, \Rustinline{match} u otras
estructuras de control. Esta versatilidad en el flujo de control proporciona la base para modelar
el comportamiento de programas complejos.

El modelo de red de Petri utilizado en la aplicación se basa en un único lugar para modelar
cada \acrshort{BB}. Podemos abstraernos del funcionamiento interno de los \acrshort{BB} y trabajar con un único
lugar. La razón de ello es que las conexiones con otros \acrshort{BB} dependen únicamente de los
terminadores y los \textit{statements} no se modelan en absoluto, como veremos en breve. Por otra parte,
la implementación\footnote{\url{https://github.com/hlisdero/cargo-check-deadlock/blob/main/src/translator/mir_function/basic_block.rs}}
mantiene un registro del nombre de la función a la que pertenece el \acrshort{BB} y del
número de \acrshort{BB} para generar etiquetas únicas.

\subsection{Statements}

Los statements \acrshort{MIR} \emph{no} se incorporan intencionalmente al modelo de red de Petri.
Teniendo en cuenta que las razones para ello y los beneficios pueden no ser evidentes de
inmediato, proporcionaremos una explicación detallada de esta decisión de implementación.

El enfoque que se aplicó anteriormente sí incluía el modelado de statements. Se basaba en el
enfoque visto en \cite{meyer2020}. Sin embargo, se observó que esto conducía a la creación de
una larga cadena de lugares y transiciones que no contribuía significativamente a la detección
de deadlocks o señales perdidas. Adicionalmente inflaba innecesariamente el tamaño de la
representación en red de Petri, dificultando su depuración y comprensión. En consecuencia,
este enfoque se revisó posteriormente y se eliminó en un
commit posterior\footnote{\url{https://github.com/hlisdero/cargo-check-deadlock/commit/b27403b6a5b2bb020a5d7ab2a9b1cacefb48be82}}.

En todos los programas probados hasta ahora, los statements no realizaban
ninguna acción que justificara su adición a la red de Petri. Al contrario, las redes que
incluían los statements eran más grandes y más difíciles de leer. Para facilitar el uso y la
adopción de la herramienta, es crucial optimizar la red de Petri para el propósito de la
herramienta. Por lo tanto, la decisión fue eliminar todo el código relacionado con el modelado
de las declaraciones y corregir los tests para que se ajustaran al nuevo resultado esperado.

La alternativa era desactivar las sentencias con una compile flag, pero eso
complicaría las pruebas y, dado que de todos modos no hay ningún caso de uso para modelar
las sentencias \acrshort{MIR}, se descartó esta opción.

A título ilustrativo, podemos remitirnos a la Fig. \ref{fig:statement-model-comparison}
que muestra una comparación entre el modelo antiguo y el nuevo.
Las diferencias entre estas dos representaciones son
evidentes, destacando la eliminación de declaraciones del modelo y la consiguiente
simplificación conseguida en la red de Petri final.

\begin{figure}[!htb]
  \centering
  \includesvg[width=0.5\linewidth]{statement-model-comparison.svg}
  \caption{Comparación lado a lado de dos posibilidades para modelar los MIR statements.}
  \label{fig:statement-model-comparison}
\end{figure}

\subsection{Terminators}
\label{sec:terminators}

Como se vio en la Sec. \ref{sec:mir-components}, existen diferentes clases de sentencias terminator.
La documentación del enum \Rustinline{TerminatorKind}\footnote{\url{https://doc.rust-lang.org/stable/nightly-rustc/rustc_middle/mir/enum.TerminatorKind.html}}
enumera, en el momento de redactar este documento, 14 variantes diferentes.
Se requiere que la implementación soporte la mayoría de
ellas, visto que se manifiestan tarde o temprano en los programas de prueba incluidos en el repositorio
y su traducción influye directamente en las conexiones entre los bloques básicos. Los restantes
terminadores que no están implementados no están presentes cuando se consulta el
\Rustinline{optimized_mir}, es decir, sólo se utilizan en pasadas previas del compilador.

La implementación de la MIR Visitor\footnote{\url{https://github.com/hlisdero/cargo-check-deadlock/blob/main/src/translator/mir_visitor.rs}}
incluye el método \Rustinline{visit_terminator} como se ha visto
antes. Aquí es donde se crean las aristas que conectan un \acrshort{BB} con otro.
En los párrafos siguientes se discuten los detalles de alto nivel de cada handler.
Se omiten algunos detalles de implementación ya que no afectan a la red de Petri.

\subsubsection{Goto}

Se trata de un tipo de terminación elemental.
El lugar de finalización del \acrshort{BB} actualmente
activo se conecta con el lugar de inicio del \acrshort{BB} objetivo
a través de una nueva transición con una etiqueta adecuada.

\subsubsection{SwitchInt}

Este tipo de terminador viene con una colección de bloques básicos objetivo.
Para cada \acrshort{BB} objetivo,
conectamos el lugar final del \acrshort{BB} actualmente activo
con el lugar inicial del \acrshort{BB} objetivo
a través de una nueva transición con una etiqueta adecuada.
Esto crea un conflicto tal y como se define en la Sec. \ref{sec:transition-firing}.

La etiqueta también debe contener algún tipo de identificador único del bloque desde el que se
inicia el salto. Se trata de una precondición para traducir correctamente varios bloques
básicos con un \Rustinline{SwitchInt} que salte al mismo bloque.

\subsubsection{Resume o Terminate}

Se trata de terminadores que modelan respectivamente un desenrollado de la pila (\textit{stack unwinding})
y el aborto inmediato del programa.
Ambos se tratan de la misma manera: Conectando el lugar de
finalización del \acrshort{BB} actualmente activo
con el lugar \Rustinline{PROGRAM_PANIC} visto en la Fig. \ref{fig:program-places}.

\subsubsection{Return}

Este es el terminador que provoca el retorno de la función \acrshort{MIR}.
Aquí se utiliza el lugar final de la función.
El lugar de finalización del \acrshort{BB} actualmente activo se conecta a él.

\subsubsection{Unreachable}

Se trata de un caso borde que aparece en algunas \Rustinline{match},
\Rustinline{while} loops, u otras estructuras de control.
La documentación lo indica: \emph{Indicates a terminator that can never be reached}.
Para tratar este caso, se ha optado por conectar el lugar de finalización del \acrshort{BB} activo en ese
momento con el lugar \Rustinline{PROGRAM_END} que se ve en la Fig. \ref{fig:program-places}.
Léase los comentarios en el repositorio para obtener más detalles.

\subsubsection{Drop}

El trait \Rustinline{std::ops::Drop} se utiliza para especificar el código que debe ejecutarse cuando el
tipo sale de \textit{scope} \cite[Chap. 15.3]{rust-book}. Es equivalente al concepto de
destructores que se encuentra en otros lenguajes de programación.

El terminador de \Rustinline{Drop} se comporta como una llamada de función con una transición de
limpieza. Por lo tanto, aplicamos el modelo mostrado en la Fig. \ref{fig:function-with-cleanup}
con etiquetas de transición modificadas.

Aquí también se produce una comprobación importante, a saber, la comprobación de si se está
haciendo \Rustinline{Drop} de una \textit{mutex guard}.
Si es así, el mutex correspondiente debe desbloquearse como
parte del disparo de la transición. Los detalles precisos se explican en la Sec. \ref{sec:mutex-algorithms}.

\subsubsection{Call}

Este es el tipo de terminador para ejecutar llamadas a funciones. La presencia de un bloque de
limpieza y la \Rustinline{UnwindAction}\footnote{\url{https://doc.rust-lang.org/stable/nightly-rustc/rustc_middle/mir/syntax/enum.UnwindAction.html}}
particular así como el nombre y el tipo de la función se analizan
para manejarla según la estrategia elaborada en la Sec. \ref{sec:function-calls}.

Nótese que \Rustinline{UnwindAction} es una refactorización de \emph{rustc}
que se introdujo el 7 de abril de 2023.
Es un buen ejemplo de una regresión que requirió cambios significativos para
adaptarse.
Se remite al lector interesado al commit\footnote{\url{https://github.com/hlisdero/cargo-check-deadlock/commit/8cf95cd54b29c210801cae2941abcbbb85051b92}}
correspondiente.

\subsubsection{Assert}

Este tipo de terminador está relacionado con la macro \Rustinline{assert!()}\footnote{\url{https://doc.rust-lang.org/std/macro.assert.html}}
y las comprobaciones de desbordamiento (\textit{overflow}) por defecto que
\emph{rustc} incorpora al realizar operaciones aritméticas.

La implementación no modela la condición para el assert. Simplemente conecta el lugar
final del \acrshort{BB} actualmente activo con el lugar inicial del \acrshort{BB} objetivo
a través de una nueva transición con una etiqueta adecuada.

En algunos casos, también hay un bloque de limpieza. Para ello, se necesita una segunda
transición, de forma análoga al caso del \Rustinline{Drop}.