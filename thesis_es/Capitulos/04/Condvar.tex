\section{Condition variable (\texttt{std::sync::Condvar})}

Una condition variable es una primitiva de sincronización que
permite que uno o más hilos esperen una determinada condición antes de proceder con
su ejecución. Los hilos esperan hasta que otro hilo les notifica que se ha cumplido la
condición deseada.

Las variables de condición suelen estar asociadas a un mutex que garantiza el acceso exclusivo
a los datos compartidos de los que depende la condición.
Cuando un hilo espera en una condition variable,
libera el mutex asociado, permitiendo que otros hilos avancen.
Cuando la condición se convierte en verdadera o se produce algún evento,
un hilo notificador señala la variable de condición,
permitiendo que uno o más hilos en espera reanuden su ejecución.

La semántica de las variables de condición, así como ejemplos en pseudocódigo, se
introdujeron en la Sec. \ref{sec:condition-variables}.
La comprensión del comportamiento preciso de las condition variables
en todas las circunstancias es un requisito previo para esta sección.

Esta sección ofrece una explicación detallada del modelo de red de Petri utilizado para representar las
condition variable de la biblioteca estándar de Rust. Le sigue un ejemplo práctico que
pretende aumentar la claridad de los conceptos. Por último, se esbozan los algoritmos para la
traducción de funciones de variables de condición.

\subsection{Modelo de red de Petri}

En este caso concreto, el modelo de red de Petri debe examinarse detenidamente
puesto que implica no sólo a
la propia variable de condición,
sino también a la variable que mantiene la condición sobre la
que espera el hilo bloqueado \emph{y} al mutex que sincroniza el acceso a dicha condición.

En general, esta interacción puede ser extremadamente compleja.
La misma variable de condición podría utilizarse
para señalar o rastrear un número arbitrario de condiciones distintas.
En consecuencia, pueden pasarse diferentes mutexes como argumento a la llamada de
\Rustinline{wait}.
Además, un número arbitrario de hilos puede bloquearse en una variable de condición y
Rust admite la operación de \emph{broadcast} para despertar a todos los hilos en espera a la vez
mediante el método \Rustinline{notify_all}\footnote{\url{https://doc.rust-lang.org/std/sync/struct.Condvar.html\#method.notify_all}}
(véase la Sec. \ref{sec:condition-variables}).
Y lo que es más importante, la condición
en sí puede ser de cualquier tipo y tomar una larga secuencia de valores durante la ejecución,
en función de los cuales los hilos en espera podrían actuar de diversas maneras en cada
escenario.

Por todo ello, es inevitable hacer suposiciones sobre los casos de uso soportados
para variables de condición reducir la complejidad de la tarea.
Abarcar y manejar todas las posibilidades queda fuera del alcance de esta tesis.

\subsubsection{Supuestos}

\begin{enumerate}
      \item \emph{Llamada única}:
            Sólo hay una llamada a esperar (\Rustinline{wait}) por variable de condición. Equivalentemente,
            \Rustinline{condvar.wait()} aparece en un único lugar del código fuente para una \Rustinline{condvar} determinada.
            Por ejemplo, puede estar dentro de un bucle pero no puede estar en dos funciones
            diferentes.
      \item \emph{Cola de un solo elemento}: Hay como máximo un hilo de espera por condition variable.
      \item \emph{Condición booleana}: La condición es un flag booleano.
            Está activo o no, dos valores posibles en total.
            Esperar en una condición que puede tomar 3 o más valores \emph{no} es soportado por este modelo.
      \item \emph{Establecimiento obligatorio de la condición / Sin ``notificación falsa''}:
            Si un hilo bloquea el mutex y accede mutablemente a la condición compartida,
            entonces siempre almacena un valor diferente.
            En términos más sencillos, los hilos que miran el valor, no lo cambian
            e inmediatamente llaman \Rustinline{signal} sobre la condition variable no están permitidos.
      \item \emph{Exclusión de broadcast}: El método \Rustinline{std::sync::Condvar::notify_all} está fuera de alcance.
\end{enumerate}

Podría implementarse el soporte para múltiples llamadas a \Rustinline{wait} y múltiples hilos de espera,
pero se requiere un considerable esfuerzo de implementación.
Por lo tanto, los Supuestos 1 y 2 pueden superarse con el modelo propuesto.

Admitir condiciones no booleanas y detectar qué valor toma la condición después de cada acceso
exige reconsiderar a fondo el enfoque de modelado
para representar valores de datos concretos en redes de Petri simples.
En consecuencia, los Supuestos 3 y 4 son especialmente desafiantes y podrían ser objeto de
futuras investigaciones sobre modelos en redes de Petri de nivel superior.
Véase la Sec. \ref{sec:future-work-higher-level-models} para algunas reflexiones en este sentido.

\subsubsection{Análisis del modelo propuesto}

La Fig. \ref{fig:condition-variable-model} muestra el modelo de red de Petri utilizado en la implementación.
El mismo diagrama en formato DOT, PNG y SVG puede encontrarse en el repositorio como documentación.

\begin{figure}[!htbp]
      \centering
      \includesvg[width=\linewidth]{condition-variable-model.svg}
      \caption{El modelo de red de Petri para las condition variables.}
      \label{fig:condition-variable-model}
\end{figure}

Los lugares de entrada son:

\begin{itemize}
      \item \Rustinline{input}: El lugar de inicio de la función wait.
            El modelo admite los métodos de la biblioteca estándar \Rustinline{std::sync::Condvar::wait}
            y su variación \Rustinline{std::sync::Condvar::wait_while}.
      \item \Rustinline{condition_not_set}: El lugar está marcado cuando la condición es \Rustinline{false}.
      \item \Rustinline{condition_set}: El lugar está marcado cuando la condición es \Rustinline{true}.
      \item \Rustinline{notify}: El lugar donde el hilo notificador coloca un token para despertar al hilo en espera.
\end{itemize}

El lugar de salida (\Rustinline{output}) es
el lugar de finalización de la llamada
a la función \Rustinline{wait} o \Rustinline{wait_while}.
La ejecución del hilo continúa a partir de ahí.

Existen dos formas posibles de pasar de \Rustinline{input} a \Rustinline{output}, representadas por dos transiciones:

\begin{itemize}
      \item \Rustinline{wait_start}: Este es el ``caso común'', el hilo se bloquea y espera la señal.
      \item \Rustinline{wait_skip}: Este es el camino alternativo que toma el hilo cuando la condición ya se ha cumplido.
            El hilo no espera, en su lugar, se salta la espera y alcanza la salida \Rustinline{output} en un solo salto.
\end{itemize}

Es esencial observar que la parte en gris
de la izquierda de la Fig. \ref{fig:condition-variable-model}
controla qué transición se activa y qué transición se desactiva.
En cuanto se establece un testigo en \Rustinline{condition_set}, \Rustinline{wait_start} se desactiva.
Antes de eso, ocurre lo contrario: \Rustinline{wait_start} puede dispararse pero \Rustinline{wait_skip} no.

Nótese los arcos entre \Rustinline{condition_not_set} y \Rustinline{wait_start}.
La ficha se regenera cada vez que se dispara \Rustinline{wait_start}.
Lo mismo ocurre con \Rustinline{condition_set} y \Rustinline{wait_skip}.
Estos arcos restauran los lugares de condición a su estado anterior.
Modelar más de 2 valores como una \acrshort{PN} implicaría
una red más enrevesada, Además, sería imposible saber en tiempo de compilación cuántos
valores posibles toma la condición para generar el número correcto de lugares.
Esta limitación es la justificación de los Supuestos 3 y 4.

Ahora concéntrese en el lado derecho de la Fig. \ref{fig:condition-variable-model}.
En el centro de la variable de condición,
encontramos el lugar para el mutex.
Como era de esperar, se desbloquea cuando comienza la
espera y se bloquea cuando se recibe la notificación.

El lugar etiquetado \Rustinline{wait_enabled} desempeña un papel central.
Por un lado, consume el testigo de \Rustinline{notify} si \Rustinline{wait_start} no se disparó.
Este es el caso arquetípico de señal perdida que nos gustaría detectar.
Por otro lado, el token en \Rustinline{wait_enabled} se consume cuando \Rustinline{wait_start} se dispara.
Esto evita que la condition variable ``acepte'' otros hilos (Supuesto 2) y
conserva el token en \Rustinline{notify}, asegurando que la señal perdida no pueda ocurrir.

Por último, la transición \Rustinline{notify_received} combina los requisitos para que el hilo abandone la
espera: El mutex debe estar desbloqueado y \Rustinline{notify_one} ha sido llamado. Para restaurar el estado
inicial de la condition variable, regenera el testigo en \Rustinline{wait_enabled}.

\subsubsection{Requisitos de traducción a nivel global del modelo de red de Petri}

Un desafío fundamental que surge
durante la implementación del modelo de la Fig. \ref{fig:condition-variable-model}
es que las conexiones a través de la frontera azul del diagrama no pueden establecerse
en el caso general cuando se procesa la llamada a \Rustinline{wait}.
Analizaremos el problema y explicaremos cómo lo afronta la solución.

La transición en la que se establece la condición, denominada \Rustinline{set_condition} en el diagrama,
es la siguiente candidata. En la implementación actual, la transición seleccionada para cumplir este
papel es la llamada a \Rustinline{std::ops::DerefMut::deref_mut} cuando se está desreferenciando un mutex o
una guarda de mutex.

Considere el Listado \ref{lst:set-condition-example},
otro programa de prueba del repositorio.
En la línea 9, la guarda de mutex es dereferenciada para escribirle el valor \Rustinline{true},
cumpliendo la condición para la condition variable.
En la \acrshort{MIR}, esto se corresponde con una llamada a
\Rustinline{std::ops::DerefMut::deref_mut}.
Aunque no es el lugar exacto donde se escribe el valor (que en
realidad sería una sentencia dentro de un \acrshort{BB}),
se aproxima lo suficiente y satisface nuestras necesidades.

\begin{listing}[!htbp]
      \begin{minted}{Rust}
      fn main() {
            let pair = std::sync::Arc::new((std::sync::Mutex::new(false), std::sync::Condvar::new()));
            let pair2 = std::sync::Arc::clone(&pair);
        
            // Inside of our lock, spawn a new thread, and then wait for it to start.
            std::thread::spawn(move || {
                let (lock, cvar) = &*pair2;
                let mut started = lock.lock().unwrap();
                *started = true;
                // We notify the condvar that the value has changed.
                cvar.notify_one();
            });
        
            // Wait for the thread to start up.
            let (lock, cvar) = &*pair;
            let mut started = lock.lock().unwrap();
            while !*started {
                started = cvar.wait(started).unwrap();
            }
      }              
      \end{minted}
      \caption{Un programa que requiere información global de la red de Petri para ser traducido.}
      \label{lst:set-condition-example}
\end{listing}

Lamentablemente, las conexiones con la variable de condición tampoco pueden establecerse al
procesar la llamada a \Rustinline{deref_mut}.
La razón es que no hay ninguna garantía de que la condition variable ya se haya visto.
Todavía podría estar por delante en el camino que sigue la traducción.
En el Listado \ref{lst:set-condition-example}, el hilo principal se traduce primero,
por lo que la variable de condición se descubre primero.
Pero si los papeles de los hilos s e intercambian, entonces la traducción no puede realizarse.

Llegamos así a una conclusión desalentadora. Para conectar el modelo de la variable de
condición con los lugares que modelan la condición y las transiciones en las que se establece,
necesitamos toda la red de Petri o en otras palabras, necesitamos información \emph{global} para traducir eficazmente la
primitiva de sincronización.

Como resultado, es inevitable incorporar algún clase de etapa de postprocesamiento a la
traducción. Las tareas también deben realizarse en un orden específico. Primero debe
descubrirse el mutex. Después puede vinculárselo a una variable de condición si se encuentra tal
condition variable (el código fuente podría no utilizar ninguna).
De ahí que sea aconsejable introducir una noción de ``prioridad'' en las tareas de postprocesamiento.

El \Rustinline{Translator} se basa en un \Rustinline{std::collections::BinaryHeap}
para implementar una cola de prioridad de
\Rustinline{PostprocessingTask}\footnote{\url{https://github.com/hlisdero/cargo-check-deadlock/blob/main/src/translator/function.rs\#L92}}.
Las tareas son devueltas por los métodos que traducen
primitivas de sincronización si es necesario. Una vez traducidos todos los hilos, el \Rustinline{Translator}
aborda las tareas de postprocesamiento. Al completarlas según su prioridad,
garantizamos que la información esté disponible en el orden requerido.

\subsubsection{Table of possible inputs and expected outputs}

Como complemento a la explicación de los apartados anteriores, he aquí una tabla que
resume la salida esperada para una entrada determinada.
El lector puede comparar la Fig. \ref{fig:condition-variable-model}
para comprobar que el modelo produce la salida correcta para cada escenario.

\begin{table}[!htb]
      \centering
      \begin{tabular}{ |c|c|c|c|l| }
            \hline
            Row \# & \multicolumn{3}{|c|}{Input} & Output                                                                                                            \\
            \hline
                   & \Rustinline{condition_set}  & \Rustinline{wait_enabled} & \Rustinline{notify} & \emph{where the initial token at \Rustinline{input} ends}       \\
            \hline
            R1     & False                       & False                     & False               & \Rustinline{waiting} (waiting for a \Rustinline{notify})        \\
            R2     & False                       & False                     & True                & \Rustinline{output} (correct wait end condition)                \\
            R3     & False                       & True                      & False               & \Rustinline{input} (initial state)                              \\
            R4     & False                       & True                      & True                & \emph{lost signal (transient state, goes to R1)}                \\
            R5     & True                        & False                     & False               & \Rustinline{waiting} (condition set, needs \Rustinline{notify}) \\
            R6     & True                        & False                     & True                & \Rustinline{waiting} (correct wait end condition)               \\
            R7     & True                        & True                      & False               & \Rustinline{output} (skip the wait)                             \\
            R8     & True                        & True                      & True                & \Rustinline{output} (skip the wait, with lost signal)           \\
            \hline
      \end{tabular}
      \caption{Resumen de los posibles estados del modelo de red de Petri para las condition variables.}
      \label{table:condition-variable-model-states}
\end{table}

\subsection{Un ejemplo práctico}

Debido a las limitaciones de tamaño de la red de Petri, nos vemos obligados a seleccionar un
programa a pequeña escala con fines de demostración.
Sería inviable incluir en una sola página
la red de Petri completa de un programa realista con condition variables y múltiples hilos.
Para obtener ejemplos más completos, se anima a los lectores a explorar el repositorio,
el cual contiene una colección de programas más intrincados incluidos como parte de las pruebas de integración.

A pesar de las limitaciones de espacio,
el ejemplo del Listado \ref{lst:condition-variable-example} comprende los elementos
centrales del modelo presentado anteriormente.
La red de Petri completa puede verse en la Fig. \ref{fig:condition-variable-example}.

Observe la siguiente secuencia de transiciones:

\begin{enumerate}
      \item El mutex se bloquea en \Rustinline{std_sync_Mutex_T_lock_0_CALL}.
      \item \Rustinline{std_sync_Condvar_notify_one_0_CALL} coloca un token en \Rustinline{CONDVAR_0_NOTIFY}.
      \item El flujo del token continúa a \Rustinline{main_BB5} justo antes de \Rustinline{CONDVAR_0_WAIT_START}.
\end{enumerate}

Cabe destacar que no se produce ningún deadlock si la transición \Rustinline{CONDVAR_0_WAIT_START}
se dispara \emph{antes} de \Rustinline{CONDVAR_0_LOST_SIGNAL}.
Dicho brevemente, existe un conflicto entre las transiciones
\Rustinline{CONDVAR_0_WAIT_START} y \Rustinline{CONDVAR_0_LOST_SIGNAL}
por el token en \Rustinline{CONDVAR_0_NOTIFY}.
No obstante, el verificador de modelos comprueba \emph{todos} los posibles disparos
y descubrirá el caso de la señal perdida sin dificultades.

Otra observación digna de mención es que este programa ilustra
el efecto que tendrían los caminos alternativos
de limpieza en la detección de señales perdidas.
Si hubiera una segunda transición al mismo nivel
que \Rustinline{CONDVAR_0_WAIT_START} o \Rustinline{std_sync_Condvar_notify_one_0_CALL},
la señal podría ``escapar'' al lugar \Rustinline{PROGRAM_PANIC} y
el deadlock sería indetectable para el verificador de modelos.

Es indispensable que la traducción ``obligue'' a la red de Petri a permanecer bloqueada y a no
abrir caminos alternativos que podrían ser utilizados por el verificador de modelos para llegar a
la conclusión de que la red nunca se bloquea.

\begin{listing}[!htbp]
      \begin{minted}{Rust}
      fn main() {
            let mutex = std::sync::Mutex::new(false);
            let cvar = std::sync::Condvar::new();
            let mutex_guard = mutex.lock().unwrap();
            cvar.notify_one();
            let _result = cvar.wait(mutex_guard);
      }     
      \end{minted}
      \caption{Un programa básico para mostrar la traducción de variables de condición.}
      \label{lst:condition-variable-example}
\end{listing}

\begin{figure}[!htbp]
      \centering
      \includesvg[width=\linewidth]{condition-variable-example.svg}
      \caption{El modelo de red de Petri para el programa del Listado \ref{lst:condition-variable-example}.}
      \label{fig:condition-variable-example}
\end{figure}

\subsection{Algoritmos para la traducción de condition variables}
\label{sec:condvar-algorithms}

Para concluir esta sección, presentaremos un resumen conciso de los algoritmos utilizados en la
traducción de condition variables.
Posteriormente se detallan también las adiciones necesarias a los algoritmos mutex.

Cuando se encuentra una llamada a \Rustinline{std::sync::Condvar::new}:

\begin{enumerate}
      \item Traducir la llamada a la función utilizando el modelo visto en la Fig. \ref{fig:function-with-cleanup}.
      \item Crear una nueva estructura \Rustinline{Condvar}\footnote{\url{https://github.com/hlisdero/cargo-check-deadlock/blob/main/src/translator/sync/condvar.rs}}
            con un índice para identificarla inequívocamente en toda la red.
      \item Vincular el valor de retorno de \Rustinline{std::sync::Condvar::new},
            la nueva condition variable, a la estructura \Rustinline{Condvar}.
\end{enumerate}

Cuando se encuentra una llamada a \Rustinline{std::sync::Condvar::notify_one}:

\begin{enumerate}
      \item Traducir la llamada a la función utilizando el modelo visto en la Fig. \ref{fig:simplest-function}.
            Ignorar el lugar de limpieza porque, de lo contrario, cualquier llamada podría fallar, lo que equivaldría a
            que la operación notificar no estuviera presente en el programa, dando lugar a una falsa
            señal perdida. Esto equivale a suponer que la función \Rustinline{notify_one} nunca falla.
      \item Recuperar la auto-referencia \Rustinline{self} a la condition variable sobre la que se llama a la función.
      \item Añadir un arco desde la transición que representa la llamada a la función hasta el lugar de
            notificación de la condition variable referenciada por \Rustinline{self}.
\end{enumerate}

Cuando se encuentra una llamada a \Rustinline{std::sync::Condvar::wait}
o \Rustinline{std::sync::Condvar::wait_while}:

\begin{enumerate}
      \item Ignorar el lugar de limpieza porque, de lo contrario, cualquier llamada puede fallar, lo que
            equivale a que la operación de espera no está presente en el programa, lo que conduce a
            un resultado incorrecto. Debemos obligar a la red a ``esperar'' a que se envíe la señal de
            notificación. Esto equivale a suponer que las funciones \Rustinline{wait} o \Rustinline{wait_while} nunca fallan.
      \item Recuperar la auto-referencia \Rustinline{self} a la condition variable sobre la que se llama a la función.
      \item Extraer la guarda de mutex pasada como argumento a la función.
      \item Si la condition variable ya está conectada a una llamada a wait, la traducción falla.
            Esto hace cumplir los Supuestos 1 y 2 vistos al principio de la sección.
      \item En caso contrario, conectar los lugares de inicio y fin a las transiciones \Rustinline{wait_start}
            y \Rustinline{notify_received} respectivamente.
      \item Vincular el valor de retorno, la misma guarda de mutex que se pasó como argumento, a la \Rustinline{MutexGuard}.
      \item Notificar al traductor que el mutex recibido debe vincularse a este \Rustinline{Condvar}.
            Para este propósito utilizar la variante enum \Rustinline{PostprocessingTask::LinkMutexToCondvar}.
            Esta tarea se procesará después de traducir todos los hilos.
\end{enumerate}

Cuando todos los hilos han terminado de traducir, es decir, cuando la cola de hilos por procesar
está vacía, el \Rustinline{Translator} entra en un loop para completar las tareas de postprocesamiento por
orden de prioridad:

\begin{enumerate}
      \item Crear al principio del loop un vector vacío de referencias a mutex.
      \item Saca del \Rustinline{std::collections::BinaryHeap} la tarea con la prioridad más baja.
            Esta será por diseño una \Rustinline{PostprocessingTask::NewMutex}.
            Añadir la referencia a mutex al vector.
      \item Después de procesar todas las tareas de menor prioridad,
            el \Rustinline{Translator} tiene referencias a
            todos los mutex del código. Continuar sacando tareas de la cola de prioridad.
      \item Eventualmente, una \Rustinline{PostprocessingTask::LinkMutexToCondvar} es extraída.
            Vincular cada mutex a la condition variable,
            lo que crea los lugares \Rustinline{condition_set}
            y \Rustinline{condition_not_set} para la condición en sí.
            También conectar las transiciones \Rustinline{deref_mut}
            a estos lugares para modificar el valor de la condición.
            Por último, conectar los lugares de la condición a las transiciones de la variable de
            condición para desactivar el \Rustinline{wait}.
\end{enumerate}

\subsubsection{Modificaciones en los algoritmos para el mutex}

Como ya se ha dicho, los algoritmos mutex requieren algunas adiciones para realizar con éxito
la detección de señales perdidas.

Agregar lo siguiente al manejador de la función \Rustinline{std::sync::Mutex::new}:

\begin{enumerate}
      \item Notificar al traductor que se ha creado un nuevo mutex.
            Para este propósito utilizar la variante enum \Rustinline{PostprocessingTask::NewMutex}.
            Esta tarea se procesará después de traducir todos los hilos.
\end{enumerate}

Cuando se encuentra una llamada a \Rustinline{std::result::Result::<T, E>::unwrap}:

\begin{enumerate}
      \item Comprobar si la auto-referencia \Rustinline{self} es un mutex o un mutex guard.
      \item Traducir la llamada a la función utilizando el modelo visto en la Fig. \ref{fig:simplest-function}.
            Ignorar el lugar de limpieza porque, de lo contrario, cualquier llamada puede fallar, como si la operación
            de bloqueo del mutex no estuviera presente en el programa, lo que daría lugar a una falsa
            señal perdida. Esto equivale a suponer que la función \Rustinline{unwrap} nunca falla cuando se
            aplica a una variable vinculada a un mutex o a una guarda de mutex.
\end{enumerate}

Cuando se encuentra una llamada a \Rustinline{std::ops::Deref::deref}
o \Rustinline{std::ops::DerefMut::deref_mut}:

\begin{enumerate}
      \item Comprobar si la auto-referencia \Rustinline{self} es un mutex o un mutex guard.
      \item Traducir la llamada a la función utilizando el modelo visto en la Fig. \ref{fig:simplest-function}.
            Ignorar el lugar de limpieza porque, de lo contrario, cualquier llamada puede fallar, como si la operación
            de bloqueo del mutex no estuviera presente en el programa, lo que daría lugar a una falsa
            señal perdida. Esto equivale a suponer que las funciones \Rustinline{deref} y \Rustinline{deref_mut} nunca fallan
            cuando se realiza una desreferencia a una variable vinculada a un mutex o a una guarda
            mutex.
      \item Si el valor está siendo dereferenciado mutablemente (\Rustinline{deref_mut}),
            extraer el primer argumento pasado a la función: El mutex o el mutex guard.
            Añadir la transición \Rustinline{deref_mut} al mutex
            para conectar la condición con una condition variable específica en la etapa de postprocesamiento.
      \item En caso contrario no haga nada. El caso inmutable no necesita ser añadido al mutex.
\end{enumerate}

Ahora debería estar claro para el lector que los algoritmos para la detección de señales
perdidas son fundamentalmente de mayor complejidad y deben manejar más casos borde
que los de detección de simples deadlocks causados por un uso incorrecto de los mutexes o
por llamar a \Rustinline{join} para hilos que nunca terminan.

Cabe mencionar que algunos casos borde surgen debido a la inclusión de la lógica de limpieza
de la \acrshort{MIR} en el modelo \acrshort{PN}.
Si en su lugar la implementación omitiera esto, bajo la hipótesis de
que las funciones de la biblioteca estándar Rust nunca pueden entrar en pánico,
entonces los algoritmos se volverían más sencillos.
La Sec. \ref{sec:future-work-no-cleanup} se ocupa
de la cuestión de no modelar los flujos alternativos de limpieza.