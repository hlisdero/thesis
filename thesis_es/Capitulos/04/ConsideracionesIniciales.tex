\section{Consideraciones iniciales}

\subsection{Lugares básicos de un programa Rust}
\label{sec:basic-places}

El modelo básico de red de Petri para un programa Rust generado por la herramienta puede
verse en la Fig. \ref{fig:program-places}.
El lugar etiquetado \Rustinline{PROGRAM_START} contiene un token y representa el
estado inicial del programa Rust. Este token se ``moverá'' de declaración en declaración y, por
tanto, puede interpretarse como el contador de programa de la \acrshort{CPU}.

Correspondientemente, el lugar etiquetado \Rustinline{PROGRAM_END} modela el estado final del programa
después de la terminación normal del programa, es decir, al volver de la función
\Rustinline{main}, independientemente del código de salida específico.
En otras palabras, una función principal que devuelve un código de error debido a parámetros no válidos o a un error interno del
programa se sigue considerando una terminación ``normal'' del programa.
En otros casos, sin embargo, el programa puede no llegar nunca a este estado si \Rustinline{main} nunca retorna.
Estas se conocen en Rust como ``funciones divergentes''\footnote{\url{https://doc.rust-lang.org/rust-by-example/fn/diverging.html}}
y están soportadas por la herramienta.

Por último, el lugar etiquetado \Rustinline{PROGRAM_PANIC} modela
la terminación anormal del programa,
que ocurre cuando el programa llama a la macro \Rustinline{panic!}.

\begin{figure}[!htb]
  \centering
  \includesvg{program-places.svg}
  \caption{Lugares básicos en todo programa Rust.}
  \label{fig:program-places}
\end{figure}

Se pueden argumentar dos razones para considerar un lugar separado para el estado final de pánico.
En primer lugar, es útil para la verificación formal distinguir el caso de pánico del caso
de terminación normal. Un programa puede entrar en pánico al detectar una posible violación
de sus garantías de seguridad de memoria. En la mayoría de las circunstancias, esta es una
opción más inteligente que simplemente ignorar el error y continuar. Por esta razón estos
programas no deben marcarse en principio como erróneos o defectuosos pero es aconsejable
registrar el estado final a efectos de solución de problemas y depuración. En segundo lugar,
incluso si el código del usuario no recurre a \Rustinline{panic!} como mecanismo de gestión de errores,
numerosas funciones de la biblioteca estándar de Rust pueden entrar en pánico en
circunstancias extraordinarias, por ejemplo, debido a una falta de memoria (\acrfull{OOM}) o a un
error de hardware, o cuando el \acrshort{OS} falla al asignar un nuevo hilo, mutex, etc. En
consecuencia, es esencial capturar este eventual fallo en el modelo \acrshort{PN}.

Hay un último punto sutil que es necesario abordar. El lugar de inicio del programa no es tan
trivial como parece. Aunque la función principal \Rustinline{main} se percibe típicamente como la primera
función que se ejecuta, en realidad no es así. En su lugar, los programas Rust tienen un \textit{runtime}
que se ejecuta antes de llamar a la función principal en el que se inicializan las
características específicas del lenguaje y la memoria estática. Normalmente se oye hablar de
lenguajes interpretados como Java o Python que tienen un tiempo de ejecución pero los
lenguajes de bajo nivel como Rust o C también tienen un pequeño \textit{runtime}.
Simplemente es una capa más fina y menos sofisticada. Para los lectores interesados, recientemente se
presentó en una conferencia sobre Rust un tour de las funciones antes de main \cite{levick2022}.

Teniendo esto en cuenta, nos enfrentamos a la cuestión de si debemos incluir este \textit{runtime}
en la traducción de la red de Petri. Por un lado, el código del tiempo de ejecución forma parte
efectivamente del binario ejecutado por la \acrshort{CPU}. No obstante, es un código dependiente de la
plataforma (el \textit{runtime} es ligeramente diferente para cada \acrshort{OS}) e independiente de
la semántica del programa, es decir, del significado específico del programa que escribió el
usuario. Dado que el usuario no tiene ninguna influencia en esta parte del binario, no se le
pueden atribuir problemas de sincronización. Como tal, este código no añade valor a la
traducción y puede abstraerse de forma segura, reduciendo en el proceso el tamaño de la \acrshort{PN}.
En conclusión, la decisión es omitir el código en tiempo de ejecución; la traducción comienza
en la función \Rustinline{main}.

\subsection{Pasaje de argumentos e introducción de la \textit{query}}

La herramienta está diseñada en torno a una sencilla interfaz de línea de comandos (\acrfull{CLI}).
Tras analizar los argumentos de la línea de comandos utilizando la conocida biblioteca
\texttt{clap}\footnote{\url{https://docs.rs/clap/latest/clap/}}, el
programa introduce una consulta al compilador rustc para iniciar el proceso de traducción.
La mayor parte del trabajo a partir de ese momento está coordinado por la estructura de tipo
\texttt{Translator}\footnote{\url{https://github.com/hlisdero/cargo-check-deadlock/blob/main/src/translator.rs}}.

El sistema de consulta (\textit{query}) se describió brevemente en la Sec. \ref{sec:rustc}.
En la documentación\footnote{\url{https://rustc-dev-guide.rust-lang.org/rustc-driver-interacting-with-the-ast.html}}\footnote{\url{https://rustc-dev-guide.rust-lang.org/rustc-driver-getting-diagnostics.html}}
se proporcionan dos ejemplos del uso de este mecanismo. Han demostrado ser extremadamente
útiles como punto de partida puesto que proporcionan un excelente y breve ejemplo funcional de
cómo interactuar con \emph{rustc}.
En términos más sencillos, son el "¡Hola, mundo!" del trabajo codo con codo con el compilador de Rust.

\subsection{Requisitos de compilación}

Como se mencionó brevemente en la Sec. \ref{sec:rust-nightly}, la herramienta debe compilarse con la versión
nightly de \emph{rustc} para acceder a sus crates y módulos internos. La sección decisiva en el archivo
\emph{lib.rs}\footnote{\url{https://github.com/hlisdero/cargo-check-deadlock/blob/main/src/lib.rs}}
se representa en el Listado \ref{lst:use-rustc-crates}.

\begin{listing}[!htb]
  \begin{minted}[firstnumber=13]{Rust}
    // This feature gate is necessary to access the internal crates of the compiler.
    // It has existed for a long time and since the compiler internals will never be stabilized,
    // the situation will probably stay like this.
    // <https://doc.rust-lang.org/unstable-book/language-features/rustc-private.html>
    #![feature(rustc_private)]
    
    // Compiler crates need to be imported in this way because they are not published on crates.io.
    // These crates are only available when using the nightly toolchain.
    // It suffices to declare them once to use their types and methods in the whole crate.
    extern crate rustc_ast_pretty;
    extern crate rustc_const_eval;
    extern crate rustc_driver;
    extern crate rustc_error_codes;
    extern crate rustc_errors;
    extern crate rustc_hash;
    extern crate rustc_hir;
    extern crate rustc_interface;
    extern crate rustc_middle;
    extern crate rustc_session;
    extern crate rustc_span;
  \end{minted}
  \caption{Extracto del archivo \emph{lib.rs} que muestra cómo utilizar las funciones internas de \emph{rustc}.}
  \label{lst:use-rustc-crates}
\end{listing}

\Rustinline{rustc_private} es una \textit{feature flag} que controla el acceso a los crates privados del
compilador. Estos crates no se instalan por defecto al instalar la cadena de herramientas de Rust
utilizando \emph{rustup}\footnote{\url{https://rustup.rs/}}.
Por esta razón, es necesario instalar los componentes adicionales \emph{rustc-dev},
\emph{rust-src} y \emph{llvm-tools-preview}.
El propósito de cada componente se detalla en \cite{rustup-book}.
En el \texttt{README}\footnote{\url{https://github.com/hlisdero/cargo-check-deadlock/blob/main/README.md}}
del repositorio se hallan instrucciones fáciles de seguir para instalar el ambiente de desarrollo.

Que este autor sepa, no existe un método alternativo para acceder a las partes internas del
compilador de Rust. Herramientas como
Clippy\footnote{\url{https://github.com/rust-lang/rust-clippy/blob/master/rust-toolchain}} y
Kani \footnote{\url{https://github.com/model-checking/kani/blob/main/rust-toolchain.toml}} o
kernels como Redox\footnote{\url{https://gitlab.redox-os.org/redox-os/redox/-/blob/master/rust-toolchain.toml}} y
RustyHermit\footnote{\url{https://github.com/hermitcore/rusty-hermit/blob/master/rust-toolchain.toml}}
también utilizan este mecanismo.