\section{Mutex (\texttt{std::sync::Mutex})}

Un mutex, abreviatura de exclusión mutua, es un mecanismo de sincronización utilizado para
controlar el acceso a un recurso compartido en un programa concurrente. Permite que múltiples
hilos accedan al recurso compartido recurso de forma mutuamente excluyente,
asegurando que sólo un hilo pueda acceder al recurso a la vez.

En esta sección, se explica el modelo de red de Petri para un mutex en Rust, después se presenta un
ejemplo práctico para facilitar la comprensión y, por último, se esbozan los algoritmos
utilizados para la traducción de funciones mutex.

\subsection{Modelo de red de Petri}

En Rust, un mutex se crea envolviendo los datos compartidos en un tipo \Rustinline{Mutex<T>},
donde \Rustinline{T} es el tipo del recurso compartido.
El tipo \Rustinline{std::sync::Mutex} expone un método llamado \Rustinline{lock} para
adquirir el bloqueo del recurso compartido. Si el mutex está actualmente desbloqueado, el
hilo adquiere con éxito el lock y puede proceder a acceder al recurso. Si el mutex ya
está bloqueado por otro hilo, el hilo que intente adquirir el lock se bloqueará hasta que
el lock esté disponible. El método \Rustinline{lock} devuelve una guarda de mutex
(\Rustinline{std::sync::MutexGuard}) que garantiza el acceso exclusivo al recurso hasta que se libere.

A diferencia de la semántica de \Rustinline{unlock} presente en C o C++, el mutex incluido por la
biblioteca estándar de Rust se desbloquea implícitamente, es decir, sin llamar a una función.
El mutex implementa la Adquisición de Recursos es Inicialización (\acrfull{RAII}) y libera el bloqueo
automáticamente cuando sale de scope, evitando los deadlocks. Alternativamente,
liberar una variable local de tipo \Rustinline{std::sync::MutexGuard} equivale a desbloquear el mutex
correspondiente.

Un mutex puede modelarse en redes de Petri como un único lugar que representa el estado del mutex,
indicando si está bloqueado o desbloqueado. El lugar se etiqueta para reflejar su propósito
como mutex. Además, el lugar se marca con un token inicialmente para significar que el mutex
comienza en el estado desbloqueado.

Las transiciones que bloquean el mutex consumen el token del lugar del mutex. Si el token
está ausente, la transición no puede dispararse. El mutex debe estar en estado desbloqueado
para activar la transición de bloqueo que es el comportamiento deseado.

Las transiciones que desbloquean el mutex producen como salida un token en el lugar del mutex. La
transición puede dispararse mientras el programa llegue a ese punto de la ejecución. Después
de que la transición se dispare el lugar del mutex vuelve a contener un token que puede ser
consumido por una transición de bloqueo. Dos tipos de transiciones pueden desbloquear el
mutex:

\begin{enumerate}
      \item Un terminador \Rustinline{Drop} (Sec. \ref{sec:terminators})
            cuando el lugar liberado es de tipo \Rustinline{std::sync::MutexGuard}.
      \item La transición para una llamada a \Rustinline{std::mem::Drop} que libera la memoria ocupada por el
            valor pasado como argumento explícitamente.
\end{enumerate}

Al conectar el lugar del mutex con las transiciones de bloqueo y desbloqueo mediante arcos de
entrada y salida, establecemos la relación entre el estado del mutex y las acciones que lo
manipulan. Este enfoque de modelado permite representar el comportamiento del mutex en una
\acrshort{PN} y facilita el análisis de sus interacciones con otras partes del sistema.

El modelo de red de Petri presentado aquí es bien conocido en la literatura y se ha aplicado con éxito en
otras herramientas. Puede encontrarse, entre otros, en \cite{kavi2002modeling,moshtaghi2001,meyer2020,zhang2022deadlocks}.

\subsection{Un ejemplo práctico}

Considere el modelo de red de Petri mostrado en la Fig. \ref{fig:mutex-example}
correspondiente al programa del Listado \ref{lst:double-lock-deadlock}.
El \acrshort{MIR} se ilustra en \ref{lst:double-lock-deadlock-mir}.
Este programa de prueba es uno de los ejemplos incluidos en el repositorio.

Observe que hay dos transiciones de bloqueo que corresponden a las dos llamadas a \Rustinline{lock} en
el código fuente. Los índices reflejan el orden de su aparición en el programa, lo
que explica las etiquetas \Rustinline{std_sync_Mutex_T_lock_0_CALL} y \Rustinline{std_sync_Mutex_T_lock_1_CALL}.
Ambas tienen un arco de entrada desde el lugar del mutex \Rustinline{MUTEX_0}.

Como ya se ha mencionado, los terminadores \Rustinline{Drop} pueden desbloquear un mutex.
No importa si fallan o no (el caso de error incluye el sufijo \Rustinline{_UNWIND}),
un arco saliente fluye de vuelta al lugar del mutex para reponer el token.

Cabe destacar que hay más arcos entrantes al lugar mutex que salientes, lo que pone
de relieve la importancia de seguir las guardas de mutex a lo largo de la \acrshort{MIR} utilizando la
estrategia explicada en la Sec. \ref{sec:intercepting-assignments}.

\begin{figure}[!htbp]
      \centering
      \includesvg[width=0.95\linewidth]{mutex-example.svg}
      \caption{El modelo de red de Petri para el programa del Listado \ref{lst:double-lock-deadlock}.}
      \label{fig:mutex-example}
\end{figure}

\subsection{Algoritmos para la traducción del mutex}
\label{sec:mutex-algorithms}

Para concluir esta sección, ofreceremos un breve resumen de los algoritmos empleados en la
traducción de funciones mutex.

Cuando se encuentra una llamada a \Rustinline{std::sync::Mutex::new}:

\begin{enumerate}
      \item Traducir la llamada a la función utilizando el modelo visto en la Fig. \ref{fig:function-with-cleanup}.
      \item Crear una nueva structura \Rustinline{Mutex}\footnote{\url{https://github.com/hlisdero/cargo-check-deadlock/blob/main/src/translator/sync/mutex.rs}}
            con un índice para identificarlo inequívocamente en toda la \acrshort{PN}.
      \item Vincular el valor de retorno de \Rustinline{std::sync::Mutex::new},
            el nuevo mutex, a la estructura \Rustinline{Mutex}.
\end{enumerate}

Cuando se encuentra una llamada a \Rustinline{std::sync::Mutex::lock}:

\begin{enumerate}
      \item Traducir la llamada a la función utilizando el modelo visto en la Fig. \ref{fig:simplest-function}.
            Ignorar el lugar de limpieza ya que debemos obligar a la red de Petri a ``esperar'' a que se marque el lugar mutex.
            Esto equivale a suponer que la función \Rustinline{lock} nunca falla.
      \item Recuperar la auto-referencia \Rustinline{self} al mutex sobre el que se llama a la función.
      \item Añadir un arco desde el lugar del mutex subyacente hacia la transición que representa la llamada a la función.
      \item Crear una nueva \Rustinline{MutexGuard} con una referencia al \Rustinline{Mutex}.
      \item Vincular el valor de retorno de \Rustinline{std::sync::Mutex::lock}, la nueva guarda de mutex,
            a la estructura \Rustinline{MutexGuard}.
\end{enumerate}

Cuando se encuentra una llamada a \Rustinline{std::mem::drop}:

\begin{enumerate}
      \item Traducir la llamada a la función utilizando el modelo visto en la Fig. \ref{fig:function-with-cleanup}.
      \item Extraer el argumento pasado a la función.
      \item Si el argumento está vinculado a una guarda de mutex, añadir un arco desde la transición de
            la llamada a la función hasta el lugar del mutex referenciado por la guarda de mutex.
      \item Si se proporcionó un lugar de limpieza, añadir también un arco de desbloqueo desde la
            transición de limpieza al lugar del mutex referenciado por la guarda de mutex.
\end{enumerate}

Cuando se encuentra un terminador del tipo \Rustinline{rustc_middle::mir::TerminatorKind::Drop}:

\begin{enumerate}
      \item Si la variable a liberar está vinculada a una guarda de mutex, añadir un arco desde
            la transición de la llamada a la función hasta el lugar del mutex referenciado por la guarda de mutex.
      \item Si se proporcionó un lugar de limpieza, añadir también un arco de desbloqueo desde la
            transición de limpieza al lugar del mutex referenciado por la guarda de mutex.
\end{enumerate}

En la próxima sección profundizaremos en los ajustes necesarios de estos algoritmos para
establecer un modelo unificado para las variables de condición que es esencial para detectar las
señales perdidas. Dado que estas modificaciones se comprenden mejor en el marco de las
variables de condición, las dilucidaremos en ese contexto específico.