\section{Llamadas a funciones}
\label{sec:function-calls}

\subsection{La pila de llamadas (\emph{The call stack})}

Un programa en Rust se compone, como en otros lenguajes de programación, de funciones. El
programa comienza (salvo las advertencias vistas en la Sec. \ref{sec:basic-places}) con una llamada a la función
\Rustinline{main} que luego puede llamar a otras funciones. Cabe destacar que las llamadas a funciones
pueden situarse en cualquier punto del código. Una función puede ser llamada desde otra
función o incluso desde dentro de sí misma, dando lugar a llamadas recursivas.

Las llamadas a funciones se almacenan en memoria en una estructura de datos llamada \emph(pila de
llamadas) (\emph{call stack}). Cuando se llama a una función en Rust, ésta se introduce en la pila de llamadas,
creando un nuevo registro de activación (\textit{stack frame}). Un \textit{stack frame} contiene información importante como las
variables locales de la función, los argumentos y la dirección de retorno que indica dónde debe
reanudarse el programa una vez que la función finaliza su ejecución.

La pila de llamadas funciona según el principio de último en entrar, primero en salir (\acrfull{LIFO}).
A medida que se llaman funciones, cada nuevo registro de activación se coloca encima del anterior.
Esto permite que el programa ejecute primero la función llamada más recientemente. Una vez
que una función completa su ejecución, se retira de la pila y el programa continúa desde el
punto en que lo dejó la función anterior.

Por consiguiente, la pila de llamadas desempeña un papel esencial en la gestión de las llamadas y
retornos de funciones porque realiza un seguimiento del flujo de llamadas a funciones y
mantiene la información necesaria para que el programa vuelva al punto de ejecución correcto
después de que una función complete su tarea.

Por las mismas razones, reflejar la pila de llamadas en el traductor es el enfoque más
adecuado para el seguimiento de las llamadas a funciones que deben traducirse, puesto que se
alinea con el flujo lógico de la ejecución del programa. A medida que se traducen las
funciones, se agregan y se sacan de la pila de llamadas del \Rustinline{Translator}, imitando el orden en que
se llaman en tiempo de ejecución. Esto nos permite manejar las invocaciones de funciones
anidadas y seguir el flujo de control de una función a otra durante el proceso de traducción.

\subsection{Funciones MIR}

En la implementación, el \Rustinline{Translator} tiene un stack que soporta las operaciones habituales \Rustinline{push},
\Rustinline{pop} y \Rustinline{peek}.
Esta pila almacena estructuras de tipo \Rustinline{MirFunction}\footnote{\url{https://github.com/hlisdero/cargo-check-deadlock/blob/main/src/translator/mir_function.rs}}.
Más adelante veremos que no todas las funciones se traducen como funciones MIR,
dado que no todas las funciones tienen una representación en \acrshort{MIR} y,
en otros casos, es conveniente manejarlas de otro modo.
No obstante, las funciones \acrshort{MIR} son el ``caso común'' en el proceso de traducción,
el caso por defecto para la mayoría de las funciones definidas por el usuario.

La interfaz disponible proporcionada por \emph{rustc} permite consultar (\textit{query}) el cuerpo \acrshort{MIR} de una sola función
a la vez, lo que puede hacerse utilizando el
\Rustinline{optimized_mir}\footnote{\url{https://doc.rust-lang.org/stable/nightly-rustc/rustc_middle/ty/context/struct.TyCtxt.html\#method.optimized\_mir}}.
Esto implica que no es posible obtener inicialmente el \acrshort{MIR} de todo el programa y que el traductor debe
obtener el \acrshort{MIR} de cada función a medida que las descubre en el código.
Pero, ¿cómo identificar cada función? Se sabe por experiencia que las funciones de módulos distintos
pueden tener el mismo nombre, lo que hace que el nombre no sea adecuado como
identificador. Por suerte, este problema ya está resuelto en el compilador. Las funciones se
identifican unívocamente mediante el tipo del compilador
\Rustinline{rustc_hir::def_id::DefId}\footnote{\url{https://doc.rust-lang.org/stable/nightly-rustc/rustc_hir/def_id/struct.DefId.html}}.
Este ID es válido para el crate que se está compilando en ese momento y ya está presente en el \acrshort{HIR}.
El algoritmo de alto nivel puede describirse como sigue.

Cuando comienza la traducción:

\begin{enumerate}
    \item Consultar el id del punto de entrada del programa (la función principal, \Rustinline{main}).
    \item Crear una \Rustinline{MirFunction} con la información necesaria.
    \item Agregarlo a la pila.
    \item Si es necesario, modificar el contenido de la función \acrshort{MIR} mediante \Rustinline{peek}.
    \item Traducir el elemento superior de la pila de llamadas.
    \item Cuando \Rustinline{main} termine, eliminarlo (\Rustinline{pop}) de la pila de llamadas.
\end{enumerate}

Cuando se descubre un terminator de tipo ``call'' (véase la Sec. \ref{sec:mir-components}):

\begin{enumerate}
    \item Consultar el id de la función llamada.
    \item Crear una \Rustinline{MirFunction} con la información necesaria.
    \item Agregarlo a la pila.
    \item Si es necesario, modificar el contenido de la función \acrshort{MIR} mediante \Rustinline{peek}.
    \item Traducir el elemento superior de la pila de llamadas.
    \item Cuando la función termine, eliminarlo (\Rustinline{pop}) de la pila de llamadas.
\end{enumerate}

Como se ha visto, el enfoque es consistente para cada función \acrshort{MIR} y,
en consecuencia, es más fácil de implementar.

El uso de un call stack en el proceso de traducción permite el cambio de contexto
(\textit{context switching}) entre funciones \acrshort{MIR}
y facilita la posibilidad de volver al bloque básico específico desde el que se
llamó a una función. Esto posibilita que la traducción del programa se realice función por
función de forma lineal, asegurando que se mantienen la estructura y el orden del programa
original.

Sin embargo, el enfoque de la pila de llamadas conlleva ciertas limitaciones.
En primer lugar, si la misma función se llama varias veces dentro del programa,
se traducirá asimismo varias veces.
Esto está relacionado con la estrategia de inlining elaborada en la Sec. \ref{sec:function-inlining}.
Aunque esto puede potencialmente mitigarse mediante el uso de algún tipo de caché, está fuera del alcance de esta
tesis. Esta optimización se discutirá en la Sec. \ref{sec:future-work-function-cache}.

La implicación más grave de utilizar el enfoque de pila de llamadas es la incapacidad de
manejar funciones recursivas. Cuando se procesa una función recursiva durante la traducción,
el proceso queda atrapado en un bucle sin fin en el que la pila crece
indefinidamente a medida que se introducen en ella nuevos stack frames,
lo que provoca un desbordamiento de la pila (\textit{stack overflow})
y la consiguiente interrupción del proceso de traducción. Este problema
también se aborda en la Sec. \ref{sec:future-work-recursion}.
Por ahora, es necesario aceptar la limitación de que las
funciones recursivas infinitas no pueden traducirse utilizando este marco teórico.

\subsection{Funciones foráneas y funciones de la biblioteca estándar}

En Rust, el compilador incluye por defecto la biblioteca estándar en todos los binarios
compilados, enlazándola de forma estática. Para anular este
comportamiento, se utiliza el atributo a nivel de crate \Rustinline{#![no_std]} para indicar que el crate
enlazará con el core-crate en lugar de con el std-crate. Consulte \cite{embedded-book} para más detalles.

Esto significa que la funcionalidad de la biblioteca estándar se convierte en parte
integrante del ejecutable resultante. Las llamadas a funciones de la biblioteca estándar
aparecen en varios contextos en el código Rust, como cuando se accede a argumentos
de la línea de comandos, se invocan iteradores,
se utilizan traits como \Rustinline{std::clone::Clone}, \Rustinline{std::deref::Deref::deref},
o se emplean tipos de la biblioteca estándar como
\Rustinline{std::result::Result} o \Rustinline{std::option::Option}.
Dada la prevalencia de estas llamadas a
funciones en todos los programas Rust, resulta esencial manejarlas por separado en el
proceso de traducción. Es evidente que estas funciones de biblioteca estándar, debido a su
propósito, no pueden conducir a un deadlock. Por lo tanto, es más práctico tratarlas
como cajas negras dentro del proceso de traducción, obviando la necesidad de traducir su
\acrshort{MIR}. Este enfoque es indispensable para evitar generar una red de Petri excesivamente
grande y enrevesada que dificulte la legibilidad y la comprensión.

El esfuerzo de traducción se centra principalmente en el código de usuario, concretamente en
las funciones que los desarrolladores escriben para implementar sus funcionalidades deseadas.
Al dirigir la atención al código de usuario y excluir la traducción de las funciones de la
biblioteca estándar, la red de Petri resulta más manejable, lo que facilita el
análisis y la verificación de posibles deadlock dentro de la base de código del usuario.
Las llamadas a la biblioteca estándar constituyen, en otras palabras, la ``frontera'' o el ``límite'' de la
traducción, el punto en el que dejamos de traducir el \acrshort{MIR} con precisión y nos basamos en
cambio en un modelo simplificado.

\subsubsection{Modelo de red de Petri para una función con bloque de limpieza}

El modelo presentado en la Fig. \ref{fig:simplest-function} es la primera aproximación.
Sin embargo, existe un detalle de implementación que requiere especial atención.
Diversas funciones de la biblioteca estándar contienen no sólo un lugar final
(``target block'', en la jerga de \emph{rustc}) sino
también un lugar de limpieza (``cleanup block'').
Esta segunda vía de ejecución se toma
cuando la función entra explícitamente en pánico o, más genéricamente, no consigue su
objetivo por cualquier motivo. En este caso, el flujo de control continúa hacia un
bloque básico diferente donde se liberan las variables y finalmente el programa termina con un
código de error de pánico. Dicho de otro modo, el desenrollado de la pila comienza en cuanto
una función encuentra un fallo no recuperable.

Teniendo en cuenta que el traductor no puede saber si esta situación anómala podría provocar
un deadlock más adelante en el proceso de traducción, es indispensable traducir esta ruta de
ejecución alternativa siempre que sea posible. Sólo en contadas excepciones, todas ellas
relacionadas con las primitivas de sincronización y tratadas en las secciones respectivas, se
ignora explícitamente este bloque de limpieza. El modelo completo para una llamada a función
abreviada con un bloque de limpieza puede verse en la Fig. \ref{fig:function-with-cleanup}.

\begin{figure}[!htb]
    \centering
    \includesvg[width=\linewidth]{function-with-cleanup.svg}
    \caption{El modelo de red de Petri para una función con un bloque de limpieza.}
    \label{fig:function-with-cleanup}
\end{figure}

\subsubsection{Funciones traducidas con el modelo de red de Petri abreviado}

Tras haber discutido la exclusión de las funciones de biblioteca estándar del proceso de
traducción, ahora nos centraremos en las funciones que sí requieren traducción utilizando el
modelo que hemos presentado antes. Sorprendentemente, incluyen un número considerable de
funciones.

\begin{itemize}
    \item Funciones que forman parte de la biblioteca estándar (std-crate\footnote{\url{https://doc.rust-lang.org/std/}}),
          salvo por la función \Rustinline{std::sync::Condvar::wait} detallada en la Sec. \ref{sec:condvar-algorithms}.
    \item Funciones que forman parte de la biblioteca core (core-crate\footnote{\url{https://doc.rust-lang.org/core/}}).
    \item Funciones en el crate alloc: the core allocation and collections library\footnote{\url{https://doc.rust-lang.org/alloc/}}.
    \item Funciones sin representación \acrshort{MIR}.
          Esto puede comprobarse con el método
          \Rustinline{is_mir_available}\footnote{\url{https://doc.rust-lang.org/stable/nightly-rustc/rustc_middle/ty/context/struct.TyCtxt.html\#method.is\_mir\_available}}.
    \item Funciones externas, es decir, importadas mediante \Rustinline{extern { ... }}.
          Esto puede comprobarse con el método \Rustinline{is_foreign_item}\footnote{\url{https://doc.rust-lang.org/stable/nightly-rustc/rustc_middle/ty/context/struct.TyCtxt.html\#method.is\_foreign\_item}}.
\end{itemize}

En el futuro, las llamadas a funciones en dependencias, es decir, en otros crates, también
deberán tratarse de este modo. En conclusión, el caso por defecto para las funciones que \emph{no}
fueron definidas por el usuario es tratarlas como una función foránea y utilizar un modelo de red
de Petri abreviado para traducirlas.

\subsection{Funciones divergentes}

Las funciones divergentes son un caso especial relativamente fácil de soportar. Se trata
simplemente de una función que nunca vuelve a la función que la llamó. Ejemplos de ello son un
\textit{wrapper} alrededor de un bucle \Rustinline{while} infinito, una función que sale del proceso o una función
que inicia un \acrshort{OS}. Basta con conectar el lugar de inicio de la función a una transición de
sumidero (Definición \ref{definition:sink-transition}) como se ve en la Fig. \ref{fig:diverging-function}.

\begin{figure}[!htb]
    \centering
    \includesvg[width=0.5\linewidth]{diverging-function.svg}
    \caption{El modelo de red de Petri para una función divergente (una función que no retorna).}
    \label{fig:diverging-function}
\end{figure}

Nótese que este caso especial no constituye un deadlock y no debe tratarse
como tal. Un bucle infinito, es decir, un ``busy wait'', es en su naturaleza inherente
distinto de la espera infinita que caracteriza un deadlock como se vio en la Sec. \ref{sec:coffman-conditions}.
En otras palabras, detectar bucles infinitos se acerca más al problema de detectar livelocks que
están fuera del alcance de esta tesis. Además, el traductor no puede saber de antemano si la
llamada divergente es benigna como una llamada a \Rustinline{std::process::exit} o una llamada a algún
tipo de función especial y cuidadosamente diseñada para bloquear el programa.

En el modelo \acrshort{PN} actual, el token se consume y la red queda en un estado final sin tokens en los
lugares \Rustinline{PROGRAM_END} o \Rustinline{PROGRAM_PANIC} mostrados en la Fig. \ref{fig:program-places}. En consecuencia, el
verificador del modelo es capaz de distinguir este estado final de los demás casos y concluir
que se ha llamado a una función divergente.

\subsection{Llamadas explícitas de pánico}

La macro \Rustinline{panic!} puede verse como un caso especial de función divergente en el que la transición que
representa la llamada a la función está conectada al lugar etiquetado \Rustinline{PROGRAM_PANIC} descrito en la Sec.
\ref{sec:basic-places}. El traductor detecta una llamada explícita de pánico que es una de las siguientes funciones:

\begin{itemize}
    \item \Rustinline{core::panicking::assert_failed}
    \item \Rustinline{core::panicking::panic}
    \item \Rustinline{core::panicking::panic_fmt}
    \item \Rustinline{std::rt::begin_panic}
    \item \Rustinline{std::rt::begin_panic_fmt}
\end{itemize}

La documentación\footnote{\url{https://rustc-dev-guide.rust-lang.org/panic-implementation.html}}
detalla por qué se define el pánico en el core-crate y en el std-crate y cómo se implementa.

Véase el Listado \ref{lst:panic} para un programa simple que entra en pánico. El modelo de red de Petri
correspondiente se representa en la Fig. \ref{fig:panic}. Este es uno de los ejemplos ilustrativos incluidos
en el repositorio.

\begin{listing}[!htb]
    \begin{minted}{Rust}
        fn main() {
            panic!();
        }        
    \end{minted}
    \caption{Un programa sencillo en Rust que llama \Rustinline{panic!}.}
    \label{lst:panic}
\end{listing}

\begin{figure}[!htb]
    \centering
    \includesvg[width=0.5\linewidth]{panic.svg}
    \caption{El modelo de red de Petri para el Listado \ref{lst:panic}.}
    \label{fig:panic}
\end{figure}