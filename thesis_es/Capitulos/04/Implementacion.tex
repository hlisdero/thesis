Este capítulo está dedicado a explorar los detalles de implementación de la herramienta de
detección de deadlocks. Su propósito es proporcionar una visión de alto nivel del código y las
estructuras de datos. También se examinan las decisiones de implementación más
importantes tomadas a lo largo del proceso de desarrollo.

En las secciones sucesivas, describiremos los componentes centrales de la herramienta de
detección de deadlocks, incluida la representación interna de la pila de llamadas, el
modelo de memoria de funciones y la traducción de cada componente de una función \acrshort{MIR}.

Más adelante, una parte importante de la discusión se dedica a explicar el soporte del multihilo
y el modelado de las primitivas de sincronización como redes de Petri. Su implementación
requirió cuidadosas consideraciones de diseño para garantizar la correctitud y la eficiencia.

La herramienta soporta actualmente las siguientes estructuras de la biblioteca estándar de Rust
para sincronizar el acceso a los recursos compartidos y proporcionar comunicación entre hilos:

\begin{itemize}
  \item mutexes (\Rustinline{std::sync::Mutex}\footnote{\url{https://doc.rust-lang.org/std/sync/struct.Mutex.html}}),
  \item condition variables (\Rustinline{std::sync::Condvar}\footnote{\url{https://doc.rust-lang.org/std/sync/struct.Condvar.html}}),
  \item atomic reference counters (\Rustinline{std::sync::Arc}\footnote{\url{https://doc.rust-lang.org/std/sync/struct.Arc.html}}).
\end{itemize}

Aunque se cubren los detalles principales, este capítulo no pretende sustituir a la
documentación en el código. La documentación del código en forma de comentarios, pruebas
unitarias y pruebas de integración proporciona información exhaustiva sobre los detalles de
bajo nivel y el uso de la herramienta. Como ya se ha indicado, el repositorio está disponible
públicamente en
GitHub\footnote{\url{https://github.com/hlisdero/cargo-check-deadlock}}\footnote{\url{https://github.com/hlisdero/netcrab}}.