\section{Multihilo}

El soporte multihilo es un requisito previo para la detección de puntos muertos y señales
perdidas. Para soportar programas del mundo real en los que los bloqueos o las señales
perdidas son posibles en primer lugar, resulta esencial soportar la existencia de varios hilos que
compartan recursos. En primer lugar, se presentarán los fundamentos para después idear un
modelo \acrshort{PN} que capture el comportamiento de los hilos en el código Rust.

\subsection{Vida útil del hilo en Rust}

La vida (\textit{lifetime}) de un hilo comienza cuando se invoca la función
\Rustinline{std::thread::spawn}\footnote{\url{https://doc.rust-lang.org/std/thread/fn.spawn.html}}.
Esta recibe como argumento una closure o función, que representa el
código que el nuevo hilo ejecutará concurrentemente con los demás hilos del programa. El
nuevo hilo puede empezar a ejecutarse inmediatamente después de ser \emph{spawned},
pero no hay garantía de que lo haga.

A diferencia de otros lenguajes de programación como C, C++ o Java, en Rust no existe la
no- ción de una variable de hilo inicializada previamente al inicio del hilo. En su lugar, la
función \Rustinline{std::thread::spawn} devuelve un \Rustinline{std::thread::JoinHandle}, que es, como su nombre
indica, un handle para llamar a \Rustinline{join} al final de la vida del hilo.

Durante su existencia, un hilo puede ejecutar de forma independiente su código designado y
realizar diversas operaciones concurrentemente con otros hilos. Puede acceder a recursos
compartidos y comunicarse con otros hilos a través de mecanismos de sincronización como
mutexes, variables de condición, canales u operaciones atómicas. Esto permite el
procesamiento concurrente y el paralelismo en los programas Rust.

Para garantizar una coordinación adecuada entre hilos, Rust proporciona un mecanismo
para unir hilos.
El método \Rustinline{std::thread::JoinHandle::join}\footnote{\url{https://doc.rust-lang.org/std/thread/struct.JoinHandle.html\#method.join}}
permite al hilo principal o a otro hilo
esperar hasta la finalización de otro hilo. Al llamar a join sobre un join handle, el hilo
llamante se bloquea hasta que el hilo iniciado previamente finaliza su ejecución. Una vez que un hilo
finaliza su ejecución y es unido por otro hilo, finaliza su vida útil y se liberan los recursos
correspondientes del sistema. De lo contrario, los hilos que no se unieron correctamente
pueden potencialmente perder recursos.

Si se libera el join handle, el hilo ya no puede unirse y se convierte implícitamente en
\emph{desacoplado} (\emph{detached}).
Un hilo \emph{detached} se refiere a un hilo sin un join handle válido.
Continuará su ejecución de forma independiente hasta que finalice o el programa termine.
Son útiles en escenarios en los que el hilo generador no necesita esperar a que este complete su tarea.
Por ejemplo, en tareas en segundo plano de larga duración o cuando el hilo principal termina
independientemente del progreso del hilo desprendido. Sin embargo, es importante destacar
que la ejecución de los hilos desvinculados puede continuar \emph{incluso} después de que el hilo
principal haya terminado.

\subsection{Modelo de red de Petri para un hilo}

Para incorporar hilos adicionales al modelo \acrshort{PN}, se añade una subred distinta a la red principal
para representar cada hilo. Esta subred encapsula la ruta de ejecución del nuevo hilo generado y
funciona como un contexto aislado. Establece interfaces precisas que conectan de nuevo con la
red principal. La \emph{closure} proporcionada a la función \Rustinline{spawn}, al ser una función MIR, puede
invocar otras funciones que a su vez requieren traducción. Por lo tanto, el procesamiento de la
función de un hilo sigue un enfoque similar al de la lógica de traducción habitual.

El aspecto de concurrencia de la ejecución del nuevo hilo se modela mediante la generación de
un nuevo token en la transición que representa la llamada a \Rustinline{spawn}. Este token puede
interpretarse, del mismo modo que el token en \Rustinline{PROGRAM_START}, como el contador de
instrucciones del nuevo hilo. Esencialmente, la operación spawn constituye una ``bifurcación''
en el flujo de tokens: Un token entra en la transición y dos tokens salen de ella. El primero
avanza por el camino del hilo principal para ejecutar la sentencia posterior, mientras que el
segundo se dirige al primer \acrshort{BB} de la función pasada al hilo

Cada hilo identificado en el código fuente posee unos lugares designados de inicio y fin
etiquetados como \Rustinline{THREAD_<index>_START} y \Rustinline{THREAD_<index>_END}, respectivamente. El índice
es obligatorio para prever la propiedad de unicidad de la etiqueta en todo el programa. Cabe
destacar que esto sigue el patrón de los lugares básicos del programa detallados en la Sec. \ref{sec:basic-places}.

Los hilos carecen de un lugar de pánico separado, ya que invocar \Rustinline{panic!} dentro de un hilo sólo
termina la ejecución de ese hilo específico. No nos interesa diferenciar entre los estados finales
de los hilos; el requisito principal es determinar si un hilo ha terminado o no. Aquí basta con un
único lugar de finalización para ambos casos.

El comportamiento de unión sirve como operación inversa de la generación. La transición
correspondiente a la llamada \Rustinline{join} consume dos tokens pero genera sólo un token.
Como resultado, la condición de espera se modela de forma directa:
El hilo principal puede continuar, o sea, la
transición \Rustinline{join} puede dispararse, si y sólo si el hilo a unir ha finalizado la ejecución, alcanzando
su respectivo lugar \Rustinline{THREAD_END}.


Recapitulando, el hilo se traslada a una subred separada que interactúa con la red principal
sólo en tres lugares:

\begin{itemize}
      \item La transición de \Rustinline{spawn} donde comienza el hilo.
      \item La transición de \Rustinline{join} (opcional) en la que se utiliza el join handle.
      \item Las conexiones debidas a variables de sincronización, analizadas más adelante en las
            secciones dedicadas de este capítulo.
\end{itemize}

\subsection{Un ejemplo práctico}

Observe el Listado \ref{lst:multithreading-example}
y su correspondiente modelo de red de Petri en la Fig. \ref{fig:multithreading-example}.
Se trata de uno de los programas de prueba que se encuentran en el repositorio.
Observe la ``bifurcación'' en la
transición de spawn descrita en la subsección previa.
La rama izquierda es el nuevo hilo,
mientras que la rama derecha es el hilo principal.
Está claro que las rutas se dividen en el \Rustinline{spawn}
y se fusionan en el \Rustinline{join}. Observe también que no hay un lugar de pánico separado para el hilo,
lo que indica que un fallo en un hilo no afecta a los demás hilos.

\begin{listing}[!htb]
      \begin{minted}{Rust}
      fn main() {
        let thread_join_handle = std::thread::spawn(move || {
            // some work here
        });
        // some work here
        let _res = thread_join_handle.join();
      }  
      \end{minted}
      \caption{Un programa básico con dos hilos para demostrar el soporte multihilo.}
      \label{lst:multithreading-example}
\end{listing}

\begin{figure}[!htbp]
      \centering
      \includesvg[width=0.5\linewidth]{multithreading-example.svg}
      \caption{El modelo de red de Petri para el programa del Listado \ref{lst:multithreading-example}.}
      \label{fig:multithreading-example}
\end{figure}

\subsection{Algoritmos para la traducción de hilos}

Para terminar esta sección, describiremos brevemente los algoritmos utilizados para traducir los
hilos. Inicialmente, cabe mencionar que, dado que la traducción la realiza un único hilo (la
herramienta no admite múltiples hilos traduciendo el código fuente), hay que tomar una
decisión sobre cuándo traducir los hilos generados:

\begin{itemize}
      \item Traducción inmediata: Traduce el hilo en cuanto lo encuentra. El traductor ``cambia'' al
            hilo generado.
      \item Traducción diferida: Almacena toda la información relevante sobre el nuevo hilo y lo
            traduce después del hilo principal.
\end{itemize}

La solución actual adopta este último enfoque.

Cuando se encuentra una llamada a \Rustinline{std::thread::spawn}:

\begin{enumerate}
      \item Traducir la llamada a la función utilizando el modelo visto en la Fig. \ref{fig:function-with-cleanup}.
      \item Recuperar el primer argumento pasado a la función: Un valor agregado que contiene las
            variables capturadas por la clausura y la función que ejecutará el hilo.
      \item Extraer el ID de la función que debe ejecutar el hilo.
      \item Extraer los valores capturados por la clausura.
      \item Crear un nuevo \Rustinline{Thread}\footnote{\url{https://github.com/hlisdero/cargo-check-deadlock/blob/main/src/translator/sync/thread.rs}}
            para almacenar la información necesaria para la traducción retardada.
      \item Vincular el valor de retorno de \Rustinline{std::thread::spawn},
            el nuevo join handle, al \Rustinline{Thread}.
      \item Agregar el hilo a una cola de hilos detectados en el \Rustinline{Translator}.
\end{enumerate}

Cuando se encuentra una llamada a \Rustinline{std::thread::JoinHandle::join}:

\begin{enumerate}
      \item Traducir la llamada a la función utilizando el modelo visto en la Fig. \ref{fig:simplest-function}.
            Ignorar el lugar de limpieza ya que debemos obligar a la \acrshort{PN} a ``esperar'' a que el hilo salga. Esto
            equivale a suponer que la función \Rustinline{join} nunca falla.
      \item Recuperar el primer argumento pasado a la función: El join handle. La posición de
            memoria está vinculada al hilo correspondiente gracias a la interceptación de
            asignaciones explicada en la Sec. \ref{sec:intercepting-assignments}.
      \item Establecer la transición de join del \Rustinline{Thread} referenciado por el join handle.
\end{enumerate}

Cuando el hilo principal termina de traducir, es decir, cuando la función \Rustinline{main} ya ha
sido procesada, el \Rustinline{Translator} entra en un bucle para traducir los hilos descubiertos hasta el
momento en orden.

\begin{enumerate}
      \item Crear un nuevo lugar de inicio y final para el hilo.
      \item Conectar la transición de spawn al lugar de inicio.
      \item Si se ha encontrado una transición de join, conectar el lugar final a ella.
      \item Sustituir el lugar \Rustinline{PROGRAM_PANIC} por el lugar \Rustinline{THREAD_<index>_END} para traducir
            correctamente las sentencias terminadoras como \Rustinline{Unwind} (Sec. \ref{sec:terminators}).
      \item Agregar la función del hilo a la pila de llamadas.
      \item Mover las variables de sincronización a la memoria de la función del hilo, es decir, asigne
            el agregado (tuple, \Rustinline{struct}, etc.) y sus campos a la memoria de la función del hilo.
      \item Traducir el elemento superior de la pila de llamadas.
\end{enumerate}

Como se anticipó antes, el algoritmo muestra semejanzas con el procedimiento general para las
llamadas a funciones esbozado en la Sec. \ref{sec:function-calls}.
Por último, la implementación es capaz de manejar programas
en los que los hilos engendran sus propios hilos de forma anidada.
Los hilos simplemente se añaden a la cola y, a medida que avanza el bucle,
los hilos anidados también se trasladan.