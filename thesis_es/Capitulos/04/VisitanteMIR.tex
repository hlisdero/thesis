\section{MIR visitor}

Esta sección está dedicada a un componente fundamental que sirve de columna vertebral del
traductor: el trait \acrshort{MIR} Visitor\footnote{\url{https://doc.rust-lang.org/stable/nightly-rustc/rustc_middle/mir/visit/trait.Visitor.html}}.
Este trait facilita la navegación por la \acrshort{MIR} del código
fuente de Rust. En otras palabras, actúa como el pegamento que une a la perfección los
distintos componentes del traductor.

El trait \acrshort{MIR} Visitor desempeña un papel fundamental en el proceso de traducción al
proporcionar un enfoque estructurado para recorrer y analizar el \acrshort{MIR}. Ofrece un conjunto de
métodos que pueden implementarse para realizar acciones específicas en distintos puntos
durante el recorrido. Implementando este trait, el traductor adquiere la capacidad de explorar
sistemáticamente el \acrshort{MIR} y extraer la información necesaria para generar la red de Petri
correspondiente.

Los métodos implementados dentro del traot \texttt{MIRVisitor} sirven como puntos de entrada
para manejar los diferentes elementos encontrados durante el recorrido. Estos métodos
permiten el procesamiento personalizado de construcciones \acrshort{MIR} específicas, por ejemplo,
bloques básicos, sentencias, terminadores, asignaciones, constantes, etc. Al definir el
comportamiento adecuado para cada método, el traductor puede extraer eficazmente los datos
relevantes y tomar decisiones correctas en función de los elementos \acrshort{MIR} encontrados.

No es necesario implementar todos los métodos posibles. Si no se definen, los métodos de
\texttt{MIRVisitor} simplemente llaman al método \Rustinline{super} correspondiente y continúan el recorrido. Por
ejemplo, \Rustinline{visit_statement} llama a \Rustinline{super_statement} si no existe una implementación
personalizada. En el caso del traductor, los métodos implementados son:

\begin{itemize}
  \item \Rustinline{visit_basic_block_data} para realizar un seguimiento del bloque básico que se está traduciendo en ese momento.
  \item \Rustinline{visit_assign} para realizar un seguimiento de las asignaciones de variables de
        sincronización (mutexes, mutex guards, join handles y condition variables)
  \item \Rustinline{visit_terminator} para procesar la sentencia terminadora de cada bloque básico, es decir,
        para conectar los bloques básicos.
\end{itemize}

Para empezar a visitar el MIR, debe utilizarse el método \Rustinline{visit_body}.
El Listado \ref{lst:mir-visit-body} muestra la función correspondiente en el traductor.

\begin{listing}[!htb]
  \begin{minted}{Rust}
    /// Main translation loop.
    /// Translates the function from the top of the call stack.
    /// Inside the MIR Visitor, when a call to another function happens, this method will be called again
    /// to jump to the new function. Eventually a "leaf function" will be reached, the functions will exit and the
    /// elements from the stack will be popped in order.
    fn translate_top_call_stack(&mut self) {
        let function = self.call_stack.peek();
        // Obtain the MIR representation of the function.
        let body = self.tcx.optimized_mir(function.def_id);
        // Visit the MIR body of the function using the methods of `rustc_middle::mir::visit::Visitor`.
        // <https://doc.rust-lang.org/stable/nightly-rustc/rustc_middle/mir/visit/trait.Visitor.html>
        self.visit_body(body);
        // Finished processing this function.
        self.call_stack.pop();
    }    
  \end{minted}
  \caption{El método del \Rustinline{Translator} que inicia el recorrido del \acrshort{MIR}.}
  \label{lst:mir-visit-body}
\end{listing}

En conclusión, el trait \texttt{MIRVisitor} simplifica notablemente la traducción porque no es
necesario implementar un mecanismo propio de recorrido gracias a las interfaces de compilador
proporcionadas. Esto también hace que el traductor sea más robusto y resistente a los cambios
en \emph{rustc}. Si la representación de cadenas de la \acrshort{MIR} cambia, el traductor no se ve afectado.
Mientras las interfaces internas para acceder al \acrshort{MIR} sigan siendo las mismas, el traductor
podrá recorrer el \acrshort{MIR} semánticamente y no en función de cómo se imprime al usuario.

Como última observación, existen traits similares para otras representaciones intermedias:

\begin{itemize}
  \item \acrshort{AST}: \url{https://doc.rust-lang.org/stable/nightly-rustc/rustc_ast/visit/trait.Visitor.html}
  \item \acrshort{HIR}: \url{https://doc.rust-lang.org/stable/nightly-rustc/rustc_hir/intravisit/trait.Visitor.html}
  \item \acrshort{THIR}: \url{https://doc.rust-lang.org/stable/nightly-rustc/rustc_middle/thir/visit/trait.Visitor.html}
\end{itemize}

Varios componentes del compilador implementan estos traits para navegar por las repre-
sentaciones intermedias. Por decirlo de forma aproximada, son análogos a los iteradores para
colecciones.