\section{Programas de pruebas notables}

Para concluir este capítulo, presentamos dos programas de prueba notables que ilustran las
capacidades actuales de la herramienta desarrollada en esta tesis.
Nuestra intención es inspirar a otros para que contribuyan a este proyecto o, por lo menos,
generar interés en el campo de la verificación de modelos.

En primer lugar, el Listado \ref{lst:dating-philosophers} muestra
una versión sencilla del famoso Problema de los Filósofos Comensales propuesto por Dijkstra.
Esta versión, cariñosamente apodada ``La cita de los filósofos'',
tiene sólo dos filósofos y dos tenedores en la mesa. Es necesario bloquear un mutex
para acceder a cada tenedor. Cuando los filósofos intentan tomar ambos tenedores para comer,
el programa se bloquea, lo que es fácil de comprobar por inspección. Este deadlock es
detectado con éxito por la herramienta.
Además en el repositorio\footnote{\url{https://github.com/hlisdero/cargo-check-deadlock/blob/main/examples/programs/thread/dining\_philosophers.rs}}
se incluye una versión más
compleja con 5 filósofos para la que también se detecta el deadlock.
Omite aquí debido a las limitaciones de espacio.

\begin{listing}[!htbp]
    \begin{minted}{Rust}
  use std::sync::{Arc, Mutex};
  use std::thread;
    
  fn main() {
      let fork0 = Arc::new(Mutex::new(0));
      let fork1 = Arc::new(Mutex::new(1));
    
      let philosopher0 = {
          let left_fork = fork0.clone();
          let right_fork = fork1.clone();
          thread::spawn(move || {
              let _left = left_fork.lock().unwrap();
              let _right = right_fork.lock().unwrap();
          })
      };
    
      let philosopher1 = {
          let left_fork = fork1.clone();
          let right_fork = fork0.clone();
          thread::spawn(move || {
              let _left = left_fork.lock().unwrap();
              let _right = right_fork.lock().unwrap();
          })
      };
    
      // Wait for all threads to finish
      philosopher0.join().unwrap();
      philosopher1.join().unwrap();
  }    
  \end{minted}
    \caption{Una versión reducida del problema de la cena de los filósofos que se bloquea.}
    \label{lst:dating-philosophers}
\end{listing}

En segundo lugar, observe el programa del Listado \ref{lst:producer-consumer}.
Modela el problema clásico de productor-consumidor.
Utiliza una condition variable y un búfer con capacidad para un solo elemento.
El acceso al búfer está protegido por un mutex. El productor genera 10 elementos
secuencialmente y el consumidor los procesa a medida que están disponibles. La herramienta
verifica con éxito la ausencia de deadlock en el programa.

\begin{listing}[!htbp]
    \begin{minted}{Rust}
  use std::sync::{Arc, Condvar, Mutex};
  use std::thread;
  
  fn main() {
      let buffer = Arc::new((Mutex::new(0), Condvar::new(), Condvar::new()));
  
      let producer_buffer = buffer.clone();
      let consumer_buffer = buffer.clone();
  
      let _producer = thread::spawn(move || {
          for i in 1..10 {
              let (lock, cvar_producer, cvar_consumer) = &*producer_buffer;
              let mut buffer = lock.lock().unwrap();
  
              while *buffer != 0 {
                  buffer = cvar_producer.wait(buffer).unwrap();
              }
  
              *buffer = i;
              println!("Produced: {}", i);
  
              cvar_consumer.notify_one();
          }
      });
  
      let _consumer = thread::spawn(move || loop {
          let (lock, cvar_producer, cvar_consumer) = &*consumer_buffer;
          let mut buffer = lock.lock().unwrap();
  
          while *buffer == 0 {
              buffer = cvar_consumer.wait(buffer).unwrap();
          }
  
          let item = *buffer;
          *buffer = 0;
          println!("Consumed: {}", item);
  
          cvar_producer.notify_one();
      });
  }
  \end{minted}
    \caption{Una solución al problema del productor-consumidor.}
    \label{lst:producer-consumer}
\end{listing}