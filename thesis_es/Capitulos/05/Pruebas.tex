La inclusión de un capítulo dedicado a las pruebas en la tesis subraya la importancia de este
aspecto indispensable del proceso de desarrollo. Las pruebas desempeñan un papel
fundamental para garantizar la fiabilidad y correctitud de la implementación del software. Se ha
desarrollado un completo conjunto de pruebas para cubrir la extensa funcionalidad y
comportamiento del traductor y la biblioteca de redes de Petri.

Las pruebas abarcan múltiples niveles que se dilucidarán en las secciones siguientes. En el
nivel más bajo, se realizan pruebas unitarias para verificar la corrección de las estructuras de
datos empleadas en el traductor y la biblioteca de redes de Petri. Estas pruebas se dirigen a componentes
individuales, examinando a fondo su funcionalidad de forma aislada.

Además de las pruebas unitarias, se ha desarrollado de forma incremental un conjunto de pruebas
de integración para evaluar la conformidad del traductor al comportamiento esperado. Estas
pruebas consisten en programas de prueba que simulan escenarios sencillos en los que se
compara el archivo de salida con los resultados esperados. Esta metodología de
pruebas ayuda a descubrir cualquier regresión en el compilador y confirma que el traductor
funciona de forma fiable en los casos de uso soportados.

Adicionalmente incorporamos una descripción de cómo generar el \acrshort{MIR} y visualizar el resultado de la
traducción para asistir en el proceso de depuración. Las herramientas permiten exponer los
detalles internos de forma accesible y comprensible.

Más adelante en este capítulo se explica el uso del verificador de modelos \acrshort{LoLA} y su
integración en el traductor. El verificador de modelos proporciona más características que el
conjunto mínimo que se integró en el traductor para responder al mero problema de la detección de
deadlocks. Por lo tanto, es beneficioso explorar qué características proporciona el verificador de
modelos para depurar la traducción a una \acrshort{PN}.

Para finalizar se demuestran las capacidades de la herramienta mediante dos programas de prueba
que modelan problemas clásicos de la programación concurrente.