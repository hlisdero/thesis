\section{Integration tests}
\label{sec:integration-tests}

We will now examine the integration tests,
which serve as the backbone for the translator's testing framework.
Two types of tests exist currently:

\begin{itemize}
  \item Translation tests.
  \item Deadlock detection tests.
\end{itemize}

The testing process has proven invaluable throughout the development of the translator,
enabling early detection of bugs and regressions in \emph{rustc}.
By relying on the previous tests and building features incrementally,
the implementation progressed with confidence and a firm step forward.
The testing capabilities offered by Rust,
including its support for unit tests and integration tests,
have been instrumental in ensuring the quality and reliability of the translator.

\subsection{Translation tests}

In translation tests, a given program is processed
\emph{without} performing the deadlock analysis.
As a result, three text files containing the model
in DOT, \acrshort{PNML}, and \acrshort{LoLA} formats are generated.
These files are then compared to the expected output,
which is stored in the repository and serves as documentation as well.

The expected output was verified manually
using the tools presented in Sec. \ref{sec:visualizing-results}.
It was committed to the repository
when the translator was first able to pass the test.
If a regression in \emph{rustc} occurs,
then the expected output files are updated accordingly.
This has happened some times in the past.
See for instance
this commit\footnote{\url{https://github.com/hlisdero/cargo-check-deadlock/commit/881a3873a3b060e70bc727f670f9426d14327fa2}}
or this one\footnote{\url{https://github.com/hlisdero/cargo-check-deadlock/commit/b032fa3cc13e631950a802dcd3f755c548afde86}}.

\subsection{Deadlock detection tests}

Deadlock detection tests are closer to an end-to-end test of the translator.
They generate the file in \acrshort{LoLA} format for the test program
and instruct the translator to perform the deadlock analysis.
The output is then contrasted to the known behavior of the test program, i.e.,
it deadlocks or it does not deadlock.
If \acrshort{LoLA} produces an incorrect result, then the test fails.
In such a case, the \acrshort{PN} model should be analyzed
to find the source of the error.
See Sec. \ref{sec:debugging} for details on how to approach this.

Observe Listing \ref{lst:double-lock-deadlock-lola}, which contains
the contents of the \Rustinline{.lola} file
for the program depicted in Listing \ref{lst:double-lock-deadlock}.
This is the file format that the model checker requires.
It is relatively simpler than \acrshort{PNML}, which is XML-based.

Due to the considerable length of the output,
it is the sole instance of the \acrshort{LoLA} format in this thesis.
It is included here for completeness.
The repository contains several other examples,
all of which are used in the integration tests.

\begin{longlisting}
  \begin{minted}{Rust}
  PLACE
      MUTEX_0,
      PROGRAM_END,
      PROGRAM_PANIC,
      PROGRAM_START,
      main_BB1,
      main_BB2,
      main_BB3,
      main_BB4,
      main_BB5,
      main_BB6,
      main_BB7;
  
  MARKING
      MUTEX_0 : 1,
      PROGRAM_END : 0,
      PROGRAM_PANIC : 0,
      PROGRAM_START : 1,
      main_BB1 : 0,
      main_BB2 : 0,
      main_BB3 : 0,
      main_BB4 : 0,
      main_BB5 : 0,
      main_BB6 : 0,
      main_BB7 : 0;
  
  TRANSITION main_DROP_3
    CONSUME
      main_BB3 : 1;
    PRODUCE
      MUTEX_0 : 1,
      main_BB4 : 1;
  TRANSITION main_DROP_4
    CONSUME
      main_BB4 : 1;
    PRODUCE
      MUTEX_0 : 1,
      main_BB5 : 1;
  TRANSITION main_DROP_6
    CONSUME
      main_BB6 : 1;
    PRODUCE
      MUTEX_0 : 1,
      main_BB7 : 1;
  TRANSITION main_DROP_UNWIND_3
    CONSUME
      main_BB3 : 1;
    PRODUCE
      MUTEX_0 : 1,
      main_BB6 : 1;
  TRANSITION main_RETURN
    CONSUME
      main_BB5 : 1;
    PRODUCE
      PROGRAM_END : 1;
  TRANSITION main_UNWIND_7
    CONSUME
      main_BB7 : 1;
    PRODUCE
      PROGRAM_PANIC : 1;
  TRANSITION std_sync_Mutex_T_lock_0_CALL
    CONSUME
      MUTEX_0 : 1,
      main_BB1 : 1;
    PRODUCE
      main_BB2 : 1;
  TRANSITION std_sync_Mutex_T_lock_1_CALL
    CONSUME
      MUTEX_0 : 1,
      main_BB2 : 1;
    PRODUCE
      main_BB3 : 1;
  TRANSITION std_sync_Mutex_T_new_0_CALL
    CONSUME
      PROGRAM_START : 1;
    PRODUCE
      main_BB1 : 1;    
  \end{minted}
  \caption{The LoLA output for the program in Listing \ref{lst:double-lock-deadlock}.}
  \label{lst:double-lock-deadlock-lola}
\end{longlisting}

\subsection{Test structure}

The test programs are in the folder \Rustinline{examples/programs}.
For each test program, there is a folder in \Rustinline{examples/results}
that contains the three files
\Rustinline{net.dot}, \Rustinline{net.pnml}, and \Rustinline{net.lola}.

The tests are grouped into categories:

\begin{itemize}
  \item Basic: For basic programs like ``Hello, World!'' and a simple arithmetic calculator.
  \item Condvar: For programs concerning condition variables.
  \item Function call: For programs that test different types
        of function calls seen in Sec. \ref{sec:function-calls}.
  \item Mutex: For programs that use mutexes.
  \item Statement: For programs that test specific constructs such as a \Rustinline{match},
        an infinite loop, an \Rustinline{Option}, a call to \Rustinline{panic!},
        or \Rustinline{std::process::abort}.
  \item Thread: For programs involving multiple threads.
\end{itemize}

The structure of the folders in \Rustinline{examples/} mimics the file structure
of the integration tests in \Rustinline{tests/}.
As usual, the whole test suite can be run with the \Rustinline{cargo test} command.

\subsection{Test implementation}

The integration tests rely on the crates \texttt{assert\_cmd}, \texttt{assert\_fs},
and \texttt{predicates}.
The idea to verify the output of the program by invoking the binary directly was
taken from a specialized book for building \acrshort{CLI} applications in Rust \cite[Chap. 1.6]{rust-cli-book}.
It was a useful resource for experimenting with \texttt{clap} to parse arguments as well.

Additionally, the integration tests use a shared submodule \cite[Chap. 11.3]{rust-book}
that contains two convenient macros that save us from writing nearly all of the boilerplate code.
These macros were defined using \cite{rust-macros} as a primary reference
and with inspiration provided by \cite{oaten2023}.

Listing \ref{lst:macro-test-generation} shows the macro responsible for generating the translation tests,
while Listing \ref{lst:macro-test-generation-example} provides an example of how it is applied
in the repository. For the sake of completeness, Listing \ref{lst:assert-output-files}
depicts the function used for the translation tests.

\begin{listing}[!htbp]
  \begin{minted}{Rust}
  macro_rules! generate_tests_for_example_program {
      ($program_path:literal, $result_folder_path:literal) => {
          #[test]
          fn generates_correct_output_files() {
              super::utils::assert_output_files($program_path, $result_folder_path);
          }
      };
  }        
  \end{minted}
  \caption{The macro that generates the translation tests.}
  \label{lst:macro-test-generation}
\end{listing}


\begin{listing}[!htbp]
  \begin{minted}{Rust}
  mod utils;
  
  mod calculator {
      super::utils::generate_tests_for_example_program!(
          "./examples/programs/basic/calculator.rs",
          "./examples/results/basic/calculator/"
      );
  }
  
  mod greet {
      super::utils::generate_tests_for_example_program!(
          "./examples/programs/basic/greet.rs",
          "./examples/results/basic/greet/"
      );
  }
  
  mod hello_world {
      super::utils::generate_tests_for_example_program!(
          "./examples/programs/basic/hello_world.rs",
          "./examples/results/basic/hello_world/"
      );
  }    
  \end{minted}
  \caption{The contents of the file \Rustinline{basic.rs} listing all translation tests in the basic category.}
  \label{lst:macro-test-generation-example}
\end{listing}

\begin{listing}[!htbp]
  \begin{minted}{Rust}
  pub fn assert_output_files(source_code_file: &str, output_folder: &str) {
      let mut cmd = Command::cargo_bin("cargo-check-deadlock").expect("Command not found");
  
      // Current workdir is always the project root folder
      cmd.arg("check-deadlock")
          .arg(source_code_file)
          .arg(format!("--output-folder={output_folder}"))
          .arg("--dot")
          .arg("--pnml")
          .arg("--filename=test")
          .arg("--skip-analysis");
  
      cmd.assert().success();
  
      for extension in ["lola", "dot", "pnml"] {
          let output_path = PathBuf::from(format!("{output_folder}test.{extension}"));
          let expected_output_path = PathBuf::from(format!("{output_folder}net.{extension}"));
  
          let file_contents =
              std::fs::read_to_string(&output_path).expect("Could not read output file to string");
  
          let expected_file_contents = std::fs::read_to_string(&expected_output_path)
              .expect("Could not read file with expected contents to string");
  
          if file_contents != expected_file_contents {
              panic!(
                  "The contents of {} do not match the contents of {}",
                  output_path.to_string_lossy(),
                  expected_output_path.to_string_lossy()
              );
          }
  
          std::fs::remove_file(output_path).expect("Could not delete output file");
      }
  }
  \end{minted}
  \caption{The function that verifies the contents of the output files.}
  \label{lst:assert-output-files}
\end{listing}

