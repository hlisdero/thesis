\section{Integration tests}
\label{sec:integration-tests}

A continuación examinaremos las pruebas de integración que constituyen la columna vertebral
del marco de pruebas del traductor. Actualmente existen dos tipos de pruebas:

\begin{itemize}
  \item Pruebas de traducción.
  \item Pruebas de detección de bloqueo.
\end{itemize}

El proceso de pruebas ha resultado inestimable a lo largo del desarrollo del traductor,
permitiendo la detección temprana de errores y regresiones en \emph{rustc}. Apoyándose en las
pruebas anteriores y construyendo características de forma incremental, la implementación
avanzó con confianza y paso firme hacia adelante. Las capacidades de prueba que ofrece
Rust, incluido su soporte para pruebas unitarias y de integración, han sido valiosas para
garantizar la calidad y fiabilidad del traductor.

\subsection{Pruebas de traducción}

En las pruebas de traducción, se procesa un programa dado \emph{sin} realizar el análisis de bloqueo.
Como resultado, se generan tres archivos de texto que contienen el modelo
en formatos DOT, \acrshort{PNML} y \acrshort{LoLA}.
Posteriormente estos archivos se comparan con la salida esperada, que se
almacena en el repositorio y sirve también como documentación.

El resultado esperado se verificó manualmente
con la ayuda de las herramientas presentadas en la Sec. \ref{sec:visualizing-results}.
Se committeó en el repositorio cuando el traductor pudo superar la prueba por primera
vez. Si se produce una regresión en \emph{rustc}, los archivos de salida esperada se actualizan en
consecuencia.
Esto ha ocurrido algunas veces en el pasado.
Véase, por ejemplo,
este commit\footnote{\url{https://github.com/hlisdero/cargo-check-deadlock/commit/881a3873a3b060e70bc727f670f9426d14327fa2}}
o este\footnote{\url{https://github.com/hlisdero/cargo-check-deadlock/commit/b032fa3cc13e631950a802dcd3f755c548afde86}}.

\subsection{Pruebas de detección de bloqueo}

Las pruebas de detección de deadlock se acercan más a una prueba \emph{end-to-end}
del traductor. Generan el archivo en formato \acrshort{LoLA} para el programa de prueba y ordenan al
traductor que realice el análisis de deadlock. El resultado se contrasta entonces con el
comportamiento conocido del programa de prueba, es decir, si se bloquea o no se bloquea. Si
\acrshort{LoLA} produce un resultado incorrecto, entonces la prueba falla. En tal caso, debe analizarse el
modelo \acrshort{PN} para encontrar la fuente del error.
Consulte la sección \ref{sec:debugging} para obtener más
detalles sobre cómo abordar esta cuestión.

Observe el Listado \ref{lst:double-lock-deadlock-lola}
que contiene el contenido del archivo \Rustinline{.lola} para el programa
representado en el Listado \ref{lst:double-lock-deadlock}.
Este es el formato de archivo que requiere el verificador de
modelos. Es relativamente más sencillo que \acrshort{PNML}, el cual está basado en XML.

Debido a su considerable longitud, es el único ejemplo del formato \acrshort{LoLA} en esta tesis.
Se lo incluye aquí por razones de completitud.
El repositorio contiene varios ejemplos más,
todos de los cuales se utilizan en las pruebas de integración.

\begin{longlisting}
  \begin{minted}{Rust}
  PLACE
      MUTEX_0,
      PROGRAM_END,
      PROGRAM_PANIC,
      PROGRAM_START,
      main_BB1,
      main_BB2,
      main_BB3,
      main_BB4,
      main_BB5,
      main_BB6,
      main_BB7;
  
  MARKING
      MUTEX_0 : 1,
      PROGRAM_END : 0,
      PROGRAM_PANIC : 0,
      PROGRAM_START : 1,
      main_BB1 : 0,
      main_BB2 : 0,
      main_BB3 : 0,
      main_BB4 : 0,
      main_BB5 : 0,
      main_BB6 : 0,
      main_BB7 : 0;
  
  TRANSITION main_DROP_3
    CONSUME
      main_BB3 : 1;
    PRODUCE
      MUTEX_0 : 1,
      main_BB4 : 1;
  TRANSITION main_DROP_4
    CONSUME
      main_BB4 : 1;
    PRODUCE
      MUTEX_0 : 1,
      main_BB5 : 1;
  TRANSITION main_DROP_6
    CONSUME
      main_BB6 : 1;
    PRODUCE
      MUTEX_0 : 1,
      main_BB7 : 1;
  TRANSITION main_DROP_UNWIND_3
    CONSUME
      main_BB3 : 1;
    PRODUCE
      MUTEX_0 : 1,
      main_BB6 : 1;
  TRANSITION main_RETURN
    CONSUME
      main_BB5 : 1;
    PRODUCE
      PROGRAM_END : 1;
  TRANSITION main_UNWIND_7
    CONSUME
      main_BB7 : 1;
    PRODUCE
      PROGRAM_PANIC : 1;
  TRANSITION std_sync_Mutex_T_lock_0_CALL
    CONSUME
      MUTEX_0 : 1,
      main_BB1 : 1;
    PRODUCE
      main_BB2 : 1;
  TRANSITION std_sync_Mutex_T_lock_1_CALL
    CONSUME
      MUTEX_0 : 1,
      main_BB2 : 1;
    PRODUCE
      main_BB3 : 1;
  TRANSITION std_sync_Mutex_T_new_0_CALL
    CONSUME
      PROGRAM_START : 1;
    PRODUCE
      main_BB1 : 1;    
  \end{minted}
  \caption{La salida LoLA para el programa del Listado \ref{lst:double-lock-deadlock}.}
  \label{lst:double-lock-deadlock-lola}
\end{longlisting}

\subsection{Estructura de ficheros de las pruebas}

Los programas de prueba se encuentran en la carpeta \Rustinline{examples/programs}.
Para cada programa de prueba, hay una carpeta en \Rustinline{examples/results}
que contiene los tres archivos \Rustinline{net.dot}, \Rustinline{net.pnml} y \Rustinline{net.lola}.

Los tests se agrupan en categorías:

\begin{itemize}
  \item Basic: Para programas básicos como "¡Hola, mundo!" y una simple calculadora aritmética.
  \item Condvar: Para programas relativos a condition variables.
  \item Function call: Para los programas que prueban los diferentes tipos de llamadas de función
        vistos en la Sec. \ref{sec:function-calls}.
  \item Mutex: Para programas hacen uso de mutexes.
  \item Statement: Para programas que comprueban construcciones específicas como una \Rustinline{match},
        un bucle infinito, una \Rustinline{Option}, una llamada a \Rustinline{panic!},
        o \Rustinline{std::process::abort}.
  \item Thread: Para programas en los que intervienen varios hilos.
\end{itemize}

La estructura de las carpetas en \Rustinline{examples/} imita la estructura de archivos
de las pruebas de integración en \Rustinline{tests/}.
Como de costumbre, todo el conjunto de pruebas puede ejecutarse con el comando \Rustinline{cargo test}.

\subsection{Implementación de las pruebas}

Las pruebas de integración se basan en los crates \texttt{assert\_cmd},
\texttt{assert\_fs} y \texttt{predicates}.
La idea de verificar la salida del programa invocando directamente al binario se tomó de un
libro especializado en la construcción de aplicaciones \acrshort{CLI} en Rust \cite[Chap. 1.6]{rust-cli-book}.
Fue un recurso útil también para experimentar con \texttt{clap} para parsear los argumentos.

Además, las pruebas de integración utilizan un submódulo compartido \cite[Chap. 11.3]{rust-book}
que contiene dos cómodas macros que nos ahorran escribir casi todo el código
boilerplate. Estas macros se definieron utilizando \cite{rust-macros} como referencia
principal y con la inspiración proporcionada por \cite{oaten2023}.

El Listado \ref{lst:macro-test-generation} muestra la macro
responsable de generar las pruebas de traducción,
mientras que el Listado \ref{lst:macro-test-generation-example}
ofrece un ejemplo de cómo se aplica en el repositorio.
En aras de la completitud, el Listado
\ref{lst:assert-output-files} representa la función utilizada para las pruebas de traducción.


\begin{listing}[!htbp]
  \begin{minted}{Rust}
  macro_rules! generate_tests_for_example_program {
      ($program_path:literal, $result_folder_path:literal) => {
          #[test]
          fn generates_correct_output_files() {
              super::utils::assert_output_files($program_path, $result_folder_path);
          }
      };
  }        
  \end{minted}
  \caption{La macro que genera las pruebas de traducción.}
  \label{lst:macro-test-generation}
\end{listing}


\begin{listing}[!htbp]
  \begin{minted}{Rust}
  mod utils;
  
  mod calculator {
      super::utils::generate_tests_for_example_program!(
          "./examples/programs/basic/calculator.rs",
          "./examples/results/basic/calculator/"
      );
  }
  
  mod greet {
      super::utils::generate_tests_for_example_program!(
          "./examples/programs/basic/greet.rs",
          "./examples/results/basic/greet/"
      );
  }
  
  mod hello_world {
      super::utils::generate_tests_for_example_program!(
          "./examples/programs/basic/hello_world.rs",
          "./examples/results/basic/hello_world/"
      );
  }    
  \end{minted}
  \caption{El contenido del archivo \Rustinline{basic.rs} que enumera
    todas las pruebas de traducción de la categoría básica.}
  \label{lst:macro-test-generation-example}
\end{listing}

\begin{listing}[!htbp]
  \begin{minted}{Rust}
  pub fn assert_output_files(source_code_file: &str, output_folder: &str) {
      let mut cmd = Command::cargo_bin("cargo-check-deadlock").expect("Command not found");
  
      // Current workdir is always the project root folder
      cmd.arg("check-deadlock")
          .arg(source_code_file)
          .arg(format!("--output-folder={output_folder}"))
          .arg("--dot")
          .arg("--pnml")
          .arg("--filename=test")
          .arg("--skip-analysis");
  
      cmd.assert().success();
  
      for extension in ["lola", "dot", "pnml"] {
          let output_path = PathBuf::from(format!("{output_folder}test.{extension}"));
          let expected_output_path = PathBuf::from(format!("{output_folder}net.{extension}"));
  
          let file_contents =
              std::fs::read_to_string(&output_path).expect("Could not read output file to string");
  
          let expected_file_contents = std::fs::read_to_string(&expected_output_path)
              .expect("Could not read file with expected contents to string");
  
          if file_contents != expected_file_contents {
              panic!(
                  "The contents of {} do not match the contents of {}",
                  output_path.to_string_lossy(),
                  expected_output_path.to_string_lossy()
              );
          }
  
          std::fs::remove_file(output_path).expect("Could not delete output file");
      }
  }
  \end{minted}
  \caption{La función que verifica el contenido de los archivos de salida.}
  \label{lst:assert-output-files}
\end{listing}

