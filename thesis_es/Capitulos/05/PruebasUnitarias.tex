\section{Pruebas unitarias}

Las pruebas unitarias constituyen la base del conjunto de pruebas. La biblioteca de \acrshort{PN} y las
estructuras de datos utilizadas en el traductor dependen en gran medida de ellas. Al probar
meticulosamente las estructuras de datos subyacentes, se pueden identificar y resolver posibles
problemas en una fase temprana del ciclo de desarrollo, antes de seguir trabajando en los
componentes de nivel superior del traductor.


\subsection{Biblioteca de redes de Petri}

La biblioteca de \acrshort{PN} \Rustinline{netcrab} utiliza pruebas unitarias para verificar que
la adición de lugares, transiciones y arcos a una red se comporta como se espera.
El traductor realiza estas operaciones con frecuencia, por lo que es importante verificarlas.
Asimismo se comprueban los iteradores sobre los que se construyen los formatos de exportación.

Por otra parte, cada uno de los tres formatos de exportación (DOT, \acrshort{PNML} and \acrshort{LoLA})
está acompañado por pruebas unitarias para comprobar que la salida es generada correctamente para los siguientes
casos sencillos:

\begin{itemize}
  \item Red vacía.
  \item Un red con 5 lugares y 0 transiciones.
  \item Un red con 5 lugares y 0 transiciones.
  \item Un red con 5 lugares marcados con diferentes números de fichas.
  \item Un red con topología en cadena.
  \item Un red con 1 lugar y 1 transición conectados en bucle.
\end{itemize}

Estas pruebas ayudaron a solucionar errores relacionados con el formato LoLA. Para ver
ejemplos concretos, consulte este
commit\footnote{\url{https://github.com/hlisdero/netcrab/commit/5745e0da5d27bd709ef479f45a6d2e75974d3745}}
o este otro\footnote{\url{https://github.com/hlisdero/netcrab/commit/dbce3f8999ece32e6731527c303a7b59858991f9}}.

\subsection{Pila (\textit{Stack})}

La sencilla estructura de datos
\Rustinline{Stack}\footnote{\url{https://github.com/hlisdero/cargo-check-deadlock/blob/main/src/data_structures/stack.rs}}
se emplea para implementar la pila de llamadas en el \Rustinline{Translator}, como se
vio en la Sec. \ref{sec:function-calls}.
Las pruebas unitarias demuestran los métodos admitidos y testean algunos casos de uso sencillos.

\subsection{Contador de mapa hash}

De forma análoga a la pila,
el \Rustinline{HashMapCounter}\footnote{\url{https://github.com/hlisdero/cargo-check-deadlock/blob/main/src/data_structures/hash_map_counter.rs}}
contiene algunas pruebas unitarias para
verificar que los métodos funcionan según lo previsto.
Esta estructura de datos constituye la
base del contador de funciones del \Rustinline{Translator}
que lleva la cuenta de cuántas veces se ha llamado a cada función para generar un índice
incremental inequívoco para las etiquetas de las transiciones.