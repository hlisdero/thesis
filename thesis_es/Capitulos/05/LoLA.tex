\section{Integrando LoLA a la solución}

Como se indica en la Sec. \ref{sec:lola-format},
\acrshort{LoLA}  es el verificador de modelos elegido para esta tesis.
Actúa como un backend que se encarga de verificar la ausencia de deadlocks.
Integrarlo no fue, por desgracia, trivial, como se detallará en los siguientes párrafos.

\subsection{Compilación}

En primer lugar, la compilación a partir del código fuente no funcionaba en el hardware del que
se disponía. Fue necesario realizar cambios en el código, ya que las versiones más recientes
del compilador de C++ tienden a ser más estrictas y rechazan o generan advertencias para el
código que antes se aceptaba. Además, una de las dependencias, kimwitu++\footnote{\url{https://www.nongnu.org/kimwitu-pp/}},
debe compilarse también a partir del código fuente ya que no está empaquetado para las distribuciones de Linux.

Para conservar en el futuro una copia funcional del verificador de modelos, indispensable
para realizar el análisis de deadlocks, se ha creado
una réplica\footnote{\url{https://github.com/hlisdero/lola}} en GitHub
en la que se ofrecen instrucciones detalladas a los usuarios.
Con ello se pretende que la instalación desde el código fuente sea lo más sencilla posible.

\subsection{Invocación del verificador de modelos}

La segunda dificultad es que \acrshort{LoLA} se compila como un ejecutable, no como una biblioteca,
por lo que nuestra herramienta no podía enlazar con él. En su lugar, nos vemos obligados a
ejecutar el binario desde \Rustinline{cargo-check-deadlock} para ejecutar el comprobador de modelos
pasando los argumentos correctos\footnote{\url{https://github.com/hlisdero/cargo-check-deadlock/blob/main/src/model_checker/lola.rs}}.
El ejecutable de \acrshort{LoLA} forma parte del repositorio porque es
necesario para ejecutar las pruebas de integración en la pipeline CI/CD implementada en GitHub Actions.
Un usuario también puede copiar este ejecutable para instalar \acrshort{LoLA} en lugar de compilarlo desde cero.

A fin de cuentas el usuario es responsable de instalar el comprobador de modelos por separado para
permitir que nuestra herramienta lo invoque.
Un script \Rustinline{copy_lola_executable_to_cargo_home.sh}
incluido en el repositorio facilita la tarea de copiar el archivo
a una carpeta que ya esté en el \Rustinline{\$PATH}.
También consideramos otras posibilidades, pero ninguna resultó viable:

In the end, the user is responsible for installing the model checker separately
to allow our tool to invoke it.
A script \Rustinline{copy_lola_executable_to_cargo_home.sh} included in the repository
facilitates the task of copying the file to a folder that is already in the \Rustinline{\$PATH}.
We also considered other possibilities but none were feasible:

\begin{enumerate}
  \item Usar build scripts (\Rustinline{build.rs}) como se describe en el Cargo Book \cite[Chap. 3.8]{cargo-book}.
  \item Modificar \acrshort{LoLA} para convertirlo en una biblioteca.
  \item Mover un ejecutable precompilado a la carpeta de instalación cuando se ejecute \Rustinline{cargo install}.
  \item Definir \acrshort{LoLA} como un binario en el \Rustinline{Cargo.toml} \cite[Chap 3.2.1]{cargo-book}
        y que, con un poco de suerte, sea movido al directorio cargo bin.
  \item Definir \acrshort{LoLA} como un \emph{example} en el \Rustinline{Cargo.toml} \cite[Chap 3.2.1]{cargo-book}
        y que, con un poco de suerte, sea movido al directorio cargo bin.
  \item Utilizar una herramienta de build de uso general como \Rustinline{make}.
\end{enumerate}

En el proceso de resolución de este segundo problema, aprendimos que cargo es adecuado
principalmente para tratar con dependencias expresadas como crates de Rust, que deben
compilarse cuando se instalan, no con archivos/artefactos arbitrarios.
En resumen, no está pensada para ser una herramienta de compilación de uso general como \Rustinline{make}.

\subsection{Expresar la propiedad a comprobar}

El tercer reto es encontrar una fórmula de \acrfull{CTL*} para ordenar
a \acrshort{LoLA} que busque los deadlocks.
Afortunadamente, podemos reutilizar la fórmula encontrada
en \cite{meyer2020}:

\begin{center}
  EF (DEADLOCK AND (PROGRAM\_END = 0 AND PROGRAM\_PANIC = 0))
\end{center}

La fórmula representa la propiedad a comprobar.
Cabe destacar que no todos los deadlocks son interesantes para nuestro análisis.
Nuestro objetivo es identificar los casos de deadlocks en
los que la ejecución del programa se bloquea \emph{inesperadamente}.
Este escenario se alinea con una red de Petri
muerta como se ve en la Definición \ref{definition:liveness} en la que no está habilitada ninguna transición
y la red alcanza un estado final.
Sin embargo, debemos actuar con cautela, ya que hay casos en los que se \emph{espera}
que la \acrshort{PN} esté muerta, como cuando el programa termina o entra en pánico.
Estos son estados normales en los que la ejecución se detiene.
Por lo tanto, si llegamos al lugar \Rustinline{PROGRAM_END}
o \Rustinline{PROGRAM_PANIC}, la ejecución ha sido satisfactoria, no se trata de un deadlock en el sentido
de la Sec. \ref{sec:coffman-conditions}.
En conclusión, excluimos los lugares \Rustinline{PROGRAM_END} y \Rustinline{PROGRAM_PANIC}
exigiendo que \emph{no estén marcados} para que se cumpla la condición de deadlock.
Esto se expresa mediante el ``= 0'' en la fórmula \acrshort{CTL*}.

Por último, debemos considerar el aspecto temporal.
Para especificar que nuestra propiedad de
estado se cumple eventualmente y encontrar un camino relevante, podemos utilizar los
operadores ``EF'' en combinación. La ``F'' significa ``eventualmente'' y la ``E'' es el cuantificador
de camino existencial \cite{meyer2020}. Así que la fórmula se lee como:

\emph{``Existe eventualmente un camino tal que DEADLOCK (ninguna transición puede dispararse) y
el lugar PROGRAM\_PANIC tiene cero fichas y el lugar PROGRAM\_END tiene cero fichas''}

Se pueden construir otras fórmulas para comprobar otras propiedades. Para este trabajo,
tomamos esta fórmula como dada y dejamos al usuario la posibilidad de comprobar otras
propiedades si así lo desea. Para una breve introducción a \acrshort{CTL*}, véase \cite{meyer2020}.