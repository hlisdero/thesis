\section{Integrating LoLA to the solution}

As stated in Sec. \ref{sec:lola-format},
\acrshort{LoLA} is the chosen model checker for this thesis.
It acts as a backend that is responsible for verifying the absence of deadlocks.
Integrating it was unfortunately not trivial.

\subsection{Compilation}

First, the compilation from the source code did not work on the hardware at our disposal.
Changes to the code were necessary as newer versions of the C++ compiler tend to be more strict
and reject or generate warnings for code that was previously accepted.
Besides, one of the dependencies, kimwitu++\footnote{\url{https://www.nongnu.org/kimwitu-pp/}},
must be compiled from the source code too
since it is not packaged for Linux distributions.

To preserve a working copy of the model checker for the future,
indispensable for performing the deadlock analysis,
a mirror\footnote{\url{https://github.com/hlisdero/lola}}
was created on GitHub where detailed instructions are provided for users.
This aims to make the installation from source as straightforward as possible.

\subsection{Invoking the model checker}

The second difficulty is that \acrshort{LoLA} is compiled to an executable,
not as a library, so our tool could not link to it.
Instead, we are compelled to execute the binary
from our \Rustinline{cargo-check-deadlock} binary
to run the model checker passing the correct
arguments\footnote{\url{https://github.com/hlisdero/cargo-check-deadlock/blob/main/src/model_checker/lola.rs}}.
The \acrshort{LoLA} executable is part of the repository because it is needed
to run the integration tests in the CI/CD pipeline using GitHub Actions.
A user may also copy this executable to install \acrshort{LoLA}
in lieu of compiling it from scratch.

In the end, the user is responsible for installing the model checker separately
to allow our tool to invoke it.
A script \Rustinline{copy_lola_executable_to_cargo_home.sh} included in the repository
facilitates the task of copying the file to a folder that is already in the \Rustinline{\$PATH}.
We also considered other possibilities but none were feasible:

\begin{enumerate}
  \item Using build scripts (\Rustinline{build.rs}) as described in the Cargo Book \cite[Chap. 3.8]{cargo-book}.
  \item Modifying \acrshort{LoLA} to turn it into a library.
  \item Move a pre-compiled executable to the installation folder when running \Rustinline{cargo install}.
  \item Define \acrshort{LoLA} as a binary in the \Rustinline{Cargo.toml} \cite[Chap 3.2.1]{cargo-book}
        and, hopefully, it gets moved to the cargo bin directory.
  \item Define \acrshort{LoLA} as an example in the \Rustinline{Cargo.toml} \cite[Chap 3.2.1]{cargo-book}
        and, hopefully, it gets moved to the cargo bin directory.
  \item Use a general-purpose build tool like \Rustinline{make}.
\end{enumerate}

In the process of solving this second problem,
we learned that cargo is mainly suited to dealing with dependencies
expressed as Rust crates, which should be compiled when installed, not with arbitrary assets.
In short, it is \emph{not} meant to be a general-purpose build tool like \Rustinline{make}.

\subsection{Expressing the property to check}

The third challenge is finding a \acrfull{CTL*} formula to instruct \acrshort{LoLA}
to search for deadlocks.
Luckily, we can reuse the formula found in \cite{meyer2020}:

\begin{center}
  EF (DEADLOCK AND (PROGRAM\_END = 0 AND PROGRAM\_PANIC = 0))
\end{center}

The formula represents the property to check for.
It should be emphasized that not all deadlocks are interesting for our analysis.
Our objective is to identify instances of deadlocks
where the program execution is \emph{unexpectedly} blocked.
This scenario aligns with a dead \acrshort{PN} as seen in Definition \ref{definition:liveness},
where no transition is enabled, and the \acrshort{PN} reaches a final state.
However, we must exercise caution as there are cases where the \acrshort{PN} is \emph{expectedly} dead,
such as when the program terminates or panics.
These are states where execution normally halts.
Thus, if we reach either the \Rustinline{PROGRAM_END} or the \Rustinline{PROGRAM_PANIC} place,
the execution was successful, not a deadlock in the sense of Sec. \ref{sec:coffman-conditions}.
In conclusion, we exclude the \Rustinline{PROGRAM_END} and the \Rustinline{PROGRAM_PANIC} places
by requiring them to be \emph{unmarked} for the deadlock condition to hold.
This is expressed by the ``= 0'' in the \acrshort{CTL*} formula.

Lastly, we need to consider the temporal aspect.
To specify that our state property eventually holds and to find a relevant path,
we can utilize the ``EF'' operators in combination.
The ``F'' stands for ``eventually''
and the ``E'' is the existential path quantifier \cite{meyer2020}.
So the formula reads as:

\emph{``There exists eventually a path such that DEADLOCK (no transition may fire)
  and the place PROGRAM\_PANIC has zero tokens and the place PROGRAM\_END has zero tokens''}

Other formulas may be constructed to check other properties.
For this work, we take this formula as a given and
we leave the user the possibility of checking other properties if she so desires.
For a brief introduction to \acrshort{CTL*}, see \cite{meyer2020}.