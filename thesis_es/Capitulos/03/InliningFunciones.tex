\section{Inlining de funciones en la traducción a redes de Petri}
\label{sec:function-inlining}

En esta sección se presenta un análisis exhaustivo y la motivación de la tercera decisión de
diseño enumerada al principio del capítulo, a saber, el inlining de las llamadas a funciones.

El modelado de funciones en \acrshort{PN} es un aspecto crucial de la traducción porque es la unidad
básica del \acrshort{MIR}. Al representar las funciones en la \acrshort{MIR} como \acrshort{PN}
y conectarlas entre sí, el flujo de control y los datos compartidos
entre los hilos del programa pueden capturarse en un marco formal.
Posteriormente, la red de Petri es analizada por un verificador de modelos con el fin de
identificar posibles deadlocks o señales perdidas. Este enfoque es especialmente útil
cuando se trabaja con sistemas grandes y complejos que pueden tener muchos hilos y funciones
interrelacionados, en los que la situación de deadlock puede no ser evidente ni siquiera para un
revisor de código experimentado.

Al traducir funciones \acrshort{MIR} a \acrshort{PN},
una cuestión clave que se plantea es si reutilizar la misma
representación para cada llamada a una función específica o hacer un ``inline'' de la representación
correspondiente cada vez que se llama a la función. Expresado de otro modo, cada función
mapea a una subred en la \acrshort{PN} final obtenida tras la traducción, es decir, un subgrafo conectado
formado por los lugares y transiciones que modelan el comportamiento de la función
específica.
Esta parte más pequeña de la red puede estar presente sólo una vez en la \acrshort{PN} y todas
las llamadas a esta función se conectan a ella, o repetirse para cada instancia de una llamada a
la función en el código Rust.

Reutilizar el mismo modelo para cada función parece a primera vista más eficiente, ya que la
\acrshort{PN} obtenido es menor. Sin embargo, este enfoque también puede dar lugar a estados no válidos
que no estaban presentes en el programa Rust original. Éstos pueden ser la fuente de falsos
positivos durante la detección de bloqueos porque que estos estados extraños pueden violar las
garantías de seguridad ofrecidas por el compilador.

Por otro lado, un inline del modelo cada vez que se llama a una función da como resultado un \acrshort{PN}
más grande que requiere más memoria y tiempo de \acrshort{CPU} para ser analizado pero también
puede mejorar la precisión del análisis al garantizar que cada llamada a una función esté
representada por una estructura de red de Petri independiente que capture sus dependencias de
datos específicas en el contexto en el que se produce la llamada a la función en el código.

\subsection{El caso básico}

El impacto de estos sutiles detalles sólo puede comprenderse plenamente con un ejemplo
apropiado. En consecuencia, consideremos primero la abstracción más simple de una llamada a una
función en el lenguaje de las redes de Petri, formada por una sola transición y dos lugares que
representan el inicio y el final de la función.
Esto se ve en la Fig. \ref{fig:simplest-function}.
La llamada a la función se trata como una caja negra, todos los
detalles se abstraen en la transición. Sólo nos importa dónde empieza y dónde acaba la
función.

\begin{figure}[!htb]
    \centering
    \includesvg[width=0.3\linewidth]{simplest-function.svg}
    \caption{El modelo de red de Petri más simple posible para una llamada de función.}
    \label{fig:simplest-function}
\end{figure}

Observe ahora una función de este tipo en el contexto de un programa Rust.
El Listado \ref{lst:repeated-function-call} ofrece un ejemplo sencillo
en el que una función es llamada cinco veces consecutivas en un bucle \Rustinline{for}.
Una posible \acrshort{PN} que modele el programa se encuentra en la Fig. \ref{fig:repeated-function-call}.
Cabe destacar que esta red y las siguientes de esta sección \emph{no} son el resultado de una traducción de la MIR.
Son simplificaciones para ilustrar las dificultades de tratar con funciones llamadas en varios
lugares en el código.

\begin{listing}[!htb]
    \begin{minted}{Rust}
        fn simple_function() {}

        pub fn main() {
            for n in 0..5 {
                simple_function();
            }
        }
    \end{minted}
    \caption{Un programa sencillo de Rust con una llamada repetida a una función.}
    \label{lst:repeated-function-call}
\end{listing}

\begin{figure}[!htb]
    \centering
    \includesvg[width=0.7\linewidth]{repeated-function-call.svg}
    \caption{Una posible red de Petri para el código del Listado \ref{lst:repeated-function-call}
        aplicando el modelo de la Fig. \ref{fig:simplest-function}.}
    \label{fig:repeated-function-call}
\end{figure}

\subsection{Una caracterización del problema}

El escenario problemático no ha surgido hasta ahora. Sólo se manifiesta cuando se llama a una
función en al menos dos lugares distintos del código o, en términos más sencillos, cuando la
expresión \Rustinline{simple_function()} aparece dos veces o más.
El Listado \ref{lst:two-simple-function-calls} satisface esta condición y
está diseñado para mostrar el comportamiento extráneo descrito al principio de la sección.

\begin{listing}[!htb]
    \begin{minted}{Rust}
        fn simple_function() {}

        pub fn main() {
            let mut second_call = false;
            simple_function();
            if second_call {
                panic!()
            }
            second_call = true;
            simple_function();
        }
    \end{minted}
    \caption{Un sencillo programa Rust que llama a una función en dos lugares diferentes.}
    \label{lst:two-simple-function-calls}
\end{listing}

Como ya se ha dicho, el primer enfoque para modelar el programa consiste en reutilizar el
modelo de función para ambas llamadas. Esto se muestra en la Fig. \ref{fig:two-function-calls-incorrect-1}.

\begin{figure}[!htbp]
    \centering
    \includesvg[width=0.88\linewidth]{two-function-calls-incorrect-1.svg}
    \caption{Una primera red de Petri (incorrecta) para el código del Listado \ref{lst:two-simple-function-calls}.}
    \label{fig:two-function-calls-incorrect-1}
\end{figure}

Es evidente para el lector que el programa del Listado \ref{lst:two-simple-function-calls}
nunca llama a la macro \Rustinline{panic!} y siempre termina con éxito,
dado que la variable \Rustinline{second_call} nunca es \Rustinline{true} antes de la línea 9.

Sin embargo, la \acrshort{PN} representada en la Fig. \ref{fig:two-function-calls-incorrect-1} es conspicuamente defectuosa,
lo que la hace inadecuada como modelo para el programa. La razón es que después de disparar la transición
etiquetada \texttt{RETURN\_simple\_function} se coloca un token en \texttt{check\_flag} pero también en
\texttt{main\_end\_place}. El token en \texttt{main\_end\_place} aparecerá finalmente en \texttt{PROGRAM\_END}, lo
que indica una terminación normal del programa.
Esto es técnicamente correcto ya que sabemos que el programa termina con éxito.

No obstante, existen problemas que conciernen al segundo token.
El token en \texttt{check\_flag} puede ser consumido
por la transición \texttt{flag\_is\_false} o \texttt{flag\_is\_true}.
Si es consumida por esta última, se colocará un testigo en \texttt{PROGRAM\_PANIC},
señalando una terminación errónea del programa.
Esto es absurdo porque significa que el programa podría entrar en pánico pero
también que \emph{siempre} termina normalmente, como se ha visto en el párrafo anterior.

La situación empeora si seguimos el camino de disparar \texttt{flag\_is\_false}.
En ese caso, el token desencadena otra llamada a la función,
lo que en principio es correcto, pero nada impide que lo haga una y otra vez.
La conclusión es que podría acumularse una cantidad infinita de
tokens en \texttt{main\_end\_place} o \texttt{PROGRAM\_END} en el caso de que,
por pura casualidad, la transición \texttt{flag\_is\_true} no se dispare.

Está claro que debemos descartar este modelo y buscar una solución mejor. Una posibilidad es
dividir la transición etiquetada \texttt{RETURN\_simple\_function} en dos transiciones separadas según
el orden de llamada a la función, como se ilustra en la Fig. \ref{fig:two-function-calls-incorrect-2}.

\begin{figure}[!htbp]
    \centering
    \includesvg[width=0.94\linewidth]{two-function-calls-incorrect-2.svg}
    \caption{Una segunda red de Petri (también incorrecta) para el código del Listado \ref{lst:two-simple-function-calls}.}
    \label{fig:two-function-calls-incorrect-2}
\end{figure}

Desafortunadamente, este segundo intento viene acompañado de su propio conjunto de estados extraños.
En primer lugar, ahora el programa puede salir después de llamar a la función una sola vez.
Nada impide que la transición \texttt{RETURN\_simple\_function\_2} se dispare primero.
Esto equivale a decir que el flujo de ejecución salta de la línea 5 a la línea 11 en el Listado \ref{lst:two-simple-function-calls},
lo que obviamente no es una propiedad presente en el código Rust original.

Por otro lado, persiste el problema del bucle infinito.
La \acrshort{PN} puede seguir disparándose indefinidamente
mientras \texttt{flag\_is\_true} y \texttt{RETURN\_simple\_function\_2} no se disparen.
No hay ninguna garantía de que las transiciones se disparen en un orden específico.
Como se vio en la Sec. \ref{sec:transition-firing}, el disparo de las transiciones no es determinista.

\clearpage
\subsection{Una solución viable}

Una vez observadas las dificultades de modelar las llamadas de función, volvemos nuestra
atención al otro enfoque para modelar las llamadas de función:
La incrustación o \textit{inlining} de la representación \acrshort{PN}.
Algunas de las lecciones aprendidas de la subsección anterior son:

\begin{itemize}
    \item Crear un bucle en la red donde no lo hay en el programa original abre la puerta a
          secuencias infinitas de disparos de transición. Esto podría a su vez romper la
          propiedad de \emph{safety} de la \acrshort{PN}.
    \item Como el token simboliza el contador del programa,
          sólo debe haber un token en la \acrshort{PN} en cualquier momento dado.
    \item El estado del programa puede cambiar entre llamadas a funciones. En consecuencia,
          deben modelarse estos estados por separado. Dicho de otro modo, el estado al llamar a
          una función la primera vez puede no ser el mismo que al llamar a la función por segunda
          vez.
\end{itemize}

La Fig. \ref{fig:two-function-calls-correct} presenta el enfoque de inlining implementado en la herramienta.
La \acrshort{PN} en ella es correcta. Se ajusta mejor a la estructura del código Rust.
No contiene ningún bucle ni crea tokens adicionales al disparar transiciones,
es decir, ninguna de las transiciones tiene dos salidas.
Cabe mencionar que la \acrshort{PN} resultante es una máquina de estados (Definición \ref{definition:state-machines}),
tal y como se espera para un programa de un solo hilo.
No es el caso de las Fig. \ref{fig:two-function-calls-incorrect-1} y \ref{fig:two-function-calls-incorrect-2}.

\begin{figure}[!htbp]
    \centering
    \includesvg[width=\linewidth]{two-function-calls-correct.svg}
    \caption{Una red de Petri correcta para el código del Listado \ref{lst:two-simple-function-calls} utilizando inlining.}
    \label{fig:two-function-calls-correct}
\end{figure}

Una ventaja significativa del enfoque inlining es que cada llamada a una función se identifica
de forma inequívoca. Esto resulta útil a la hora de interpretar la salida del verificador de
modelos o los mensajes de error durante la traducción de un programa determinado. El uso de
un id incremental no negativo es arbitrario pero conveniente. Además, la precisión de la
detección del bloqueo se incrementa porque ciertas clases de estados extraños, como los del \acrshort{PN}
mostrado en la sección anterior, no están presentes. Minimizar el número de falsos positivos
desempeña un papel importante a la hora de considerar qué enfoque aplicar para una
herramienta que pretende ser fácil de usar y de configurar.

Una desventaja mencionada anteriormente es que el tamaño de la red resultante es mayor. La
penalización exacta en el número de lugares y transiciones adicionales depende de la
frecuencia con la que se reutilizan las funciones por término medio en el código base. Es
razonable suponer que las funciones se llaman desde varios lugares. Sin embargo, se pueden
aplicar ciertas optimizaciones que pueden reducir considerablemente el tamaño de la red,
compensando así el efecto de utilizar inlining. Estas optimizaciones se tratan en detalle en las
Sec. \ref{sec:future-work-petri-net-reduction} y \ref{sec:future-work-no-cleanup}.

Por último, un lector atento puede notar que
el análisis de la \acrshort{PN} de la Fig. \ref{fig:two-function-calls-correct} lleva a la
conclusión de que el programa puede llamar a \Rustinline{panic!} y terminar abruptamente,
lo que no coincide con la ejecución del programa Rust.
Esto es correcto pero es una limitación de las redes de Petri de bajo nivel
que no puede resolverse en el marco del modelo y va más allá del
alcance de este trabajo.
Sec. \ref{sec:future-work-higher-level-models} explora
las consecuencias de esta restricción y propone posibles soluciones.

Armados con nuevas ideas y conocimientos sobre las decisiones de diseño,
ahora estamos en condiciones de describir completamente la implementación.