\section{En busca de un backend}

Para ponerlo de forma sucinta, existen dos enfoques para traducir código Rust a redes de Petri. La primera
opción es crear un traductor desde cero, mientras que la segunda consiste en basarse en una
herramienta ya existente.

La primera opción puede parecer atractiva al principio, teniendo en cuenta que da al
desarrollador la libertad de moldear la herramienta según sus deseos. Se pueden añadir
funciones según las necesidades y adaptar las estructuras de datos al propósito específico. Sin
embargo, esta flexibilidad tiene un alto precio. Para dar soporte a un subconjunto razonable del
lenguaje de programación Rust, es necesario invertir grandes cantidades de esfuerzo en la tarea.
Las construcciones complejas del lenguaje, como las macros, los \textit{generics} o mismo el rico sistema
de tipos, deben ser comprendidas en sus detalles más intrincados para poder ser
traducidas con eficacia. El resultado es, esencialmente, un nuevo compilador para código Rust.
Teniendo en cuenta que el compilador de Rust se desarrolló a lo largo de muchos años y con el
apoyo de una gran comunidad de colaboradores, queda claro que este camino no es más que
una duplicación de trabajo. De hecho, es una labor hercúlea que requeriría la dedicación a
tiempo completo de un equipo completo para mantener al día de los cambios más recientes en el lenguaje Rust y
en el compilador.

Por otro lado, existe la posibilidad de integrarse con el compilador de Rust existente, que está
disponible bajo una licencia de código abierto y su documentación es extensa y se actualiza con
regularidad. Esto libera en parte a la implementación de tener que ocuparse de los cambios en
el lenguaje, lo que da más tiempo para centrarse en las características que añaden valor a los
usuarios. De ahí que el compilador desempeñe el papel de \emph{backend} en el que se apoya el
análisis estático. Por supuesto, esto requiere aprender las interioridades del compilador, pero no
es la primera vez que una herramienta se propone ello. A modo de ejemplo, el linter oficial de Rust,
\emph{clippy}\footnote{\url{https://github.com/rust-lang/rust-clippy}},
analiza el código Rust en busca de construcciones incorrectas, ineficaces o no
idiomáticas. Se trata de una herramienta muy valiosa para los desarrolladores que va más allá
de las comprobaciones estándar realizadas durante la compilación.

Dar soporte a todas las características del lenguaje desde el principio y colaborar con la comunidad es
clave para el éxito de la solución propuesta. Por lo tanto, es aconsejable integrarse con el
ecosistema existente y reutilizar todo el trabajo posible. Por todo ello, este proyecto se basa en
\emph{rustc}. A continuación estudiaremos con más detalle los componentes centrales del compilador de
Rust.