Una vez cubiertos los temas de fondo pertinentes, podemos proceder a profundizar en los
aspectos específicos del diseño del proceso de traducción. El diseño está marcado por tres
opciones arquitectónicas cruciales sobre las que se profundizará en este capítulo:

\begin{enumerate}
    \item La decisión de utilizar el compilador Rust como backend para la traducción.
    \item Basar la traducción en el \acrfull{MIR}.
    \item Hacer un \textit{inlining} de las llamadas a funciones en la red de Petri.
\end{enumerate}

A lo largo de este capítulo, analizaremos en profundidad los mecanismos internos del
compilador de Rust y sus etapas de compilación relevantes para este trabajo.