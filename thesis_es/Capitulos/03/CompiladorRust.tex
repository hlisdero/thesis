\section{El compilador de Rust: \emph{rustc}}
\label{sec:rustc}

El compilador de Rust, \emph{rustc}, se encarga de traducir el código Rust en código ejecutable. Sin
embargo, \emph{rustc} no es un compilador tradicional en el sentido de que realiza múltiples pasadas
sobre el código, como se describe en la Sec. \ref{sec:compiler-architecture}.
En su lugar, \emph{rustc} está construido sobre un
sistema basado en consultas que soporta la compilación incremental.

En el sistema de consulta de \emph{rustc}, el compilador calcula un grafo de dependencias entre los
artefactos de código, incluidos los archivos fuente, los crates y los artefactos intermedios, como
los archivos objeto. A continuación, el sistema de consulta utiliza este grafo para recompilar
eficientemente sólo aquellos artefactos que hayan cambiado
desde la última compilación\footnote{\url{https://rustc-dev-guide.rust-lang.org/queries/incremental-compilation.html}}.
Esta compilación incremental puede reducir significativamente el tiempo de compilación de grandes
proyectos, facilitando el desarrollo y la iteración del código Rust.

El sistema de consulta también permite al compilador de Rust realizar otras optimizaciones,
como la memoización y el almacenamiento en caché de resultados intermedios. Por ejemplo, si
el valor de retorno de una función se ha calculado antes, el sistema de consulta puede devolver
el resultado almacenado en caché en lugar de volver a calcularlo, lo que reduce aún más el
tiempo de compilación.

Otra elección de diseño importante en \emph{rustc} es el \emph{interning}. \emph{Interning} es una técnica para
almacenar cadenas de texto y otras estructuras de datos de forma eficiente en memoria. En lugar de
almacenar varias copias de la misma cadena o estructura de datos, el compilador de Rust
almacena sólo una copia en un \textit{allocator} especial llamado \emph{arena}.
Las referencias a los valores almacenados en la arena se pasan
de una parte a otra del compilador y pueden compararse de forma barata comparando punteros.
Esto permite reducir el uso de memoria y
acelerar las operaciones que comparan o manipulan cadenas de texto y estructuras de datos.

\emph{rustc} utiliza la infraestructura del compilador LLVM\footnote{\url{https://llvm.org/}}
para realizar la generación de código de bajo nivel y la optimización.
LLVM proporciona un marco flexible para compilar código a
una variedad de \emph{targets}, incluyendo código máquina nativo y \acrfull{WASM}.
El compilador de Rust utiliza LLVM para optimizar el código en términos de rendimiento y
generar código de alta calidad para una gran variedad de plataformas. En lugar de generar
código máquina, sólo necesita generar la \acrfull{IR} de LLVM del código
fuente y luego ordenar a LLVM que lo transforme al objetivo de compilación (\textit{compilation target}),
aplicando las optimizaciones deseadas.

\emph{rustc} está programado en Rust. Para compilar la versión más reciente del compilador y la
versión más reciente de la biblioteca estándar que lo acompaña, se utiliza una versión
ligeramente más antigua de \emph{rustc} y de la biblioteca estándar. Este proceso se denomina
\emph{bootstrapping} e implica que uno de los principales usuarios de Rust es el propio compilador de
Rust. Teniendo en cuenta que cada seis semanas se publica una nueva versión estable, el
bootstrapping implica una gran complejidad y se describe detalladamente
en la documentación\footnote{\url{https://rustc-dev-guide.rust-lang.org/building/bootstrapping.html}}
y en conferencias \cite{nelson2022} y tutoriales \cite{klock2022} hechos por miembros del Rust team.

\subsection{Etapas de compilación}

La existencia del sistema de consulta no implica que \emph{rustc} no tenga fases de compilación en
absoluto. Al contrario, se requieren varias etapas de compilación para transformar el código
fuente de Rust en código máquina que pueda ejecutarse en una computadora. Estas etapas implican
múltiples representaciones intermedias del programa, cada una optimizada para un propósito
específico. A continuación describiremos brevemente estas etapas. Encontrará una descripción
más completa en la documentación\footnote{\url{https://rustc-dev-guide.rust-lang.org/overview.html}}.

\subsubsection{Lexado y análisis sintáctico}

En primer lugar, el texto fuente en bruto de Rust es analizado por un \emph{lexer} de bajo nivel.
En esta etapa, el texto fuente se convierte en un flujo (\textit{stream}) de unidades atómicas de código fuente
conocidas como tokens.

A continuación se realiza el análisis sintáctico (\textit{parsing}).
El flujo de tokens se convierte en un \acrshort{AST}.
Aquí se produce el interning de los valores de cadena. La expansión de macros, la validación del
\acrshort{AST}, la resolución de nombres y el linting temprano también tienen lugar durante esta e t a p a .
La representación intermedia resultante de esta etapa es, a fin de cuentas, el \acrshort{AST}.

\subsubsection{HIR lowering}

Posteriormente, el \acrshort{AST} se convierte en \acrfull{HIR}. Este
proceso se conoce como ``rebajar'' (\textit{lowering}).
Esta representación se parece al código Rust pero con
construcciones complejas convertidas en versiones más simples.
Por ejemplo, todos los bucles \texttt{while} y \texttt{for} se convierten en
versiones más simples con bucles \texttt{loop}.

La \acrshort{HIR} se utiliza para realizar algunos pasos importantes:

\begin{enumerate}
      \item \emph{Inferencia de tipo (type inference)}: La detección automática del tipo de una expresión, por ejemplo, al
            declarar variables con \texttt{let}.
      \item \emph{Resolución de traits (trait solving)}: Garantizar que cada bloque de implementación (\texttt{impl}) hace
            referencia a un trait válido y existente.
      \item \emph{Comprobación de tipos (type checking)}: Este proceso convierte los tipos escritos por el usuario en la
            representación interna utilizada por el compilador. Es, en otras palabras, donde se
            internan los tipos. Después, utilizando esta información, se verifica la seguridad de tipos,
            la correctitud y la coherencia.
\end{enumerate}

\subsubsection{MIR lowering}

En esta etapa, la \acrshort{HIR} se rebaja a \acrfull{MIR}, que se
utiliza para el \emph{borrow checking}. Como parte del proceso, se construye la
\acrfull{THIR}, que es una representación más
fácil de convertir a \acrshort{MIR} que la \acrshort{HIR}.

La \acrshort{THIR} es una versión aún más ``desugarizada'' (\emph{desugared}) del \acrshort{HIR}.
Se utiliza para el \textit{pattern matching} y el \textit{exhaustiveness matching}.
Es similar a la \acrshort{HIR} pero con todos los tipos y llamadas a métodos explícitos.
Además se incluyen desreferencias implícitas cuando es necesario.

Muchas optimizaciones se realizan sobre la \acrshort{MIR}
puesto que sigue siendo una representación muy genérica.
Las optimizaciones son en algunos casos más fáciles de realizar sobre la \acrshort{MIR} que
sobre la posterior \acrshort{IR} de LLVM.

\subsubsection{Generación de código}

Esta es la última etapa en la producción de un binario. Incluye la llamada a LLVM para la
generación de código y las optimizaciones correspondientes.
Para ello, la \acrshort{MIR} se convierte en LLVM \acrshort{IR}.

LLVM \acrshort{IR} es la forma estándar de entrada para el compilador LLVM que utilizan todos los
compiladores que utilizan LLVM, como el compilador de C \emph{clang}. Es un tipo de lenguaje
ensamblador bien anotado y diseñado para que otros compiladores puedan producirlo
fácilmente. Además, está diseñado para ser lo suficientemente rico como para permitir a
LLVM realizar varias optimizaciones sobre él.

LLVM transforma el LLVM \acrshort{IR} a código máquina y aplica muchas más optimizaciones. Por
último, los archivos objeto que contienen código ensamblador pueden enlazarse (\emph{linking})
entre sí para formar el binario.

\subsection{Rust nightly}
\label{sec:rust-nightly}

Comprender el modelo de lanzamiento de versiones de Rust es
indispensable para implementar con éxito la herramienta propuesta en este trabajo.
La razón es que para utilizar las crates de \emph{rustc} como
dependencia en nuestro proyecto, debe compilarse con la versión \emph{nightly}.

El compilador nightly de Rust se refiere a una compilación específica de \emph{rustc} que se actualiza
cada noche con los últimos cambios y mejoras pero que también incluye características
experimentales o inestables que aún no forman parte de la versión estable. En Rust, el lenguaje
y su biblioteca estándar se versionan utilizando un modelo de ``tren de versiones'' (\textit{release train}),
en el que existen tres canales de versiones principales: estable, beta y nightly\footnote{\url{https://forge.rust-lang.org/}}.

La versión estable del compilador de Rust es la más utilizada y recomendada para su uso en
producción. Pasa por un riguroso proceso de pruebas y estabilización para garantizar que
proporciona una experiencia estable y fiable a los desarrolladores. La versión estable sólo
incluye características y mejoras que han sido revisadas a fondo, testeadas y consideradas lo
suficientemente estables para su uso en producción.

Por otro lado, el compilador nightly de Rust es la versión más \emph{bleeding-edge},
en la que se introducen a diario nuevas características,
correcciones de errores y cambios experimentales.
Es utilizado por los desarrolladores y colaboradores del lenguaje Rust con fines de prueba y desarrollo, pero
no se recomienda su uso en producción debido a la inestabilidad potencial y a la falta de
soporte a largo plazo.

Cada característica exclusiva de la versión nightly está detrás de una \emph{feature flag}.
Sólo pueden utilizarse al compilar con la \emph{toolchain} nightly. Las
feature flags pueden habilitar


\begin{itemize}
      \item construcciones sintácticas que no están disponibles en la versión estable,
      \item funciones de biblioteca exclusivas de la versión nocturna,
      \item soporte para instrucciones de hardware específicas de un \acrshort{ISA} o plataforma determinados,
      \item flags adicionales del compilador.
\end{itemize}

La lista completa de banderas de características se encuentra en \cite{unstable-book} y
contiene más de 500 entradas en total. De forma más concisa, el lenguaje Rust utilizado dentro
de \emph{rustc} es un superconjunto del lenguaje Rust estable utilizado fuera de él.
Estas diferencias deben tenerse en cuenta cuando se trabaje en el compilador
o se construya software que dependa directamente del compilador.