\section{Selección de un punto de partida adecuado para la traducción}
\label{sec:interception-strategy}

En esta sección, se elucidan los motivos para seleccionar la \acrfull{MIR}
como punto de partida para la traducción a una red de Petri. Esta elección
de diseño arquitectural se justifica por varias razones.

\subsection{Beneficios}

En primer lugar, la \acrshort{MIR} es la \acrshort{IR} más baja en rustc
que aún es independiente de la arquitectura de la computadora.
Captura la semántica del código Rust después de que haya sido sometido a una serie de pases
de optimización sin depender de los detalles de cualquier máquina en particular.
Al interceptar la traducción en esta fase, la herramienta de
análisis estático aprovecha las ventajas de estas optimizaciones,
como el plegado de constantes (\textit{constant folding}),
la eliminación de código muerto y el \textit{inlining}, lo que da como resultado una representación de
la red de Petri más eficiente y, en general, más pequeña.

En segundo lugar, interceptar la compilación una vez completadas las etapas anteriores ofrece
una ventaja en términos de eficiencia y reutilización del código. En esta etapa, el compilador de
Rust ya ha realizado pasos cruciales como el \emph{borrow checking}, la comprobación de
tipos, la monomorfización del código genérico y la expansión de macros, entre otros. Estos
pasos consumen muchos recursos e implican un análisis complejo del código Rust para
garantizar su correctitud y seguridad. Reimplementar estos pasos en nuestra herramienta desde
cero sería redundante y llevaría mucho tiempo. Requeriría duplicar los esfuerzos del
compilador de Rust y podría introducir posibles incoherencias o errores. Al construir sobre el
\acrshort{MIR} existente, aprovechamos al máximo el trabajo ya realizado por \emph{rustc}.
Esto no sólo ahorra esfuerzo,
sino que también alinea nuestra herramienta de análisis estático con el mismo nivel de
correctitud y seguridad que el compilador de Rust.

En tercer lugar, simplifica la tarea de mantenimiento de mantenerse al día con las continuas
incorporaciones al lenguaje Rust y a su compilador. Rust es un lenguaje en rápida evolución y
su compilador se actualiza constantemente con nuevas características, correcciones de errores y
mejores de performance. Reutilizar la \acrshort{MIR} significa que nuestra herramienta puede
beneficiarse de estas actualizaciones sin tener que implementar y mantener esos cambios de
forma independiente. Esto proporciona en general una solución de análisis estático más robusta
y fiable.

Adicionalmente, como se explicará en la siguiente sección, la \acrshort{MIR} se basa en el concepto de grafo de
flujo de control (\acrfull{CFG}), o en otras palabras, un tipo de grafo que se encuentra en los compiladores.
Esto significa que tanto la \acrshort{MIR} como las redes de Petri son representaciones gráficas, lo que hace
que la \acrshort{MIR} sea especialmente adecuada para una traducción. Tanto la \acrshort{MIR}  como las redes de Petri
pueden considerarse modelos gráficos que capturan las relaciones e interacciones entre
diferentes entidades. El grafo \acrshort{MIR} representa el flujo de ejecución subyacente dentro de un
programa Rust, mientras que una red Petri captura las transiciones de estado y las ocurrencias
de eventos en un sistema. Consecuentemente, resulta más fácil convertir la \acrshort{MIR} en una red de Petri,
dado que la estructura del grafo y las relaciones ya están presentes. Esto permite un proceso de
traducción más directo y eficiente sin tener que crear una estructura de grafos de la nada, lo que
resulta en una mejor integración entre el \acrshort{MIR} y el modelo de red de Petri para la detección de
deadlocks.

Para concluir, trabajar con la \acrshort{MIR} crea sinergias con la compilación incremental y el análisis
modular. De hecho, una de las razones por las que se introdujo MIR en primer lugar fue la
compilación incremental \cite{matsakis2016mir}. Aunque no es obligatorio en la implementación
inicial, la herramienta podría beneficiarse de la compilación incremental y realizar análisis por
crate/por módulo, lo que permitiría un análisis más rápido y eficiente de grandes bases de
código Rust.

\subsection{Limitaciones}

A pesar de los numerosos beneficios, el enfoque de basar la traducción
en la \acrfull{MIR} tiene algunas limitaciones.

La más importante es que la \acrshort{MIR} está sujeta a cambios. No se ofrecen garantías de estabilidad
en cuanto a cómo se traducirá el código Rust a \acrshort{MIR} o cuáles son sus elementos constitutivos.
Se trata de detalles internos que los desarrolladores del compilador se reservan para sí
mismos. En resumen, la \acrshort{MIR} como interfaz no es estable. A medida que se sigue trabajando en
el compilador, la \acrshort{MIR} sufre modificaciones para incorporar nuevas características del lenguaje,
optimizaciones o correcciones de errores, lo que puede requerir frecuentes actualizaciones y
ajustes en el proceso de traducción, aumentando el coste de mantenimiento.

En el transcurso de este proyecto esta situación se produjo en varias ocasiones. A
modo de ejemplo, en el periodo comprendido entre mediados de febrero de 2023 y mediados
de abril de 2023, el código se modificó 7 veces para dar cabida a estos cambios. Siempre
fueron de unas pocas líneas de código y se detectaron mediante pruebas. Hablaremos de cómo
las pruebas desempeñan un papel importante a la hora de lidiar con estos cambios en la
Sec. \ref{sec:integration-tests}.

En la misma línea, \cite{meyer2020} también se basó en la \acrshort{MIR} pero no incorporó pruebas para
hacer frente a las versiones nightly más recientes. Como resultado, la cadena de herramientas
se fijó a una versión nightly exacta\footnote{\url{https://github.com/Skasselbard/Granite/blob/master/rust-toolchain}}
para evitar que la implementación se rompiera antes de la publicación de la tesis.

Otro inconveniente digno de mención es que, en algunos casos, el código genérico podría
adoptar la forma de una función cuyo comportamiento puede ser modelado por la misma red de
Petri en todos los casos. En estas circunstancias, el \acrshort{MIR} podría ``condensarse'' aún más antes de
traducirlo a una red de Petri. Del mismo modo, algunas partes del MIR pueden ser superfluas
para el análisis de detección de bloqueo y su traducción puede agrandar la salida, lo que
ralentiza el análisis de alcanzabilidad realizado por el verificador de modelos. Esto puede
contrarrestarse con optimizaciones cuidadosas
que se propondrán en las Sec. \ref{sec:future-work-petri-net-reduction} y \ref{sec:future-work-no-cleanup}.

\subsection{Síntesis}

En conclusión, a pesar de los inconvenientes mencionados anteriormente, interceptar la
traducción en el nivel \acrshort{MIR} ofrece ventajas significativas, entre las que se incluyen la
maximización de la utilización del código del compilador existente, la reducción del esfuerzo
de implementación y un mapeo más natural a las redes de Petri. Estas ventajas superan a los
contras y hacen de la \acrshort{MIR} un punto de partida convincente para la traducción en el contexto
de la construcción de una herramienta de análisis estático para detectar deadlocks y señales
perdidas en el código Rust.

Tanto \cite{meyer2020} como \cite{zhang2022deadlocks} basan también sus traducciones en la \acrshort{MIR} y,
hasta donde sabe este autor, no existe ninguna herramienta análoga que realice una traducción
a redes de Petri partiendo de una representación intermedia de nivel superior.