Esta tesis ha explorado la traducción de programas Rust a modelos de redes de Petri con el fin
de detectar deadlocks y señales perdidas. A lo largo del estudio, se han examinado
diversos aspectos del proceso de traducción, incluido el manejo de llamadas a funciones,
hilos, mutexes y variables de condición. El traductor que hemos desarrollado ha demostrado su
capacidad para capturar con precisión el comportamiento de concurrencia y sincronización de
programas Rust más bien sencillos.

El enfoque de traducción presentado en esta tesis ha mostrado resultados prometedores,
modelando y detectando con éxito bloqueos en una serie de programas de prueba que
comprenden incluso dos problemas clásicos de la programación concurrente. Aprovechando el
poder expresivo de las redes de Petri, el traductor proporciona una representación visual del
comportamiento del programa, facilitando la identificación de posibles problemas de
sincronización. Y lo que es más importante, la traducción produce un modelo que puede ser
analizado por una miríada de herramientas de comprobación de modelos,
dándole una aplicación práctica al trabajo académico existente
para aportar soluciones a los problemas de la industria.
La incorporación de un modelo sucinto para las variables de condición mejora las capacidades de
modelado y permite la detección de señales perdidas que son una clase más intrincada de
deadlock en los sistemas concurrentes.

De cara al futuro, existen varias vías de investigación y mejora.
Una dirección potencial es la exploración de programas más complejos y aplicaciones del mundo real
para evaluar la escalabilidad y eficacia del enfoque de traducción.
Además, un mayor refinamiento y optimización de los algoritmos de traducción
podría mejorar la eficacia del análisis,
especialmente modelos de más alto nivel
que permitirían modelar la memoria con mayor eficacia.

En general, esta tesis supone una contribución significativa al desarrollar un traductor
que tiende un puente entre los programas Rust y las redes de Petri.
Los conocimientos obtenidos en esta investigación arrojan luz sobre los retos
y las oportunidades de modelar y analizar sistemas concurrentes en tiempo de compilación.
Idealmente, un lenguaje de programación cuyo compilador detectara los problemas de concurrencia
sería una bendición para muchas aplicaciones.
Construyendo sobre los puntos fuertes de las redes de Petri,
esta posibilidad podría volverse un poco más real en el lenguaje de programación Rust.

En otro orden de c o s a s , la contribución de esta tesis va
más allá de los beneficios inmediatos del traductor propuesto y sus capacidades.
Al proporcionar una base sólida y bien documentada
para la traducción de programas Rust a redes de Petri, este trabajo pretende hacer una
contribución significativa a la comunidad Rust en su conjunto.
Sirve de trampolín para futuros emprendimientos,
ofreciendo una base fiable sobre la que pueden construirse otras herramientas y
proyectos de investigación.
Abre nuevas posibilidades para explorar el análisis y la verificación
de programas Rust concurrentes utilizando redes de Petri.
Esto, a su vez, tiene el potencial de impulsar nuevos avances en el campo,
estimulando la innovación y promoviendo una
comprensión más profunda de la programación concurrente en Rust.
Con su documentación completa y su implementación clara,
el traductor no sólo facilita su uso inmediato,
sino que también sirve como valioso recurso para quienes estén interesados
en estudiar o ampliar las técnicas de traducción empleadas.
En última instancia, este trabajo aspira a encender la curiosidad e
inspirar nuevas contribuciones al ecosistema Rust,
fomentando la colaboración y el crecimiento de la comunidad.