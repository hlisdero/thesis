\section{Formatos de archivo para intercambio de redes de Petri}

Como se ha observado en el capítulo anterior, las redes de Petri son una herramienta muy
utilizada para modelar sistemas de software. Sin embargo, debido a las diferentes clases de
redes de Petri (redes de Petri simples, redes de Petri de alto nivel, redes de Petri con tiempo,
redes de Petri estocásticas, redes de Petri coloreadas, por nombrar algunas), diseñar un
formato de archivo de intercambio estandarizado compatible con todas las aplicaciones ha
resultado todo un reto. Una de las razones es que las redes de Petri pueden implementarse y
representarse de múltiples formas, en función de los objetivos específicos, visto que son un tipo
de grafo.

Para garantizar un cierto grado de interoperabilidad entre la herramienta desarrollada en el
marco de esta tesis y otras herramientas existentes y futuras, es primordial investigar qué
formatos de archivo sería más conveniente soportar. El objetivo es admitir formatos de archivo
que sean adecuados tanto para el análisis como para la visualización, permitiendo la posibilidad
de ampliación a formatos adicionales en el futuro, a través de una API bien definida en la
biblioteca de redes de Petri. Una revisión de la literatura condujo a tres formatos de archivo
relevantes que se presentan a continuación.

\subsection{Petri Net Markup Language}

El \acrfull{PNML}\footnote{\url{https://www.pnml.org/}}
es un formato de archivo estándar diseñado
para el intercambio de redes de Petri entre distintas herramientas y aplicaciones de software. Su
desarrollo se inició en el ``Meeting on XML/SGML based Interchange Formats for Petri Nets''
celebrada en Aarhus en junio de 2000 \cite{jungel2000petri,weber2003petri}
con el objetivo de proporcionar un formato estandarizado y ampliamente
aceptado para redes de Petri. PNML es una norma ISO que consta, a partir de
2023, de tres partes:

\begin{itemize}
      \item ISO/IEC 15909-1:2004\footnote{\url{https://www.iso.org/standard/38225.html}}
            (y su última revisión ISO/IEC 15909-1:2019\footnote{\url{https://www.iso.org/standard/67235.html}})
            para conceptos, definiciones y notación gráfica.
      \item ISO/IEC 15909-2:2011\footnote{\url{https://www.iso.org/standard/43538.html}}
            para la definición de un formato de transferencia basado en XML.
      \item ISO/IEC 15909-3:2021\footnote{\url{https://www.iso.org/standard/81504.html}}
            para las extensiones y los mecanismos de estructuración.
\end{itemize}

Se ha convertido en un estándar \emph{de facto} para intercambiar modelos en redes de Petri entre
diferentes herramientas y sistemas. Es el resultado de muchos años de duro trabajo para
unificar la notación, tal y como se expone en \cite{hillah:hal-01176335}.

\acrshort{PNML} ha sido diseñado para ser un formato flexible y extensible que pueda representar
diferentes clases de redes de Petri, incluidas las redes de Petri simples y las redes de Petri de
alto nivel. Se basa en \acrfull{XML}, lo que facilita su lectura y
análisis tanto por humanos como por máquinas. Además, \acrshort{PNML} admite el uso de metadatos
para proporcionar información adicional sobre los modelos de redes de Petri, como la autoría,
la fecha de creación e información sobre licencias.

has been designed to be a flexible and extensible format
that can represent different classes of Petri nets,
including simple Petri nets and high-level Petri nets.
It is based on the
which makes it easy to read and parse by humans and machines alike.
Additionally,  supports the use of metadata
to provide additional information about the Petri net models,
such as authorship, date of creation, and licensing information.

El desarrollo de \acrshort{PNML} ha mejorado significativamente la interoperabilidad y el intercambio de
modelos de redes de Petri entre diferentes herramientas y sistemas. Antes de la adopción de
\acrshort{PNML} el intercambio de modelos de redes de Petri era una tarea ardua puesto que las distintas
herramientas utilizaban formatos propietarios que a menudo eran incompatibles entre sí. El
\acrshort{PNML} ha simplificado enormemente este proceso, permitiendo a investigadores y profesionales
compartir y colaborar en modelos de redes de Petri con facilidad. Su uso también
ha facilitado el desarrollo de nuevas herramientas y aplicaciones de software para redes de
Petri porque proporciona un formato estándar que puede ser analizado y procesado fácilmente
por distintos sistemas. En particular es el formato utilizado en \cite{zhang2022deadlocks} y está
soportado en \cite{meyer2020}.

\subsection{Formato GraphViz DOT}

El formato DOT es un lenguaje de descripción de grafos utilizado para crear representaciones
visuales de grafos y redes, que forma parte de la suite de código abierto GraphViz\footnote{\url{https://graphviz.org/}}.
Fue creado a principios de la década de 1990 en AT\&T Labs Research como un lenguaje sencillo, conciso y
legible por humanos para la descripción de grafos. La suite GraphViz proporciona varias
herramientas para trabajar con archivos DOT, incluida la capacidad de generar
automáticamente diseños para gráficos complejos y de exportar visualizaciones en diversos
formatos, como PNG, PDF y SVG.

DOT puede utilizarse para representar redes de Petri en un formato gráfico, lo que facilita la
visualización de la estructura y el comportamiento del sistema que se está modelando.
Resulta especialmente útil para visualizar redes de Petri de gran tamaño, ya que el usuario puede
navegar por la imagen para comprender cómo fluyen las marcas por la red.

El formato DOT está basado en texto plano y es fácil de usar, conviertiéndolo en una opción
popular para generar representaciones visuales de gráficos.
Esta facilidad también significa que los archivos DOT pueden ser generados fácilmente por programas y pueden ser leídos por una
amplia gama de herramientas de software, un aspecto esencial para la interoperabilidad.
Además, DOT permite especificar diversas propiedades de los grafos, como formas de
nodos, colores y estilos \cite{dot2015} que pueden utilizarse para representar diferentes
aspectos de una red de Petri, como lugares, transiciones y arcos. Esta flexibilidad a la hora de
especificar las propiedades visuales también permite a los usuarios personalizar la
visualización según sus necesidades y resaltar características particulares de la red de Petri que
sean relevantes para su análisis.

\subsection{LoLA - Low-Level Petri Net Analyzer}
\label{sec:lola-format}

\acrfull{LoLA} \cite{schmidt2000lola} es un verificador de modelos de
última generación cuyo desarrollo comenzó en 1998 en la Universidad Humboldt de Berlín.
Actualmente lo mantiene la Universidad de Rostock y se publica bajo la Licencia Pública
General Affero de GNU. \acrshort{LoLA} es una herramienta que puede comprobar si un sistema
satisface una propiedad dada expresada en \acrfull{CTL*}. Su punto
fuerte es la evaluación de propiedades sencillas como la libertad de bloqueo (\textit{deadlock freedom}) o la
alcanzabilidad, tal y como se indica en la página web.
Este es el verificador de modelos utilizado en \cite{meyer2020} y en este trabajo.
En consecuencia, es
necesario implementar el formato de archivo requerido por la herramienta.
En la Sec. \ref{sec:integration-tests} se presentan ejemplos.