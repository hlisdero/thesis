\section{Verificación formal de código Rust}

Existen numerosas herramientas de verificación automática disponibles para el código Rust.
Una primera aproximación recomendable al tema es la encuesta elaborada por Alastair Reid,
investigador de Intel. En ella se enumera explícitamente que la mayoría de las herramientas de
verificación formal no soportan la concurrencia \cite{reid2021}.

El intérprete \emph{Miri}\footnote{\url{https://github.com/rust-lang/miri}}
desarrollado por el proyecto Rust en GitHub es un intérprete experimental
para la representación intermedia del lenguaje Rust (``Mid-level Intermediate Representation'',
comúnmente conocida como ``\acrshort{MIR}'') que permite ejecutar binarios estándar de proyectos de
\emph{cargo} en una forma granularizada, instrucción por instrucción, para comprobar la ausencia de
Comportamientos Indefinidos (\acrfull{UB}) y otros errores en el manejo de la memoria. Detecta fugas
de memoria, accesos a memoria no alineados, carreras de datos y violaciones de
precondiciones o invariantes en el código marcado como inseguro (\texttt{unsafe}).

\cite{toman2015crust} presenta un verificador formal para Rust que no requiere modificaciones en el
código fuente. Se probó en versiones anteriores de módulos de la biblioteca estándar de Rust. Como
resultado se detectaron errores en el uso de la memoria en código Rust \texttt{unsafe} que,
en circunstancias reales, el equipo de desarrollo tardó meses en descubrir manualmente.
Esto ejemplifica la importancia de utilizar herramientas
de verificación automática para complementar las revisiones manuales del código (\textit{code reviews}).

\cite{kani2023} es otra herramienta popular para la verificación formal de código Rust
destinada a comprobar los bloques inseguros (\texttt{unsafe}) a nivel de bits.
Ofrece un framework de pruebas análogo al framework de pruebas proporcionado por Rust.
Además, dispone de un plugin para \emph{cargo} y VS Code.

Como explica la documentación del repositorio\footnote{\url{https://github.com/model-checking/kani}}, Kani verifica (entre otros):

\begin{itemize}
  \item Uso seguro de la memoria, por ejemplo, desreferencias de punteros nulos (\textit{Memory safety, e.g., null pointer dereferences})
  \item Aserciones especificadas por el usuario (\textit{User-specified assertions, i.e., \Rustinline{assert!(...)}})
  \item La ausencia de panics, por ejemplo, \Rustinline{unwrap()} en valores \Rustinline{None}
  \item La ausencia de algunos tipos de comportamiento inesperado, por ejemplo, desbordamientos aritméticos (\textit{arithmetic overflows})
\end{itemize}

Sin embargo, los programas concurrentes están actualmente fuera de alcance\footnote{\url{https://model-checking.github.io/kani/rust-feature-support.html}}.
La conclusión es que Kani ofrece una \acrshort{CLI} fácil de usar y un framework de pruebas que se integran perfectamente en
el proceso de desarrollo. Sirve como ilustración de las capacidades de la comprobación de
modelos en el desarrollo de software moderno.