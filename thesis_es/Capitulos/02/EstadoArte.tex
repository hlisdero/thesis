En este capítulo, se revisa brevemente la literatura sobre la verificación formal del código Rust
y el modelado de redes de Petri para la detección de deadlocks. Algunas de estas publicaciones
anteriores contienen enfoques que han guiado este trabajo.

En las dos secciones sucesivas examinaremos las herramientas existentes, su alcance y sus
objetivos en comparación con la herramienta desarrollada en esta tesis.

A continuación, se ofrece un estudio de las bibliotecas de redes de Petri existentes en el
ecosistema Rust a principios de 2023 para justificar la necesidad de implementar una biblioteca
desde cero.

Posteriormente, exploramos la comunidad de investigadores detrás del Concurso
de Verificación de Modelos (\acrfull{MCC}) y los verificadores de modelos que participan en él
para confirmar el potencial de estas herramientas para analizar modelos de redes de Petri de
tamaño significativo. Esto es relevante ya que el verificador de modelos actúa como backend
de la herramienta desarrollada en este trabajo.

Por último, se presentan tres de los formatos de archivo existentes para el intercambio de redes
de Petri y se explica su finalidad en el contexto de este trabajo.