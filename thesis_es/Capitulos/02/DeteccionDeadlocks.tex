\section{Detección de deadlocks mediante redes de Petri}

La prevención de los bloqueos es una de las estrategias clásicas para abordar este problema
fundamental en la programación concurrente, como se explica en la Sec. \ref{sec:deadlock-strategies}.
El principal problema del enfoque de detectar los bloqueos antes de que se produzcan es probar que se
detecta el tipo de bloqueo deseado en todos los casos y que no se producen falsos negativos en
el proceso. El enfoque basado en redes de Petri, al ser un método formal, satisface estas
condiciones. Sin embargo, la dificultad de su adopción radica principalmente en la
practicabilidad de la solución debido al gran número de estados posibles en un proyecto de
software real.

En \cite{karatkevich2014deadlock}, se propone un método para reducir el número de estados
explorados durante la detección de bloqueos mediante el análisis de alcanzabilidad. Esta
heurística ayuda a mejorar el rendimiento del enfoque basado en redes de Petri. Otra
optimización se presenta en \cite{kungas2005petri}. El autor propone un método de orden polinómico
muy prometedor para evitar el problema de la explosión de estados que subyace al algoritmo
ingenuo de detección de deadlocks.
A través de un algoritmo que abstrae una red de Petri dada a una representación más simple, se obtiene una
jerarquía de redes de tamaño creciente para las que la verificación de la ausencia de bloqueos es
sustancialmente más rápida. Se trata, dicho crudamente, de una estrategia de ``divide y
vencerás'' que comprueba la ausencia de deadlocks en partes de la red para construir después la
verificación del conjunto final añadiendo partes a la pequeña red inicial.

A pesar de las advertencias mencionadas anteriormente, el uso de las redes de Petri como
método formal de verificación de software se ha establecido desde finales de la década de
1980. Las redes de Petri permiten un modelado intuitivo de las primitivas de sincronización,
como el envío de un mensaje o la espera de la recepción de un mensaje.
En \cite{heiner1992petri} encontrará ejemplos de estas sencillas redes con un comportamiento correspondientemente
simple. Estas redes son bloques de construcción que pueden combinarse para formar un
sistema más complejo.

Para poner en práctica estos modelos, existen dos posibilidades:

\begin{itemize}
      \item Una es diseñar el sistema en términos de redes de Petri
            y luego traducir las redes de Petri al código fuente.
      \item La otra consiste en traducir el código fuente existente
            a una representación de red de Petri y, a continuación,
            verificar que el modelo de red de Petri satisface las propiedades deseadas.
\end{itemize}

A efectos de este trabajo, nos interesa esta última. Este enfoque no es novedoso. Ya se ha
implementado para otros lenguajes de programación como C y Rust, como se ve en la
bibliografía existente.

En \cite{kavi2002modeling} y \cite{moshtaghi2001}, se describe una traducción de algunas primitivas de
sincronización disponibles como parte de la biblioteca POSIX de hilos (\texttt{pthread}) en C a redes
de Petri. En concreto, la traducción admite:

\begin{itemize}
      \item La creación de hilos con la función \texttt{pthread\_create} y el manejo de la variable de tipo
            \texttt{pthread\_t}.
      \item La operación de unión de hilos con la función \texttt{pthread\_join}.
      \item La operación de adquirir un mutex con \texttt{pthread\_mutex\_lock} y su eventual liberación
            manual con \texttt{pthread\_mutex\_unlock}.
      \item Las funciones \texttt{pthread\_cond\_wait} y \texttt{pthread\_cond\_signal}
            para trabajar con condition variables.
\end{itemize}

Lamentablemente, el código fuente de esta biblioteca llamada ``C2Petri'' no se encuentra en
línea, ya que las publicaciones son bastante antiguas.

En una tesis de máster más reciente, \cite{meyer2020} establece las bases de una semántica de
redes de Petri para el lenguaje de programación Rust. Sin embargo, centra sus esfuerzos en el
código de un solo hilo, limitándose a la detección de los deadlocks causados por la ejecución
de la operación de \texttt{lock} dos veces sobre el mismo mutex en el hilo principal.
Desafortunadamente, el código disponible en GitHub\footnote{\url{https://github.com/Skasselbard/Granite}}
como parte de la tesis ya no es válido para la nueva versión de \emph{rustc},
puesto que las partes internas del compilador han cambiado significativamente en los últimos tres años.

En un \emph{pre-print} de finales de 2022, \cite{zhang2022deadlocks} implementan una traducción del
código fuente de Rust a redes de Petri para encontrar deadlocks. La traducción se
centra en los bloqueos muertos causados por dos tipos de bloqueos de la biblioteca estándar:
\texttt{std::sync::Mutex} y \texttt{std::sync::RwLock}.
La red de Petri resultante se expresa en el lenguaje de
marcado de redes de Petri (\acrfull{PNML}) y se introduce en el
verificador de modelos \acrfull{PIPE2}\footnote{\url{https://pipe2.sourceforge.net/}}
para realizar el análisis de alcanzabilidad.
Las llamadas a funciones se manejan de una forma muy diferente a la de este trabajo y las señales perdidas
no se modelan en absoluto. El código fuente de su herramienta, denominada TRustPN, no está
disponible públicamente en el momento de escribir este artículo. A pesar de estas limitaciones,
los autores ofrecen un estudio muy detallado y actualizado de las herramientas de análisis
estático para la verificación de código Rust que podría resultar atractivo para el lector interesado
en ahondar en esta temática. Además, enumeran varios trabajos dedicados a formalizar la semántica del lenguaje
de programación Rust que quedan fuera del alcance de este trabajo.