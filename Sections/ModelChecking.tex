% !TeX root = ../Thesis.tex
\documentclass[../Thesis.tex]{subfiles}
\graphicspath{{\subfix{../images/}}}

\begin{document}

\section{Model checking}

Model checking is a technique used in software development
to formally verify the correctness of a system's behavior
with respect to its specifications or requirements.
It involves constructing a mathematical model of the system
and analyzing it to ensure that it meets certain properties,
such as mutual exclusion when accessing shared resources,
absence of data races and deadlock-freedom.

The process of model checking begins by constructing a finite-state model of the system,
typically using a formal language, in the case of this work the language of Petri nets.
The model captures the system's behavior and the properties that are to be verified.
The next step is to perform an exhaustive search of the state space of the model
to ensure that all possible behaviors have been considered.
This search can be performed automatically using specialized software tools.

During the search, the model checker looks for counterexamples,
which are sequences of events that violate the system's specifications.
If a counterexample is found, the model checker provides information
on the state of the system at the time of the violation,
helping developers to identify and fix the problem.

Model checking has become an important technique
in the development of critical software systems,
such as aerospace \cite{carreno2005safety,monzon2009deadlock}
and automotive control systems \cite{Perronnet2019},
medical devices, and financial systems.
By verifying the correctness of the software before it is deployed,
developers can ensure that the system meets its requirements and is safe to use.

One of the main advantages of model checking is
that it provides a formal and rigorous approach to verifying software correctness.
Unlike traditional testing methods, which can only demonstrate the presence of errors,
model checking can prove the absence of errors.
This is particularly important for safety-critical systems like the ones mentioned before,
where a single error can have catastrophic consequences for human lives.
Model checking can also be automated, allowing developers to quickly
and efficiently verify the correctness of complex software systems.
This reduces the time and cost of software development
and increases confidence in the correctness of the system.

It is known that formal software verification tools are currently applied
in a few very specific fields where
formal proof of the correctness of the system is required.
\cite{reid:hatra:2020} discusses the importance of bringing verification
tools closer to developers through an approach
that seeks to maximize the cost-benefit ratio of its use.
Improvements in the usability of existing tools and
approaches to incorporate their use into the developer's routine are presented.
The paper starts from the premise that from the developer's point of view,
verification can be seen as a different type of unit or integration test.
Therefore, it is of utmost importance that running the verification is as easy as possible
and feedback is provided to the developer quickly during the development process
to increase adoption.

In this work, we develop a formal verification tool
for the Rust programming language that is able to detect some classes of deadlocks
and missed signals in the source code at compile time.
This tool can help developers quickly verify
that their use of synchronization primitives
is correct and will not cause hard-to-detect bugs later on.
Our goal is to make the tool user-friendly and easy to get started with,
so that its adoption benefits the larger community of Rust developers.

In the next two sections, we will look at the existing tools,
their scope and their goals compared to the tool we developed.

\subsection{Formal verification of Rust code}

There are numerous automatic verification tools available for Rust code.
A recommended first approximation to the topic is
the survey produced by Alastair Reid, a researcher at Intel.
It explicitly lists that most formal verification tools
do not support concurrency \cite{reid2021}.

The \emph{Miri}\footnote{\url{https://github.com/rust-lang/miri}} interpreter
developed by the Rust project on GitHub is an experimental interpreter
for the intermediate representation of the Rust language (\acrfull{MIR})
that allows executing standard cargo project binaries
in a granularized way, instruction by instruction,
to check for the absence of \acrfull{UB}
and other errors in memory handling.
It detects memory leaks, unaligned memory accesses, data races,
and precondition or invariant violations in code marked as \texttt{unsafe}.

\cite{toman2015crust} introduces a formal checker for Rust
that does not require modifications to the source code.
It was tested on past versions of modules from the Rust standard library.
As a result, errors were detected in the use of memory in unsafe Rust code
which in reality took months to be discovered manually by the development team.
This exemplifies the importance of using automatic verification tools
to complement manual code reviews.

\subsection{Deadlock detection using Petri nets}

Deadlock prevention is one of the classic strategies
to address this crucial problem in concurrent programming,
as discussed in Sec. \ref{sec:deadlock-strategies}.
The main problem with the approach of detecting deadlocks before they occur
is proving that the desired type of deadlock is detected in all cases
and that no false negatives are produced in the process.
The Petri net-based approach, being a formal method, satisfies these conditions.
However, the difficulty of adoption lies mainly in the practicability of the solution
due to the large number of possible states in a real software project.

In \cite{karatkevich2014deadlock}, a method is proposed
to reduce the number of explored states
during the detection of deadlocks using reachability analysis.
These heuristics help improve the performance of the Petri net-based approach.
Another optimization is presented in \cite{kungas2005petri}.
The author proposes a very promising polynomial order method to avoid the problem
of the state explosion that underlies the naïve deadlock detection algorithm.
Through an algorithm that abstracts a given Petri net to a simpler representation,
a hierarchy of networks of increasing size is obtained
for which the verification of the absence of deadlocks is substantially faster.
It is, crudely put, a ``divide and conquer'' strategy
that checks for the absence of deadlocks in parts of the network
to later build the verification of the final whole
by adding parts to the initial small network.

Despite the previously mentioned caveats,
the use of Petri nets as a formal software verification method
has been established since the late 1980s.
Petri nets allow for intuitive modeling of synchronization primitives,
such as sending a message or waiting for the reception of a message.
Examples of these simple nets with correspondingly simple behavior
are found in \cite{heiner1992petri}.
These nets are construction blocks that can be combined to form a more complex system.

To put these models to use, there are two possibilities:

\begin{itemize}
      \item One is designing the system in terms of Petri nets
            and then translating the Petri nets to the source code.
      \item The other one is to translate the existing source code to a Petri net representation
            and then verify that the Petri net model satisfies the desired properties.
\end{itemize}

For the purposes of this work, we are interested in the latter.
This approach is not novel.
It has been implemented for other programming languages like C and Rust already,
as seen in the literature.

In \cite{kavi2002modeling} and \cite{moshtaghi2001},
a translation of some synchronization primitives available as part of
the POSIX library of threads (\texttt{pthread}) in C to Petri nets is described.
In particular, the translation supports:

\begin{itemize}
      \item The creation of threads with the function \texttt{pthread\_create}
            and the handling of the variable of type \texttt{pthread\_t}.
      \item The thread join operation with the \texttt{pthread\_join} function.
      \item The operation of acquiring a mutex with \texttt{pthread\_mutex\_lock}
            and its eventual manual release with \texttt{pthread\_mutex\_unlock}.
      \item The \texttt{pthread\_cond\_wait} and \texttt{pthread\_cond\_signal} functions
            for working with condition variables.
\end{itemize}

In his master's thesis, \cite{meyer2020} establishes
the bases for a Petri net semantic for the Rust programming language.
He focuses his efforts however on single-threaded code,
limiting himself to the detection of deadlocks caused by
executing the \texttt{lock} operation twice on the same mutex in the main thread.
Unfortunately, the code available as part of the thesis is
no longer valid for the new version of the \emph{rustc} compiler,
since the internals of the compiler changed significantly in the last two years.

\end{document}