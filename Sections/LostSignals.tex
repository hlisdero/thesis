% !TeX root = ../Thesis.tex
\documentclass[../Thesis.tex]{subfiles}
\graphicspath{{\subfix{../images/}}}

\begin{document}

\section{Condition variables}

Condition variables are a synchronization primitive in concurrent programming
that allows threads to efficiently wait
for a specific condition to be met before proceeding.
They were first introduced by \cite{hoare1974monitors}
as part of a building block for the concept of monitor developed
originally by \cite{hansen1973operating}.

Following the classic definition, two main operations
can be called on a condition variable:

\begin{itemize}
    \item \texttt{wait}: Blocks the current thread or process.
          In some implementations, the associated mutex
          is released as part of the operation.
    \item \texttt{signal}: Wakes up one thread or process
          waiting on the condition variable.
          In some implementations, the associated mutex lock is immediately
          acquired by the signaled thread or process.
\end{itemize}

Condition variables are typically associated
with a boolean predicate (a condition) and a mutex.
The boolean predicate is the condition
on which the threads or processes are waiting for.
When it is set to a particular value (either true or false),
the thread or process should continue executing.
The mutex ensures that only one thread or process
may access the condition variable at a time.

Condition variables do not contain an actual value
accessible to the programmer inside of them.
Instead, they are implemented using a queue data structure, where
threads or processes are added to the queue when they enter the wait state.
When another thread or process signals the condition,
an element from the queue is selected to resume execution.
The specific scheduling policy may vary depending on the implementation.

Over the years, various implementations and optimizations have been developed
for condition variables to improve performance and reduce overhead.
For example, some implementations allow multiple threads to be awakened at once
(an operation called \emph{broadcast}),
while others use a priority queue
to ensure that the most important threads are awakened first.

Condition variables are part of the POSIX standard library
for threads \cite{nichols1996pthreads},
and they are now widely used in concurrent programming languages and systems.
They are found among others in:

\begin{itemize}
    \item UNIX\footnote{\url{https://man7.org/linux/man-pages/man3/pthread_cond_init.3p.html}},
    \item Rust\footnote{\url{https://doc.rust-lang.org/std/sync/struct.Condvar.html}}
    \item Python\footnote{\url{https://docs.python.org/3/library/threading.html}}
    \item Go\footnote{\url{https://pkg.go.dev/sync}}
    \item Java\footnote{
              \href{https://docs.oracle.com/en/java/javase/20/docs/api/java.base/java/util/concurrent/locks/Condition.html}
              {Condition Interface}}
\end{itemize}

Despite their widespread use, condition variables can be tricky to use correctly,
and incorrect use can lead to subtle and
hard-to-debug errors such as missed signals or spurious wakeups.
We will look now at these errors in detail.

\section{Compiler arquitecture}

\section{Model checking}

\end{document}