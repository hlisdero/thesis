% !TeX root = ../Thesis.tex
\documentclass[../Thesis.tex]{subfiles}
\graphicspath{{\subfix{../images/}}}

\begin{document}

\section{Deadlocks}

Deadlocks are a common problem that can occur in concurrent systems,
which are systems where multiple threads or processes are
running simultaneously and potentially sharing resources.

A deadlock occurs when two or more threads or processes
are blocked and unable to continue executing
because each is waiting for the other to release a resource that it needs.
This results in a situation where none of the threads or processes
can make progress and the system becomes effectively stuck.

This can be a serious problem in concurrent systems,
as they can cause the system to become unresponsive or even crash.
Therefore, it is important to be able to detect and prevent deadlocks.
They can occur in any concurrent system where multiple threads or processes
are competing for shared resources.
Examples of shared resources that can lead to deadlocks include system memory,
input/output devices, locks, and other types of synchronization primitives.

Deadlocks can be difficult to detect and prevent
because they depend on the precise timing of events in the system.
Even in cases where deadlocks can be detected, resolving them can be difficult,
as it may require releasing resources that have already been acquired or
rolling back completed transactions.
To avoid deadlocks, it is important to carefully manage shared resources in a concurrent system.
This can involve using techniques such as resource allocation algorithms,
deadlock detection algorithms, and other types of synchronization primitives.
By carefully managing shared resources, it is possible to prevent deadlocks from occurring and
ensure the smooth operation of concurrent systems.

To understand the concept in more detail,
consider a simple example where two processes, A and B,
are competing for two resources, X and Y.
Initially, process A has acquired resource X and is waiting to acquire resource Y,
while process B has acquired resource Y and is waiting to acquire resource X.
In this situation, neither process can continue executing
because it is waiting for the other process to release a resource that it needs.
This results in a deadlock, as neither process can make progress.

\end{document}