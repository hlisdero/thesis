\documentclass{beamer}
\usepackage{Presentation}

\title[Compile-time Deadlock Detection in Rust]{Compile-time Deadlock Detection \\ in Rust using Petri Nets}
\author{Horacio Lisdero Scaffino}
\institute[FIUBA]{Facultad de Ingeniería\\Universidad de Buenos Aires}
\date{June 30, 2023}
% TWEAK THE FONT SIZE
\setbeamerfont{date}{size=\scriptsize}
% ADD LOGO
\logo{\includegraphics[height=1cm]{FIUBA-Logo.png}}

\AtBeginSection[]
{
  \begin{frame}{Agenda}
  \footnotesize
    \tableofcontents[currentsection]
  \end{frame}
}

\AtBeginSubsection[]
{
  \begin{frame}{Agenda}
  \footnotesize
    \tableofcontents[currentsection, currentsubsection]
  \end{frame}
}

\begin{document}

\begin{frame}
  \titlepage
\end{frame}

% REMOVE THE LOGO FROM NOW ON %
\logo{}

\begin{frame}{Agenda}
  \tableofcontents
\end{frame}

\section{Introduction}

\section{Rust}

\subsection{What is Rust?}

\begin{frame}{What is Rust?}
  Rust is a multi-paradigm, general-purpose programming language that
  aims to provide developers with a safe and efficient way to write low-level code.

  \pause
  \vfill

  \begin{itemize}
    \item Memory-safe
    \item Compiled to machine code, no runtime needed
    \item High-level simplicity
    \item Low-level performance (on the same level as C or C++)
  \end{itemize}
\end{frame}

\begin{frame}{Brief timeline of Rust}
  \begin{description}
    \item [2007] Started as a side project by Graydon Hoare, a programmer at Mozilla
    \item [2009] Mozilla officially started sponsoring the project
    \item [2015] First stable version 1.0
    \item [2016] Mozilla releases Servo, a browser engine built with Rust
    \item [2019] \texttt{async/await} support stabilized
    \item [2021] The Rust Foundation is founded by AWS, Huawei, Google, Microsoft, and Mozilla
    \item [2021] The Android Open Source Project encourages the use of Rust for the SO components below the ART
    \item [2022] The Linux kernel adds support for Rust alongside C
    \item [2023] 8 years in a row the most loved programming language in the Stack Overflow Developer Survey
  \end{description}
\end{frame}

\begin{frame}{Memory safety}
  It achieves memory safety without using a garbage collector or reference counting.
  Instead, it uses the concept of \textbf{ownership} and \textbf{borrowing}.

  \vfill
  \pause

  It prevents a wide variety of error classes at compile-time:

  \begin{itemize}
    \item Double free
    \item Use after free
    \item Dangling pointers
    \item Data races
    \item Passing non-thread-safe variables
  \end{itemize}

  \vfill

  If a violation of the compiler rules is found, the program will simply not compile.
\end{frame}

\subsection{Why Rust?}

\section{Petri nets}

\subsection{Examples}

\section{Translation}

\subsection{MIR}

\subsection{Modelling threads}

\subsection{Modelling mutexes}

\subsection{Modelling condition variables}

\begin{frame}{Bibliography}
  \tiny
  \bibliographystyle{apalike}
  \bibliography{
    ../Bibliography/Articles.bib,
    ../Bibliography/Blogs.bib,
    ../Bibliography/Books.bib,
    ../Bibliography/Conferences.bib,
    ../Bibliography/Proceedings.bib
  }
\end{frame}

\end{document}
