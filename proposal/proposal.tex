\documentclass[12pt]{article}
\usepackage[utf8]{inputenc}
\usepackage{minted}
\usepackage{multirow}
\usepackage{listings}

\usepackage{amssymb,amsmath,amsthm,amsfonts}
\usepackage{calc}
\usepackage{graphicx}
\usepackage{subfigure}
\usepackage{indentfirst}
\usepackage{titlesec}
\newcommand{\sectionbreak}{\clearpage}
\DeclareGraphicsExtensions{.bmp,.png,.pdf,.jpg}
\usepackage{gensymb}

\usepackage{url}
\usepackage{amsmath}
\usepackage{graphicx}
\graphicspath{{images/}}
\usepackage{parskip}
\usepackage{fancyhdr}
\usepackage{vmargin}
\setmarginsrb{2 cm}{2 cm}{2 cm}{2 cm}{1 cm}{1.5 cm}{1 cm}{1.5 cm}

\usepackage[spanish]{babel}
\usepackage[colorlinks=true, allcolors=blue]{hyperref}
\hypersetup{
    colorlinks=true,% make the links colored
}
\usepackage[nolist]{acronym}
\usepackage[table]{xcolor}
\usepackage{url}


\begin{document}

\begin{titlepage}

    \title{     \textbf{Propuesta de Tesis de Grado \\ de Ingeniería en Informática}\\[2.5ex]
        \textit{Detección de Deadlocks en Rust \\en tiempo de compilación \\mediante Redes de Petri} }

    \author{
        \textbf{Director:} Ing. Pablo A. Deymonnaz\\[2.5ex]
        \textbf{Alumno:} Horacio Lisdero Scaffino, \textit{(Padrón \# 100.132)}                                \\
        \texttt{ hlisdero@fi.uba.ar }                                    \\[2.5ex]
        \normalsize{Facultad de Ingeniería, Universidad de Buenos Aires}        \\
    }

    \date{17 de febrero de 2023}

\end{titlepage}

\maketitle
\thispagestyle{empty}

\maketitle{
    \hypersetup{linkcolor=black}
    \tableofcontents
}

%##########################
% INTRODUCCION
%##########################

\section{Introducción}

En el área de computación concurrente, uno de los desafíos principales es probar la correctitud de un programa concurrente.
A diferencia de un programa secuencial donde para cada entrada se obtiene siempre la misma salida, en un programa concurrente
la salida puede depender de cómo se intercalaron las instrucciones de los diferentes procesos o threads durante la ejecución.

La correctitud de un programa concurrente se define entonces en términos de propiedades del cómputo realizado y no en términos del
resultado obtenido. En la literatura se definen dos tipos de propiedades de correctitud\cite{ben-ari2006}\cite{coulouris2012}\cite{tanenbaum2017}:

\begin{itemize}
    \item Propiedades de \textit{safety}: Propiedades que se deben cumplir \textit{siempre}.
    \item Propiedades de \textit{liveness}: Propiedades que se deben cumplir \textit{eventualmente}.
\end{itemize}

Dos de las propiedades de tipo \textit{safety} deseables en un programa concurrente son:

\begin{itemize}
    \item \textbf{Exclusión mutua}: dos procesos no deben acceder a recursos compartidos al mismo tiempo.
    \item \textbf{Ausencia de \textit{deadlock}}: un sistema en ejecución debe poder continuar realizando su tarea, es decir, avanzar produciendo trabajo útil.
\end{itemize}

Usualmente se utilizan primitivas de sincronización tales como mutexes, semáforos, monitores y \textit{condition variables}
para implementar el acceso coordinado de los procesos o hilos a los recursos compartidos. No obstante, el uso correcto de estas primitivas es difícil de lograr en la práctica.

Los deadlocks son uno de los grandes problemas de los sistemas distribuidos.
Su aparición puede dejar sistemas enteros fuera de uso y causar graves daños a personas y equipos.
Numerosas teorías y métodos se han propuesto para prevenir los deadlocks.


\subsection{Motivación}

\bigskip

%##########################
% ESTADO DEL ARTE
%##########################

\section{Estado del arte / Literatura relacionada}


\bigskip

%##########################
% OBJETIVOS
%##########################

\section{Objetivos}

El objetivo general del trabajo es ...

Los objetivos particulares son:

\begin{enumerate}
    \item 1
    \item 2
    \item 3
          \begin{enumerate}
              \item 3.1
              \item 3.2
          \end{enumerate}
\end{enumerate}

\bigskip

%##########################
% METODOLOGIA
%##########################

\section{Metodologías propuestas}

\bigskip

%##########################
% DATOS
%##########################

\section{Conjuntos de datos (dependiendo del área de la tesis)}

\bigskip

%##########################
% RECURSOS
%##########################

\section{Recursos informáticos (opcional)}

\subsection{Hardware}

\subsection{Software}

\subsubsection{Otras herramientas}

\bigskip

%##########################
% PLANIFICACIÓN
%##########################

\section{Cronograma de trabajo}

Cantidad de horas: [768-1000]

\bigskip

\begin{center}
    \def\arraystretch{1.5}
    \begin{tabular}{ |l|c|c|c|c|c|c|c|c|c|c|c|c| }

        \hline
        \multirow{2}{1em}{Tareas} & \multicolumn{12}{|c|}{Meses}                                                                                                                                                                      \\  \cline{2-13} &
        1                         & 2                            & 3                & 4                & 5                & 6                & 7                & 8 & 9 & 10 & 11               & 12                                  \\  \hline
        Tarea A                   & \cellcolor{gray}             & \cellcolor{gray} &                  &                  &                  &                  &   &   &    &                  &                  &                  \\
        \hline
        Tarea B                   &                              & \cellcolor{gray} & \cellcolor{gray} &                  &                  &                  &   &   &    &                  &                  &                  \\
        \hline
        Tarea C                   &                              & \cellcolor{gray} & \cellcolor{gray} & \cellcolor{gray} & \cellcolor{gray} & \cellcolor{gray} &   &   &    &                  &                  &                  \\
        \hline
        Tarea D                   &                              &                  &                  &                  &                  &                  &   &   &    &                  &                  &                  \\
        \hline
        Tarea E                   &                              &                  &                  &                  &                  &                  &   &   &    & \cellcolor{gray} & \cellcolor{gray} & \cellcolor{gray} \\
        \hline
    \end{tabular}
\end{center}

\bigskip

\begin{itemize}
    \item Tarea A [100 horas]: (descripción)
    \item Tarea B [180 horas]:
    \item ...
\end{itemize}

\bigskip

%\clearpage

\bibliographystyle{acm}
\bibliography{bibliography.bib}

\end{document}