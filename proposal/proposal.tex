\documentclass[12pt]{article}
\usepackage[utf8]{inputenc}
\usepackage{minted}
\usepackage{multirow}
\usepackage{listings}

\usepackage{amssymb,amsmath,amsthm,amsfonts}
\usepackage{calc}
\usepackage{graphicx}
\usepackage{subfigure}
\usepackage{indentfirst}
\usepackage{titlesec}
\newcommand{\sectionbreak}{\clearpage}
\DeclareGraphicsExtensions{.bmp,.png,.pdf,.jpg}
\usepackage{gensymb}

\usepackage{url}
\usepackage{amsmath}
\usepackage{graphicx}
\graphicspath{{images/}}
\usepackage{parskip}
\usepackage{fancyhdr}
\usepackage{vmargin}
\setmarginsrb{2 cm}{2 cm}{2 cm}{2 cm}{1 cm}{1.5 cm}{1 cm}{1.5 cm}


\usepackage[spanish]{babel}
\usepackage[colorlinks=true, allcolors=blue]{hyperref}
\hypersetup{
    colorlinks=true,% make the links colored
}
\usepackage[nolist]{acronym}
\usepackage[table]{xcolor}
\usepackage{url}

%% Biblatex
%\usepackage[style=plain]{biblatex}
%\addbibresource{bibliography.bib}

\begin{document}

\begin{titlepage}

    \title{     \textbf{Propuesta de Tesis de Grado \\ de Ingeniería en Informática}\\[2.5ex]
        \textit{Título de Tesis \\(Segundo renglón)} }

    \author{
        \textbf{Director:} Gral. José de San Martín \\[2.5ex]
        \textbf{Co-director:} Gral. Manuel Belgrano \\[2.5ex]
        \textbf{Alumno:} Luis A. Huergo, \textit{(Padrón \# 000.001)}                                \\
        \texttt{ luis.huergo@fi.uba.ar }                                    \\[2.5ex]
        \normalsize{Facultad de Ingeniería, Universidad de Buenos Aires}        \\
    }
    \date{}

\end{titlepage}

\maketitle
\thispagestyle{empty}

\maketitle

{
    \hypersetup{linkcolor=black}
    \tableofcontents
}

%##########################
% INTRODUCCION
%##########################

\section{Introducción}

\subsection{Motivación}

\bigskip

%##########################
% ESTADO DEL ARTE
%##########################

\section{Estado del arte / Literatura relacionada}

Inicialmente este problema fue estudiado por~\cite{knuth2014art}.

\bigskip

%##########################
% OBJETIVOS
%##########################

\section{Objetivos}

El objetivo general del trabajo es ...

Los objetivos particulares son:

\begin{enumerate}
    \item 1
    \item 2
    \item 3
          \begin{enumerate}
              \item 3.1
              \item 3.2
          \end{enumerate}
\end{enumerate}

\bigskip

%##########################
% METODOLOGIA
%##########################

\section{Metodologías propuestas}

\bigskip

%##########################
% DATOS
%##########################

\section{Conjuntos de datos (dependiendo del área de la tesis)}

\bigskip

%##########################
% RECURSOS
%##########################

\section{Recursos informáticos (opcional)}

\subsection{Hardware}

\subsection{Software}

\subsubsection{Otras herramientas}

\bigskip

%##########################
% PLANIFICACIÓN
%##########################

\section{Cronograma de trabajo}

Cantidad de horas: [768-1000]

\bigskip

\begin{center}
    \def\arraystretch{1.5}
    \begin{tabular}{ |l|c|c|c|c|c|c|c|c|c|c|c|c| }

        \hline
        \multirow{2}{1em}{Tareas} & \multicolumn{12}{|c|}{Meses}                                                                                                                                                                      \\  \cline{2-13} &
        1                         & 2                            & 3                & 4                & 5                & 6                & 7                & 8 & 9 & 10 & 11               & 12                                  \\  \hline
        Tarea A                   & \cellcolor{gray}             & \cellcolor{gray} &                  &                  &                  &                  &   &   &    &                  &                  &                  \\
        \hline
        Tarea B                   &                              & \cellcolor{gray} & \cellcolor{gray} &                  &                  &                  &   &   &    &                  &                  &                  \\
        \hline
        Tarea C                   &                              & \cellcolor{gray} & \cellcolor{gray} & \cellcolor{gray} & \cellcolor{gray} & \cellcolor{gray} &   &   &    &                  &                  &                  \\
        \hline
        Tarea D                   &                              &                  &                  &                  &                  &                  &   &   &    &                  &                  &                  \\
        \hline
        Tarea E                   &                              &                  &                  &                  &                  &                  &   &   &    & \cellcolor{gray} & \cellcolor{gray} & \cellcolor{gray} \\
        \hline
    \end{tabular}
\end{center}

\bigskip

\begin{itemize}
    \item Tarea A [100 horas]: (descripción)
    \item Tarea B [180 horas]:
    \item ...
\end{itemize}

\bigskip

%\clearpage

\bibliographystyle{apalike}
\bibliography{bibliografia.bib}

\end{document}