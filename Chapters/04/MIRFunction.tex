\section{MIR function}

In the following section,
we will delve into the translation process of a \acrshort{MIR} function.
This section aims to provide a comprehensive understanding
of the translation techniques applied to specific \acrshort{MIR} elements,
namely \acrfull{BB}, statements, and terminators.
These components were introduced previously in Sec. \ref{sec:mir-components}.

The implementation in the repository is accordingly named
\Rustinline{MirFunction}\footnote{\url{https://github.com/hlisdero/cargo-check-deadlock/blob/main/src/translator/mir_function.rs}}.
This type stores the start place and the end place of the function.
These must be supplied to the \acrshort{MIR} function because
they also represent where the function call took place and where it should return to.
The end place is in simpler terms the return place in the Petri net.
See Fig. \ref{fig:simplest-function} for an illustration.

The start place of the function overlaps with the place
that models the first basic block in the function.
This matches more closely the \acrshort{MIR} as the code only lives inside of basic blocks,
so the function call begins at the first basic block (BB0).

The \Rustinline{MirFunction} also stores the ID that identifies it.
This is necessary for performing function calls from this function.
Moreover, the function requires a name that is different for every function call,
so it receives a name with an index appended, making it unique across the whole Petri net.

We will now explain how each component is expressed in the language of Petri nets.
Through a detailed exploration of the translation techniques employed for
basic blocks, statements, and terminators, we will develop a formal model that
accurately captures the behavior of a \acrshort{MIR} function for deadlock detection.

\subsection{Basic blocks}

One important aspect of the translation process involves transforming basic blocks into Petri nets,
which serve as a fundamental building block for modeling the control flow within the \acrshort{MIR} function.
As seen in Fig. \ref{fig:mir-cfg-example}, a basic block in \acrshort{MIR} acts as a container
that houses a sequence of zero or more statements, as well as a mandatory terminator statement.

As nodes in a graph, the main property of basic blocks is
their ability to direct the flow of control within a program.
Each basic block may have one or more basic blocks pointing to it,
indicating the potential paths from which the control flow can reach it.
Similarly, a basic block can point to one or more other basic blocks,
signifying the possible paths the control flow may take after executing the current basic block.
It is worth mentioning that isolated basic blocks with no connections
do not make sense since they would never be executed, i.e., they are dead code.

This branching behavior allows for dynamic control flow within the program,
as multiple basic blocks can continue the control flow to the same target basic block
(for instance to a block that performs cleanup tasks).
Conversely, a basic block can branch and
determine the next basic block based on specific conditions or program logic, e.g.,
in an \Rustinline{if}, \Rustinline{while}, \Rustinline{match} or other control structures.
This versatility in control flow provides the foundation for modeling complex program behavior.

The Petri net model used in the implementation relies on a single place to model each \acrshort{BB}.
We can abstract away the inner workings of the \acrshort{BB} and work with a single place.
The rationale behind it is that the connections to other \acrshort{BB} depend solely on the terminators
and statements are not modeled at all as we will see shortly.
Additionally, the implementation\footnote{\url{https://github.com/hlisdero/cargo-check-deadlock/blob/main/src/translator/mir_function/basic_block.rs}}
keeps track of the function name to which the \acrshort{BB} belongs and
the \acrshort{BB} number to generate unique labels.

\subsection{Statements}

\acrshort{MIR} statements are intentionally \emph{not} incorporated into the Petri net model.
Considering the reasons for this and the benefits may not be immediately apparent,
we will provide a detailed explanation for this implementation decision.

The approach that was previously implemented did include the modeling of statements.
It was based on the approach seen in \cite{meyer2020}.
However, it was observed that this led to the creation of a long chain of places and transitions
that did not significantly contribute to the detection of deadlocks or missed signals.
Furthermore, it unnecessarily inflated the size of the Petri net representation,
making it more difficult to debug and understand.
Consequently, this approach was later revised and removed in a later
commit\footnote{\url{https://github.com/hlisdero/cargo-check-deadlock/commit/b27403b6a5b2bb020a5d7ab2a9b1cacefb48be82}}.

In all the programs we had tested so far, the statements did not perform any action
that may justify their addition to the Petri net.
On the contrary, the nets that include the statements were larger and more difficult to read.
In order to facilitate the use and adoption of the tool,
it is crucial to optimize the Petri net for the purpose of the tool.
The decision was therefore to eliminate all the code related to the modeling of the statements
and fix the tests accordingly to match the new output.

The alternative was to disable the statements with a compile flag
but that would complicate testing and
since there is no use case for modeling the \acrshort{MIR} statements anyway, this option was discarded.

For illustrative purposes, we can refer to Fig. \ref{fig:statement-model-comparison},
which showcases a comparison between the old model and the new model.
The differences between these two representations are evident,
highlighting the removal of statements from the model and
the subsequent simplification achieved in the resulting Petri net.

\begin{figure}[!htb]
  \centering
  \includesvg[width=0.5\linewidth]{statement-model-comparison.svg}
  \caption{A side-by-side comparison of two possibilities to model the MIR statements.}
  \label{fig:statement-model-comparison}
\end{figure}

\subsection{Terminators}
\label{sec:terminators}