\section{Mutex (\texttt{std::sync::Mutex})}

A mutex, short for mutual exclusion, is a synchronization mechanism
used to control access to a shared resource in a concurrent program.
It allows multiple threads to access the shared resource in a mutually exclusive manner,
ensuring that only one thread can access the resource at a time.

In this section, the \acrshort{PN} model for a mutex in Rust is explained,
then a practical example is presented to ease comprehension and finally
the algorithms used for the translation of mutex functions are outlined.

\subsection{Petri net model}

In Rust, a mutex is created by wrapping the shared data in a \Rustinline{Mutex<T>} type,
where \Rustinline{T} is the type of the shared resource.
The \Rustinline{std::sync::Mutex} type exposes a method named \Rustinline{lock}
to acquire the lock on the shared resource.
If the mutex is currently unlocked,
the thread successfully acquires the lock and can proceed with accessing the resource.
If the mutex is already locked by another thread,
the thread attempting to acquire the lock will be blocked until the lock becomes available.
The \Rustinline{lock} method returns a mutex guard (\Rustinline{std::sync::MutexGuard})
that grants exclusive access to the resource until it is dropped.

Contrary to the \Rustinline{unlock} semantics present in C or C++,
the mutex included by the Rust standard library is unlocked implicitly,
i.e., without calling a function.
The mutex implements \acrfull{RAII} and releases the lock automatically
when it goes out of scope, preventing deadlocks.
Alternatively, dropping a local variable of type \Rustinline{std::sync::MutexGuard}
is equivalent to unlocking the corresponding mutex.

A mutex can be modeled in \acrshort{PN} as a single place
that represents the state of the mutex,
indicating whether it is locked or unlocked.
The place is labeled to reflect its purpose as a mutex.
Besides, the place is marked with a token initially
to signify that the mutex starts in the unlocked state.

Transitions that lock the mutex consume the token from the mutex place.
If the token is absent, the transition may not fire.
The mutex must be in the unlocked state to enable the locking transition,
which is the desired behavior.

Transitions that unlock the mutex produce a token at the mutex place.
The transition can fire as long as the program reached that point in the execution.
After the transition fires, the mutex place holds again a token
that can be consumed by a locking transition.
Two types of transitions may unlock the mutex:

\begin{enumerate}
  \item A \Rustinline{Drop} terminator (Sec. \ref{sec:terminators})
        when the dropped place is of type \Rustinline{std::sync::MutexGuard}.
  \item The transition for a call to \Rustinline{std::mem::Drop},
        which frees the memory occupied by the value passed in explicitly.
\end{enumerate}

By connecting the mutex place
to the locking and unlocking transitions using input and output arcs,
we establish the relationship between the mutex state and the actions that manipulate it.
This modeling approach allows for the representation of the mutex's behavior in a \acrshort{PN}
and facilitates the analysis of its interactions with other parts of the system.

The \acrshort{PN} model presented here is well-known in the literature
and has been applied successfully in other tools.
It can be found among others in \cite{kavi2002modeling,moshtaghi2001,meyer2020,zhang2022deadlocks}.

\subsection{A practical example}

Consider the \acrshort{PN} model shown in Fig. \ref{fig:mutex-example}
corresponding to the program in Listing \ref{lst:double-lock-deadlock}.
The \acrshort{MIR} is depicted in \ref{lst:double-lock-deadlock-mir}.
This test program is one of the examples included in the repository.

Observe that there are two locking transitions
mapping the two calls to \Rustinline{lock} in the source code.
The indexing system mirrors the order of their appearance in the program,
which justifies the labels \Rustinline{std_sync_Mutex_T_lock_0_CALL}
and \Rustinline{std_sync_Mutex_T_lock_1_CALL}.
Both have an incoming arc from the mutex place \Rustinline{MUTEX_0}.

As explained before, \Rustinline{Drop} terminators may unlock a mutex.
No matter if they fail or not (the error case includes the suffix \Rustinline{_UNWIND}),
an outgoing arc flows back to the mutex place to replenish the token.

One should take note that there are more incoming arcs to the mutex place than outgoing arcs,
which highlights the importance of following the mutex guards throughout the \acrshort{MIR}
using the strategy explained in Sec. \ref{sec:intercepting-assignments}.

\begin{figure}[!htbp]
  \centering
  \includesvg[width=0.95\linewidth]{mutex-example.svg}
  \caption{The Petri net model for the program in Listing \ref{lst:double-lock-deadlock}.}
  \label{fig:mutex-example}
\end{figure}

\subsection{Algorithms for mutex translation}
\label{sec:mutex-algorithms}