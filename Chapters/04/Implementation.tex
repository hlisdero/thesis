This chapter is dedicated to exploring
the implementation details of the deadlock detection tool.
Its purpose is to provide a high-level view of the code and the data structures.
The most important implementation decisions made throughout the development process
are examined as well.

In the subsequent sections,
we will describe the central components of the deadlock detection tool,
including the internal representation of the call stack, the function memory model,
and the translation of every constituent of a \acrshort{MIR} function.

Later on, a significant portion of the discussion is devoted to explaining
the support of multithreading and the modeling of synchronization primitives as Petri nets.
Its implementation required careful design considerations
to ensure correctness and efficiency.

The tool currently supports the following structures
from the Rust standard library to synchronize access to shared resources
and provide communication among threads:

\begin{itemize}
  \item mutexes (\texttt{std::sync::Mutex}\footnote{\url{https://doc.rust-lang.org/std/sync/struct.Mutex.html}}),
  \item condition variables (\texttt{std::sync::Condvar}\footnote{\url{https://doc.rust-lang.org/std/sync/struct.Condvar.html}}),
  \item atomic reference counters (\texttt{std::sync::Arc}\footnote{\url{https://doc.rust-lang.org/std/sync/struct.Arc.html}}).
\end{itemize}

While the main details are covered, this chapter is not intended
to serve as a substitute for the code documentation.
The code documentation in the form of comments, unit tests, and integration tests
provides comprehensive information on the low-level specifics and usage of the tool.
As stated before, the repository is publicly available on
GitHub\footnote{\url{https://github.com/hlisdero/granite2}}\footnote{\url{https://github.com/hlisdero/netcrab}}.