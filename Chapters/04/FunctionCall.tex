\section{Function calls}
\label{sec:function-calls}

\subsection{The call stack}

A Rust program is composed, as in other programming languages, of functions.
The program begins (except for the caveats seen in Sec. \ref{sec:basic-places})
with a call to the \Rustinline{main} function, which then may call other functions.
It should be emphasized that function calls may be placed at any point within the code.
A function can be called from another function or even from within itself, resulting in recursive calls.

Function calls are stored in memory in a data structure called the \emph{call stack}.
When a function is called in Rust, it gets pushed onto the call stack, creating a new stack frame.
A stack frame contains important information such as the function's local variables, arguments,
and the return address indicating where the program should resume once the function finishes its execution.

The call stack operates based on the principle of \acrfull{LIFO}.
As functions are called, each new stack frame is placed on top of the previous one.
This allows the program to execute the most recently called function first.
Once a function completes its execution, it is popped off the stack,
and the program continues from the point where it left off in the previous function.

Hence, the call stack plays an essential role in managing function calls and returns,
since it keeps track of the flow of function calls and maintains the necessary information
for the program to return to the correct execution point after a function completes its task.

For the same reasons, mirroring the call stack in the translator
is the most suitable approach for tracking function calls to be translated
because it aligns with the logical flow of program execution.
As functions are translated, they are pushed and popped
from the call stack of the \Rustinline{Translator},
reflecting the order in which they are called at runtime.
This enables us to handle nested function invocations and
follow the control flow from one function to the other during the translation process.

\subsection{MIR functions}

In the implementation, the \Rustinline{Translator} has a stack
that supports the usual operations \Rustinline{push}, \Rustinline{pop}, and \Rustinline{peek}.
This stack stores structures of type \Rustinline{MirFunction}\footnote{\url{https://github.com/hlisdero/cargo-check-deadlock/blob/main/src/translator/mir_function.rs}}.
Later, we will see that not all functions are translated as \acrshort{MIR} functions
since not all functions have a representation in \acrshort{MIR}
and, in other cases, it is convenient to handle them differently.
Nevertheless, \acrshort{MIR} functions are the ``common case'' in the translation process,
the default case for the majority of user-defined functions.

The available interface provided by \emph{rustc}
allows for querying the \acrshort{MIR} body of only one function at a time,
which can be done using the \Rustinline{optimized_mir}\footnote{\url{https://doc.rust-lang.org/stable/nightly-rustc/rustc_middle/ty/context/struct.TyCtxt.html\#method.optimized\_mir}}
method.
This implies that it is not possible to get the \acrshort{MIR} of the whole program initially
and the translator must obtain the \acrshort{MIR} from each function as it reaches them in the code.
But how to identify each function?
It is known from experience that functions in distinct modules may have the same name,
making the name unsuitable as an identifier.
Luckily, this problem is already solved in the compiler.
The functions are uniquely identified by the compiler type \Rustinline{rustc_hir::def_id::DefId}\footnote{\url{https://doc.rust-lang.org/stable/nightly-rustc/rustc_hir/def_id/struct.DefId.html}}.
This ID is valid for the crate currently being compiled and it is already present in the \acrshort{HIR}.
The high-level algorithm can be described as follows.

When the translation starts:

\begin{enumerate}
    \item Query the id of the entry point of the program (the \Rustinline{main} function).
    \item Create a \Rustinline{MirFunction} with the necessary information.
    \item Push it to the stack.
    \item If necessary, modify the \acrshort{MIR} function contents using \Rustinline{peek}.
    \item Translate the top of the call stack.
    \item When \Rustinline{main} finishes, remove it (\Rustinline{pop}) from the call stack.
\end{enumerate}

When a terminator of type ``call'' (see Sec. \ref{sec:mir-components}) is encountered:

\begin{enumerate}
    \item Query the id of the called function.
    \item Create a \Rustinline{MirFunction} with the necessary information.
    \item Push it to the stack.
    \item If necessary, modify the \acrshort{MIR} function contents using \Rustinline{peek}.
    \item Translate the top of the call stack.
    \item When the function finishes, remove it (\Rustinline{pop}) from the call stack.
\end{enumerate}

As seen, the approach is consistent for every \acrshort{MIR} function
and thus is easier to implement.

The use of a call stack in the translation process enables context switching
between \acrshort{MIR} functions and facilitates the ability to return to the specific basic block
from which a function was called.
This allows for the translation of the program to be performed function by function,
in a linear fashion, ensuring that the structure and order of the original program are maintained.

However, employing the call stack approach does come with certain limitations.
First, if the same function is called multiple times within the program,
it will be translated multiple times as well.
This is related to the inlining strategy elaborated in Sec. \ref{sec:function-inlining}.
Although this can potentially be mitigated through the use of some sort of cache,
it is out of the scope of this thesis.
This optimization will be discussed in Sec. \ref{sec:future-work-function-cache}.

The more severe implication of using the call stack approach
is the inability to handle recursive functions.
When encountering a recursive function within the translation process,
the process becomes trapped in an endless loop
where the stack grows indefinitely as new stack frames are pushed to it,
leading to a stack overflow and a subsequent crash of the translation process.
This problem is addressed in Sec. \ref{sec:future-work-recursion} too.
For now, it is necessary to accept the limitation that recursive functions
cannot be translated using this framework.

\subsection{Foreign functions and functions in the standard library}

In Rust, the compiler includes by default
the standard library in all compiled binaries, effectively linking it statically.
To override this behavior, the crate-level attribute \Rustinline{#![no_std]}
is used to indicate that the crate will link to the core-crate instead of the std-crate.
See \cite{embedded-book} for more details.

This means that the standard library's functionality becomes an integral part of the resulting executable.
Function calls to the standard library appear in various contexts in Rust code,
such as when accessing command line arguments, invoking iterators,
utilizing traits like \Rustinline{std::clone::Clone}, \Rustinline{std::deref::Deref::deref},
or employing standard library types like \Rustinline{std::result::Result} or \Rustinline{std::option::Option}.
Given the prevalence of these function calls throughout Rust programs,
it becomes essential to handle them separately in the translation process.
It is evident that these standard library functions, due to their purpose, cannot lead to a deadlock.
Therefore, it is more practical to treat them as black boxes within the translation process,
bypassing the need to translate their \acrshort{MIR}.
This approach is indispensable in order to avoid generating
an excessively large and convoluted Petri net that would hinder readability and comprehension.

The focus of the translation effort lies primarily on the user code, specifically
the functions that developers write to implement their desired functionalities.
By directing attention to the user code and excluding the translation of standard library functions,
the resulting Petri net remains more manageable,
facilitating the analysis and verification of potential deadlocks within the user's codebase.
The calls to the standard library constitute, in other words,
the ``frontier'' or ``boundary'' of the translation,
the point at which we stop translating the \acrshort{MIR} accurately
and rely instead on a simplified model.

\subsubsection{Petri net model for a function with cleanup block}

The model presented in Fig. \ref{fig:simplest-function} is the first approximation.
There is, however, an implementation detail that requires careful attention.
Numerous functions in the standard library contain not only an end place
(``target block'', in the \emph{rustc} parlance) but also a cleanup place (``cleanup block'').
This second execution path is taken when the function explicitly panics
or more generally fails to achieve its goal for whatever reason.
In this case, the control flow continues to a different basic block,
where variables are freed and eventually the program ends with a panic error code.
Stated differently, the unwind of the stack begins
as soon as a function encounters a non-recoverable failure.

Considering that the translator cannot tell if this abnormal situation could lead
to a deadlock later in the translation process, it is imperative to translate
this alternative execution path whenever possible.
Only in counted exceptions, all related to the synchronization primitives and discussed in Sec.
\ref{sec:sync-primitives}, this cleanup block is ignored explicitly.
The complete model for an abridged function call with a cleanup block
can be seen in Fig. \ref{fig:function-with-cleanup}.

\begin{figure}[!htb]
    \centering
    \includesvg[width=\linewidth]{function-with-cleanup.svg}
    \caption{The Petri net model for a function with a cleanup block.}
    \label{fig:function-with-cleanup}
\end{figure}

\subsubsection{Functions translated with the abridged Petri net model}

Having discussed the exclusion of standard library functions from the translation process,
we now shift our focus towards the functions that indeed require translation using the model we presented earlier.
Surprisingly, they include a considerable number of functions.

\begin{itemize}
    \item Functions part of the standard library (the std-crate\footnote{\url{https://doc.rust-lang.org/std/}}),
          save for some special cases detailed in Sec. \ref{sec:sync-primitives}.
    \item Functions part of the core library (the core-crate\footnote{\url{https://doc.rust-lang.org/core/}}).
    \item Functions in the alloc-crate: the core allocation and collections library\footnote{\url{https://doc.rust-lang.org/alloc/}}.
    \item Functions without a \acrshort{MIR} representation
          This can be checked with the \Rustinline{is_mir_available}
          method\footnote{\url{https://doc.rust-lang.org/stable/nightly-rustc/rustc_middle/ty/context/struct.TyCtxt.html\#method.is\_mir\_available}}.
    \item Functions which are a foreign item i.e., linked via \Rustinline{extern { ... }}.
          This can be checked with the \Rustinline{is_foreign_item}
          method\footnote{\url{https://doc.rust-lang.org/stable/nightly-rustc/rustc_middle/ty/context/struct.TyCtxt.html\#method.is\_foreign\_item}}.
\end{itemize}

In the future, calls to functions in dependencies, i.e., in other crates, should also be handled in this fashion.
In conclusion, the default case for functions that are \emph{not} user-defined
is to treat them as a foreign function and use an abridged Petri net model to translate them.

\subsection{Diverging functions}

Diverging functions are a special case that is relatively easy to support.
It is simply a function that never returns to the caller.
Examples of this are a wrapper around an infinite \Rustinline{while} loop,
a function that exits the process,
or a function that starts an \acrshort{OS}.
It suffices to connect the start place of the function to
a sink transition (Definition \ref{definition:sink-transition})
as seen in Fig. \ref{fig:diverging-function}.

\begin{figure}[!htb]
    \centering
    \includesvg[width=0.5\linewidth]{diverging-function.svg}
    \caption{The Petri net model for a diverging function (a function that does not return).}
    \label{fig:diverging-function}
\end{figure}

Note that this special case does not constitute a deadlock and must \emph{not} be treated as such.
An infinite loop, i.e., a ``busy wait'', is in its inherent nature distinct from the infinite wait that
characterizes a deadlock as seen in Sec. \ref{sec:coffman-conditions}.
In other words, detecting infinite loops is closer to the problem of detecting livelocks,
which are out of the scope of this thesis.
Besides, the translator cannot tell ahead of time
if the diverging call is benign like a call to \Rustinline{std::process::exit}
or a call to some kind of function carefully designed to block the program.

In the current \acrshort{PN} model, the token is consumed and the net is left
in an end state without tokens in the places
\Rustinline{PROGRAM_END} or \Rustinline{PROGRAM_PANIC} shown in Fig. \ref{fig:program-places}.
Consequently, the model checker is able to distinguish this end state from the other cases
and conclude that a diverging function has been called.

\subsection{Explicit calls to panic}

The \Rustinline{panic!()} macro can be seen as a special case of a divergent function
where the transition representing the function call is connected to
the place labeled \Rustinline{PROGRAM_PANIC} described in Sec. \ref{sec:basic-places}.
The translator detects an explicit call to panic,
which is one of the following functions:

\begin{itemize}
    \item \Rustinline{core::panicking::assert_failed}
    \item \Rustinline{core::panicking::panic}
    \item \Rustinline{core::panicking::panic_fmt}
    \item \Rustinline{std::rt::begin_panic}
    \item \Rustinline{std::rt::begin_panic_fmt}
\end{itemize}

The documentation\footnote{\url{https://rustc-dev-guide.rust-lang.org/panic-implementation.html}}
elaborates on why panic is defined in the core-crate and the std-crate and how it is implemented.

See Listing \ref{lst:panic} for a simple program that panics. The corresponding Petri net model is depicted in Fig. \ref{fig:panic}.
This is one of the illustrative examples included in the repository.

\begin{listing}[!htb]
    \begin{minted}{Rust}
        fn main() {
            panic!();
        }        
    \end{minted}
    \caption{A simple Rust program that calls \Rustinline{panic!()}.}
    \label{lst:panic}
\end{listing}

\begin{figure}[!htb]
    \centering
    \includesvg[width=0.5\linewidth]{panic.svg}
    \caption{The Petri net model for Listing \ref{lst:panic}.}
    \label{fig:panic}
\end{figure}