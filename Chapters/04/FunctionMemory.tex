\section{Function memory}

We will now proceed to explore the memory characteristics
of the \acrshort{MIR} function in detail.
It is important to acknowledge that
the need to record values being assigned between memory locations in the \acrshort{MIR} arises
from the requirements of the deadlock and missed signals detection.
In simpler teams, we are forced to model the memory only
because the supported synchronization variables need to be tracked
during the translation process.

The translator must track variables of the following types:

\begin{itemize}
  \item Mutexes (\Rustinline{std::sync::Mutex}).
  \item Mutex guards (\Rustinline{std::sync::MutexGuard}).
  \item Join handles (\Rustinline{std::thread::JoinHandle}).
  \item Condition variables (\Rustinline{std::sync::Condvar}).
  \item Aggregates, i.e., wrappers such as \Rustinline{std::sync::Arc} or
        types that contain multiple values like tuples or a structured type (\Rustinline{struct}).
\end{itemize}

Before calling methods on these types of synchronization variables,
immutable or mutable references to the original memory location are created.
The translator must somehow know
which specific synchronization variable is behind a given reference.
Knowing the type of the memory location is \emph{not} enough,
the \emph{value} must be readily available to the translator
to operate on the Petri net model of the specific synchronization variable.

\subsection{A guided example to introduce the challenges}

To illustrate the situation described previously, consider the Rust program
shown in Listing \ref{lst:double-lock-deadlock}.
It is again one of the example programs found in the repository.
As it should be evident to the reader, this program deadlocks when executed.
The reason is that the \Rustinline{std::sync::Mutex::lock} method is
being called twice on the same mutex.
To detect this deadlock, the translator must be able to at the very least
identify that the invocation of \Rustinline{lock} takes place on the same mutex.

\begin{listing}[!htb]
  \begin{minted}{Rust}
    fn main() {
      let data = std::sync::Mutex::new(0);
      let _d1 = data.lock();
      let _d2 = data.lock(); // cannot lock, since d1 is still active
    }
  \end{minted}
  \caption{A deadlock caused by calling \Rustinline{lock} twice on the same mutex.}
  \label{lst:double-lock-deadlock}
\end{listing}

Observe now an excerpt of the \acrshort{MIR}
of the same erroneous program in Listing \ref{lst:double-lock-deadlock-mir}.
The comments have been removed for clarity.
In BB0 the mutex is created through a call to \Rustinline{std::sync::Mutex::new}.
The new mutex is the return value of the function.
It is assigned to the local variable \Rustinline{_1}.
Then the execution continues in BB1.
Focus on the first statement of BB1:
An immutable reference to the local variable \Rustinline{_1} is stored in \Rustinline{_3}.
Next, the reference is moved to the function \Rustinline{std::sync::Mutex::lock}.
This reference is consumed by lock, that is to say,
the local variable \Rustinline{_3} is not used anywhere else in the \acrshort{MIR}
because, from that point on, the ownership of the reference is transferred
to the \Rustinline{std::sync::Mutex::lock} function.

\begin{listing}[!htb]
  \begin{minted}{Rust}
    fn main() -> () {
      let mut _0: ();
      let _1: std::sync::Mutex<i32>;
      let mut _3: &std::sync::Mutex<i32>;
      let mut _5: &std::sync::Mutex<i32>;
      scope 1 {
          debug data => _1;
          let _2: std::result::Result<std::sync::MutexGuard<'_, i32>, std::sync::PoisonError<std::sync::MutexGuard<'_, i32>>>;
          scope 2 {
              debug _d1 => _2;
              let _4: std::result::Result<std::sync::MutexGuard<'_, i32>, std::sync::PoisonError<std::sync::MutexGuard<'_, i32>>>; 
              scope 3 {
                  debug _d2 => _4;
              }
          }
      }
  
      bb0: {
          _1 = Mutex::<i32>::new(const 0_i32) -> bb1;
      }
  
      bb1: {
          _3 = &_1;
          _2 = Mutex::<i32>::lock(move _3) -> bb2;
      }
  
      bb2: {
          _5 = &_1;
          _4 = Mutex::<i32>::lock(move _5) -> [return: bb3, unwind: bb6];
      }
  \end{minted}
  \caption{An except of the MIR of the program from Listing \ref{lst:double-lock-deadlock}.}
  \label{lst:double-lock-deadlock-mir}
\end{listing}

Immediately after the statement, the translator encounters the terminator of BB1.
It contains a call to \Rustinline{std::sync::Mutex::lock}.
How would the translator know, when translating this call,
that \Rustinline{_3} is indeed the mutex stored in \Rustinline{_1}?
This is the problem that the modeling of the function's memory aims to solve.

The problem goes even further.
The local variable \Rustinline{_2} contains a mutex guard after the call to lock,
which should be recorded too.
Notice how BB2 repeats the same operations as BB1 but uses different local variables,
\Rustinline{_5} and \Rustinline{_4}.
The translator should know that \Rustinline{_5} is an alias for \Rustinline{_1} as well.
Furthermore, the mutex guards in \Rustinline{_2} and \Rustinline{_4} will eventually be dropped,
which indirectly unlocks the mutex.
There has to be a link from the mutex guard in \Rustinline{_2} and \Rustinline{_4}
to the mutex in \Rustinline{_1}.
More concisely, the translator should monitor which mutex is behind each mutex guard.

To make matters more complex, each \acrshort{MIR} function has its own stack memory,
with its separate local variables \Rustinline{_0}, \Rustinline{_1},
\Rustinline{_2}, \Rustinline{_3}, and so on.
Thus, the mapping of memory locations to synchronization variables
cannot be a single global structure.
It is instead dependent on the context of the current function being translated.
Lastly, a synchronization variable could migrate from one function to another
and the translator must be able to re-map them correctly.

This suffices as a brief practical example of the challenges of memory modeling.
We can now introduce the solution that has been implemented.

\subsection{A mapping of \texttt{rustc\_middle::mir::Place} to shared counted references}

The implementation is suitably named
\Rustinline{Memory}\footnote{\url{https://github.com/hlisdero/cargo-check-deadlock/blob/main/src/translator/mir_function/memory.rs}}.
As anticipated in the previous section,
there is one instance of \Rustinline{Memory} per \Rustinline{MirFunction}.
The memory is tightly connected to the context of the \acrshort{MIR} function.

Rather than moving values between different memory locations,
as observed in the \acrshort{MIR},
our solution relies on the simpler concept of ``linking''.
This entails associating a specific \Rustinline{rustc_middle::mir::Place}
with the corresponding value.
This association is not removed when moving the variable to a different function.
It also does not differentiate a shallow copy of the value
from taking a reference or a mutable reference.
To put it shortly, it is an all-encompassing mapping between places and values.

To accommodate the possibility of linking the same value to multiple places,
particularly when multiple memory locations hold an immutable reference to the value,
it becomes necessary for the stored value to be a reference to the synchronization variable.
To clarify, this introduces a second level of indirection.
In order to facilitate the required cloning operations,
we have opted to utilize \Rustinline{std::rc::Rc},
which is a smart pointer provided by the Rust standard library.
The ownership of the referenced value (the synchronization variable) is shared
and every time that the value is cloned, an internal counter is incremented.
When the count reaches zero, the value is freed \cite[Chap. 15.4]{rust-book}.

The \Rustinline{Memory} utilizes a \Rustinline{std::collections::HashMap} data structure
that establishes a mapping between \Rustinline{rustc_middle::mir::Place} instances and
an enum with 5 variants corresponding to
the 5 types mentioned previously that the translator tracks.
4 of these 5 variants enclose a \Rustinline{std::rc::Rc}
reference to the synchronization variable.
The aggregate case instead contains a vector of \Rustinline{Value}.
This enables nesting aggregate values inside of each other,
which is a critical requirement for supporting more complex programs with nested \Rustinline{structs}.

\begin{listing}[!htb]
  \begin{minted}{Rust}
    #[derive(Default)]
    pub struct Memory<'tcx> {
        map: HashMap<Place<'tcx>, Value>,
    }

    /// ...

    /// Possible values that can be stored in the `Memory`.
    /// A place will be mapped to one of these.
    #[derive(PartialEq, Clone)]
    pub enum Value {
        Mutex(MutexRef),
        MutexGuard(MutexGuardRef),
        JoinHandle(ThreadRef),
        Condvar(CondvarRef),
        Aggregate(Vec<Value>),
    }

    /// ...
    
    /// A mutex reference is just a shared pointer to the mutex.
    pub type MutexRef = std::rc::Rc<Mutex>;

    /// A mutex guard reference is just a shared pointer to the mutex guard.
    pub type MutexGuardRef = std::rc::Rc<MutexGuard>;

    /// A condvar reference is just a shared pointer to the condition variable.
    pub type CondvarRef = std::rc::Rc<Condvar>;

    /// A thread reference is just a shared pointer to the thread.
    pub type ThreadRef = std::rc::Rc<Thread>;
  \end{minted}
  \caption{A summary of the type definitions of the \Rustinline{Memory} implementation.}
  \label{lst:memory-implementation}
\end{listing}

Using a hash map allows for efficient retrieval
and management of the associated values during the translation process.
The \Rustinline{Memory} also takes care of providing typedefs for the different
references to synchronization variables.
Listing \ref{lst:memory-implementation} depicts an excerpt of the source file
with the essential type definitions used in the implementation.
Improvements to the current implementation are discussed in Sec. \ref{sec:future-work-memory-model}.

\subsection{Intercepting assigments}
\label{sec:intercepting-assignments}

The missing piece in the puzzle of the memory model is
where to link the memory locations exactly.
There are three separate places in the code in which this takes place.

On the one hand, the translator functions responsible for processing the methods
of mutexes, condition variables, and threads create new synchronization variables
that are linked to the return value of the corresponding method.
This is where the lifetime of each synchronization variable starts.
The specifics are expanded upon
in Sec. \ref{sec:mutex-algorithms} and \ref{sec:condvar-algorithms}.

On the other hand, the synchronization variable may be assigned in any other \acrshort{BB}.
For this reason, the translator incorporates a custom implementation of the method
\Rustinline{visit_assign} to intercept every assignment in the \acrshort{MIR}.
Listing \ref{lst:visit-assign} shows precisely
that all cases of copying, moving, or referencing the \acrfull{RHS}
are handled by the same mechanism:
The \acrfull{LHS} is linked to the \acrfull{RHS}
if the type of the variable is a supported synchronization variable.
The listing also shows how the compiler uses nested enums to model its data.
Inside the variants of a right-hand side value (\Rustinline{rustc_middle::mir::Rvalue}),
one can find operands (\Rustinline{rustc_middle::mir::Operand}).
These operands also appear when passing function arguments.

\begin{listing}[!htb]
  \begin{minted}{Rust}
    fn visit_assign(
        &mut self,
        place: &rustc_middle::mir::Place<'tcx>,
        rvalue: &rustc_middle::mir::Rvalue<'tcx>,
        location: rustc_middle::mir::Location,
    ) {
        match rvalue {
            rustc_middle::mir::Rvalue::Use(
                rustc_middle::mir::Operand::Copy(rhs) | rustc_middle::mir::Operand::Move(rhs),
            )
            | rustc_middle::mir::Rvalue::Ref(_, _, rhs) => {
                let function = self.call_stack.peek_mut();
                link_if_sync_variable(place, rhs, &mut function.memory, function.def_id, self.tcx);
            }
            rustc_middle::mir::Rvalue::Aggregate(_, operands) => {
                let function = self.call_stack.peek_mut();
                handle_aggregate_assignment(
                    place,
                    &operands.raw,
                    &mut function.memory,
                    function.def_id,
                    self.tcx,
                );
            }
            // No need to do anything for the other cases for now.
            _ => {}
        }

        self.super_assign(place, rvalue, location);
    }
  \end{minted}
  \caption{The custom implementation of \Rustinline{visit_assign} to track synchronization variables.}
  \label{lst:visit-assign}
\end{listing}

The most peculiar case is the aggregate assignment.
It materializes from assignments in Rust source code
that create tuples, closures, or \Rustinline{structs}.
It necessitates special handling as the value to be linked in memory must be assembled
from the constituents of the aggregated value that are a synchronization variable.
This implies that the \Rustinline{Memory} solely retains the portion of the aggregated value
formed by the synchronization variables.

In every instance, the handling of assignments has no impact on the Petri net.
No places or transitions are added when intercepting assignments.

Finally, some memory locations are passed to a new thread when
calling \Rustinline{std::thread::spawn}
and mapped again to the memory of the thread's function.
The next section will demonstrate the method used to accomplish this.