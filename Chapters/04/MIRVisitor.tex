\section{MIR visitor}

This section is dedicated to a pivotal component that serves as the backbone of the translator:
the MIR Visitor trait\footnote{\url{https://doc.rust-lang.org/stable/nightly-rustc/rustc_middle/mir/visit/trait.Visitor.html}}.
This trait enables straightforward navigation of the \acrshort{MIR} of the Rust source code.
In other words, it acts as the glue that seamlessly binds the various components of the translator together.

The MIR Visitor trait plays a fundamental role in the translation process
by providing a structured approach to traverse and analyze the \acrshort{MIR}.
It offers a set of methods that can be implemented to perform specific actions
at different points during the traversal.
By employing this trait,
the translator gains the ability to systematically explore the \acrshort{MIR}
and extract the necessary information for generating the corresponding Petri net.

The implemented methods within the MIR Visitor trait serve
as entry points for handling different elements encountered during the traversal.
These methods allow for customized processing of specific \acrshort{MIR} constructs,
e.g., basic blocks, statements, terminators, assignments, constants, etc.
By defining appropriate behavior for each method,
the translator can efficiently extract relevant data and
make informed decisions based on the encountered MIR elements.

It was not required to implement all possible methods.
If not defined, the methods in MIR Visitor simply call
the corresponding \Rustinline{super} method and continue the traversal.
For instance, \Rustinline{visit_statement} calls
\Rustinline{super_statement} if no custom implementation is present.
In the case of the translator, the implemented methods are:

\begin{itemize}
  \item \Rustinline{visit_basic_block_data} for keeping track of the basic block currently being translated.
  \item \Rustinline{visit_assign} for keeping track of assignments of synchronization variables
        (mutexes, mutex guards, join handles, and condition variables).
  \item \Rustinline{visit_terminator} for processing the terminator statement of each basic block,
        that is, connecting the basic blocks.
\end{itemize}

To start visiting the MIR, the method \Rustinline{visit_body} must be used.
Listing \ref{lst:mir-visit-body} shows the corresponding function in the translator.

\begin{listing}[!htb]
  \begin{minted}{Rust}
    /// Main translation loop.
    /// Translates the function from the top of the call stack.
    /// Inside the MIR Visitor, when a call to another function happens, this method will be called again
    /// to jump to the new function. Eventually a "leaf function" will be reached, the functions will exit and the
    /// elements from the stack will be popped in order.
    fn translate_top_call_stack(&mut self) {
        let function = self.call_stack.peek();
        // Obtain the MIR representation of the function.
        let body = self.tcx.optimized_mir(function.def_id);
        // Visit the MIR body of the function using the methods of `rustc_middle::mir::visit::Visitor`.
        // <https://doc.rust-lang.org/stable/nightly-rustc/rustc_middle/mir/visit/trait.Visitor.html>
        self.visit_body(body);
        // Finished processing this function.
        self.call_stack.pop();
    }    
  \end{minted}
  \caption{The method in the \Rustinline{Translator} that starts the traversal of the \acrshort{MIR}}
  \label{lst:mir-visit-body}
\end{listing}

In conclusion, the MIR Visitor trait simplifies remarkably the translation
as it is not necessary to implement a traversal mechanism thanks to the provided
compiler interfaces.
This also makes the translator more robust and resistant to changes in \emph{rustc}.
If the string representation of the \acrshort{MIR} changes, the translator is left unaffected.
As long as the internal interfaces for accessing the MIR stay the same,
the translator can navigate the MIR semantically and not based on how it is printed to the user.

As a last remark, similar traits exist for other intermediate representations:

\begin{itemize}
  \item \acrshort{AST}: \url{https://doc.rust-lang.org/stable/nightly-rustc/rustc_ast/visit/trait.Visitor.html}
  \item \acrshort{HIR}: \url{https://doc.rust-lang.org/stable/nightly-rustc/rustc_hir/intravisit/trait.Visitor.html}
  \item \acrshort{THIR}: \url{https://doc.rust-lang.org/stable/nightly-rustc/rustc_middle/thir/visit/trait.Visitor.html}
\end{itemize}

Various components in the compiler implement these traits
to navigate the intermediate representations.
To put it roughly, they are analogous to iterators for collections.