\section{Condition variable (\texttt{std::sync::Condvar})}

A condition variable is a synchronization primitive used in concurrent programming
to enable threads to wait for a certain condition
to become true before proceeding with their execution.
Threads block and wait until they are notified by another thread
that the desired condition has been met.
Condition variables are typically associated with a mutex,
which ensures exclusive access to the shared data that the condition depends on.
When a thread waits on a condition variable, it releases the associated mutex,
allowing other threads to make progress.
When the condition becomes true or some event occurs,
a notifying thread signals the condition variable,
allowing one or more waiting threads to resume their execution.

The semantics of condition variables, as well as examples in pseudocode,
were introduced in Sec. \ref{sec:condition-variables}.
The understanding of the precise behavior of condition variables in all circumstances
is a prerequisite for this section.

This section provides an elaborate explanation of the \acrshort{PN} model
used to represent condition variables from the Rust standard library.
It is followed by a practical example that aims to enhance the clarity of the concepts.
Finally, the algorithms for the translation of condition variable functions are outlined.

\subsection{Petri net model}

\subsection{A practical example}

\subsection{Algorithms for condition variable translation}
\label{sec:condvar-algorithms}

To wrap up this section, we will present a concise summary
of the algorithms utilized in the translation of condition variables.
The additions required to the mutex algorithms are included afterward as well.

When a call to \Rustinline{std::sync::Condvar::new} is encountered:

\begin{enumerate}
      \item Translate the function call using the model seen in Fig. \ref{fig:function-with-cleanup}.
      \item Create a new \Rustinline{Condvar}\footnote{\url{https://github.com/hlisdero/cargo-check-deadlock/blob/main/src/translator/sync/condvar.rs}}
            structure with an index to identify it unequivocally across the Petri net.
      \item Link the return value of \Rustinline{std::sync::Condvar::new},
            the new condition variable, to the \Rustinline{Condvar} structure.
\end{enumerate}

When a call to \Rustinline{std::sync::Condvar::notify_one} is encountered:

\begin{enumerate}
      \item Translate the function call using the model seen in Fig. \ref{fig:simplest-function}.
            Ignore the cleanup place because, otherwise, any call may fail,
            which amounts to the notify operation not being present in the program,
            leading to a false lost signal.
            This is equivalent to assuming that the \Rustinline{notify_one} function never fails.
      \item Retrieve the \Rustinline{self} reference to the condition variable
            on which the function is called.
      \item Add an arc from the transition representing the function call
            to the \Rustinline{notify} place of the underlying condition variable.
\end{enumerate}

When a call to \Rustinline{std::sync::Condvar::wait}
or \Rustinline{std::sync::Condvar::wait_while} is encountered:

\begin{enumerate}
      \item Ignore the cleanup place because, otherwise, any call may fail,
            which amounts to the wait operation not being present in the program,
            leading to an incorrect result.
            We must force the Petri net
            to ``wait'' for the notifying signal to be sent.
            This is equivalent to assuming that the \Rustinline{wait}
            or \Rustinline{wait_while} function never fails.
      \item Retrieve the \Rustinline{self} reference to the condition variable
            on which the function is called.
      \item Extract the mutex guard passed into the function.
      \item Connect the start and end place to the
            \Rustinline{wait_start} and \Rustinline{notify_received} transitions respectively.
      \item Link the return value, the same mutex guard that was passed as an argument,
            to the \Rustinline{MutexGuard} structure.
      \item Notify the translator that the mutex received must be linked to this \Rustinline{Condvar}.
            For this purpose, use the enum variant \Rustinline{PostprocessingTask::LinkMutexToCondvar}.
            This task will be processed after translating all the threads.
\end{enumerate}

\subsubsection{Modifications to the mutex algorithms}

As stated before, the mutex algorithms require some additions to successfully
perform missed signal detection.

Add the following to the handler of the \Rustinline{std::sync::Mutex::new} function:

\begin{enumerate}
      \item Notify the translator that a new mutex has been created.
            For this purpose, use the enum variant \Rustinline{PostprocessingTask::NewMutex}.
            This task will be processed after translating all the threads.
\end{enumerate}

When a call to \Rustinline{std::result::Result::<T, E>::unwrap} is encountered:

\begin{enumerate}
      \item Check that the \Rustinline{self} reference is a mutex or a mutex guard.
      \item Translate the function call using the model seen in Fig. \ref{fig:simplest-function}.
            Ignore the cleanup place because, otherwise, any call may fail,
            as if the mutex lock operation
            were not present in the program, leading to a false lost signal.
            This is equivalent to assuming that the \Rustinline{unwrap} function never fails
            when applied to a variable linked to a mutex or a mutex guard.
\end{enumerate}

When a call to \Rustinline{std::ops::Deref::deref}
or \Rustinline{std::ops::DerefMut::deref_mut} is encountered:

\begin{enumerate}
      \item Check that the \Rustinline{self} reference is a mutex or a mutex guard.
      \item Translate the function call using the model seen in Fig. \ref{fig:simplest-function}.
            Ignore the cleanup place because, otherwise, any call may fail,
            as if the mutex lock operation
            were not present in the program, leading to a false lost signal.
            This is equivalent to assuming that the \Rustinline{deref}
            and \Rustinline{deref_mut} functions never fail
            when dereferencing a variable linked to a mutex or a mutex guard.
\end{enumerate}

It should now be clear to the reader that the algorithms for missed signal detection are
fundamentally of higher complexity and ought to handle more border cases than
those for detecting simple deadlocks caused by incorrect usage of mutexes or
calling \Rustinline{join} on threads that never terminate.

It is worth mentioning that some border cases arise
due to the inclusion of the cleanup logic
from the \acrshort{MIR} in the \acrshort{PN} model.
If the implementation instead skipped this,
under the hypothesis that Rust standard library functions may never panic,
then the algorithms would become simpler.
Sec. \ref{sec:future-work-no-cleanup} is concerned with
the question of not modeling the cleanup paths.