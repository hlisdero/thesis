\section{Notable test programs}

To wrap up this chapter, we introduce two noteworthy test programs that
illustrate the current capabilities of the tool developed in this thesis.
Our intention is to inspire others to contribute to this project or,
at the very least, generate interest in the field of model checking.

First, Listing \ref{lst:dating-philosophers} showcases
a simple version of the famous Dining Philosophers Problem proposed by Dijkstra.
This version, affectionately nicknamed ``Dating Philosophers'',
has only two philosophers and two forks on the table.
A mutex needs to be locked to access each fork.
When the philosophers try to grab both forks to eat, the program deadlocks,
which is easy to verify by inspection.
This deadlock is successfully detected by the tool.
Moreover, a more complex version with 5 philosophers,
for which the deadlock is also detected, is included in the
repository\footnote{\url{https://github.com/hlisdero/cargo-check-deadlock/blob/main/examples/programs/thread/dining\_philosophers.rs}}.
It was omitted here due to the space constraints.

\begin{listing}[!htbp]
    \begin{minted}{Rust}
  use std::sync::{Arc, Mutex};
  use std::thread;
    
  fn main() {
      let fork0 = Arc::new(Mutex::new(0));
      let fork1 = Arc::new(Mutex::new(1));
    
      let philosopher0 = {
          let left_fork = fork0.clone();
          let right_fork = fork1.clone();
          thread::spawn(move || {
              let _left = left_fork.lock().unwrap();
              let _right = right_fork.lock().unwrap();
          })
      };
    
      let philosopher1 = {
          let left_fork = fork1.clone();
          let right_fork = fork0.clone();
          thread::spawn(move || {
              let _left = left_fork.lock().unwrap();
              let _right = right_fork.lock().unwrap();
          })
      };
    
      // Wait for all threads to finish
      philosopher0.join().unwrap();
      philosopher1.join().unwrap();
  }    
  \end{minted}
    \caption{A reduced version of the dining philosophers problem that deadlocks}
    \label{lst:dating-philosophers}
\end{listing}

Second, observe the program in Listing \ref{lst:producer-consumer}.
It models the classical producer-consumer problem.
It uses a condition variable and a buffer with capacity for a single element.
The access to the buffer is protected by a mutex.
The producer generates 10 elements sequentially and
the consumer processes them as they become available.
The tool successfully verifies the absence of deadlock in the program.

\begin{listing}[!htbp]
    \begin{minted}{Rust}
  use std::sync::{Arc, Condvar, Mutex};
  use std::thread;
  
  fn main() {
      let buffer = Arc::new((Mutex::new(0), Condvar::new(), Condvar::new()));
  
      let producer_buffer = buffer.clone();
      let consumer_buffer = buffer.clone();
  
      let _producer = thread::spawn(move || {
          for i in 1..10 {
              let (lock, cvar_producer, cvar_consumer) = &*producer_buffer;
              let mut buffer = lock.lock().unwrap();
  
              while *buffer != 0 {
                  buffer = cvar_producer.wait(buffer).unwrap();
              }
  
              *buffer = i;
              println!("Produced: {}", i);
  
              cvar_consumer.notify_one();
          }
      });
  
      let _consumer = thread::spawn(move || loop {
          let (lock, cvar_producer, cvar_consumer) = &*consumer_buffer;
          let mut buffer = lock.lock().unwrap();
  
          while *buffer == 0 {
              buffer = cvar_consumer.wait(buffer).unwrap();
          }
  
          let item = *buffer;
          *buffer = 0;
          println!("Consumed: {}", item);
  
          cvar_producer.notify_one();
      });
  }
  \end{minted}
    \caption{A solution to the producer-consumer problem}
    \label{lst:producer-consumer}
\end{listing}