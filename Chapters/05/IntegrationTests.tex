\section{Integration tests}
\label{sec:integration-tests}

We will now examine the integration tests,
which serve as the foundation for the translator's testing framework.
Two types of tests exist currently:

\begin{itemize}
  \item Translation tests.
  \item Deadlock detection tests.
\end{itemize}

\subsection{Translation tests}

In translation tests, a given program is processed
\emph{without} performing the deadlock analysis.
As a result, three text files containing the model
in DOT, \acrshort{PNML}, and \acrshort{LoLA} formats are generated.
These files are then compared to the expected output,
which is stored in the repository and serves as documentation as well.

The expected output was verified manually
using the tools presented in Sec. \ref{sec:visualizing-results}.
It was committed to the repository
when the translator was first able to pass the test.
If a regression in \emph{rustc} occurs,
then the expected output files are updated accordingly.
This has happened some times in the past.
See for instance
this commit\footnote{\url{https://github.com/hlisdero/cargo-check-deadlock/commit/881a3873a3b060e70bc727f670f9426d14327fa2}}
or this one\footnote{\url{https://github.com/hlisdero/cargo-check-deadlock/commit/b032fa3cc13e631950a802dcd3f755c548afde86}}.

\subsection{Deadlock detection tests}

Deadlock detection tests are closer to an end-to-end test of the translator.
They generate the file in \acrshort{LoLA} format for the test program
and instruct the translator to perform the deadlock analysis.
The output is then contrasted to the known behavior of the test program, i.e.,
it deadlocks or it does not deadlock.
If \acrshort{LoLA} produces an incorrect result, then the test fails.
In such a case, the \acrshort{PN} model should be analyzed
to find the source of the error.

\subsection{Test structure}

The test programs are in the folder \Rustinline{examples/programs}.
For each test program, there is a folder in \Rustinline{examples/results}
that contains the three files
\Rustinline{net.dot}, \Rustinline{net.pnml}, and \Rustinline{net.lola}.

The tests are grouped into categories:

\begin{itemize}
  \item Basic: For basic programs like ``Hello, World!'' and a simple arithmetic calculator.
  \item Condvar: For programs concerning condition variables.
  \item Function call: For programs that test different types
        of function calls seen in Sec. \ref{sec:function-calls}.
  \item Mutex: For programs that use mutexes.
  \item Statement: For programs that test specific constructs such as a \Rustinline{match},
        an infinite loop, an \Rustinline{Option}, a call to \Rustinline{panic!},
        or \Rustinline{std::process::abort}.
  \item Thread: For programs involving multiple threads.
\end{itemize}

The structure of the folders in \Rustinline{examples/} mimics the file structure
of the integration tests in \Rustinline{tests/}.
As usual, the whole test suite can be run with the \Rustinline{cargo test} command.

