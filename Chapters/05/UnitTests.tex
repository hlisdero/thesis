\section{Unit tests}

The unit tests form the base of the test suite.
The \acrshort{PN} library and the data structures
used in the translator rely on them extensively.
By testing the underlying data structures meticulously,
potential issues can be identified and resolved early in the development cycle
before continuing work on the higher-level components of the translator.

\subsection{Petri net library}

The \acrshort{PN} library \Rustinline{netcrab} uses unit tests to verify that
adding places, transitions, and arcs to a net behaves as expected.
The translator performs these operations often so it is important to verify them.
The iterators on which the export formats are built are tested as well.

On the other hand, each one of the three export formats
(DOT, \acrshort{PNML} and \acrshort{LoLA}) comes with unit tests
to check that the output is generated correctly for the following simple cases:

\begin{itemize}
  \item Empty net.
  \item A \acrshort{PN} with 5 places and 0 transitions.
  \item A \acrshort{PN} with 0 places and 5 transitions.
  \item A \acrshort{PN} with 5 places marked with different numbers of tokens.
  \item A \acrshort{PN} with a chain topology.
  \item A \acrshort{PN} with 1 place and 1 transition connected in a loop.
\end{itemize}

These tests helped to troubleshoot bugs related to the \acrshort{LoLA}
format.
For concrete examples,
see this commit\footnote{\url{https://github.com/hlisdero/netcrab/commit/5745e0da5d27bd709ef479f45a6d2e75974d3745}}
or this other commit\footnote{\url{https://github.com/hlisdero/netcrab/commit/dbce3f8999ece32e6731527c303a7b59858991f9}}.

\subsection{Stack}

The simple
\Rustinline{Stack}\footnote{\url{https://github.com/hlisdero/cargo-check-deadlock/blob/main/src/data_structures/stack.rs}}
data structure is employed to implement the call stack in the \Rustinline{Translator},
as seen in Sec. \ref{sec:function-calls}.
Unit tests demonstrate the supported methods and check some simple use cases.

\subsection{Hash map counter}

Analogous to the stack,
the \Rustinline{HashMapCounter}\footnote{\url{https://github.com/hlisdero/cargo-check-deadlock/blob/main/src/data_structures/hash_map_counter.rs}}
contains some unit tests to verify that the methods work as intended.
This data structure forms the basis for the function counter in the \Rustinline{Translator}
that keeps track of how many times each function was called
to generate a unique incremental index for the transition labels.