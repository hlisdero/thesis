The inclusion of a dedicated chapter on testing in the thesis underscores
the significance of this indispensable aspect of the development process.
Tests play a fundamental role in ensuring
the reliability and correctness of the software implementation.
A comprehensive test suite has been developed
to cover the extensive functionality and behavior
of the translator and the \acrshort{PN} library.

The tests encompass multiple levels
that will be elucidated in the subsequent sections.
At the lowest level, unit tests are conducted
to verify the correctness of the data structures
employed within the translator and the \acrshort{PN} library.
These tests target individual components,
thoroughly examining their functionality in isolation.

In addition to unit tests,
a suite of integration tests has been constructed incrementally
to evaluate the translator's adherence to expected behavior.
These tests consist of test programs that simulate simple scenarios where
the resulting file output is compared against the expected results.
This testing methodology helps to uncover any regressions in the compiler
and confirms that the translator functions reliably in the supported use cases.

Furthermore, we incorporated a description of how to generate the \acrshort{MIR} and
visualize the result of the translation to aid in the debugging process.
The tooling enables exposing the internal details in an accessible and understandable manner.

Later in this chapter, the usage of the model checker \acrshort{LoLA}
and its integration within the translator is explained.
The model checker provides more features than the minimal set
that was integrated into the translator to answer the deadlock detection problem.
Hence, it is beneficial to explore
which features the model checker provides for debugging the \acrshort{PN} translation.

Finally, the capabilities of the tool are demonstrated by means of two test programs
that model classical problems in concurrent programming.