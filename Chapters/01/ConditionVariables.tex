% !TeX root = ../../Thesis.tex
\documentclass[../../Thesis.tex]{subfiles}
\graphicspath{{\subfix{../../images/}}}

\begin{document}

\section{Condition variables}

Condition variables are a synchronization primitive in concurrent programming
that allows threads to efficiently wait
for a specific condition to be met before proceeding.
They were first introduced by \cite{hoare1974monitors}
as part of a building block for the concept of monitor developed
originally by \cite{hansen1973operating}.

Following the classic definition, two main operations
can be called on a condition variable:

\begin{itemize}
    \item \texttt{wait}: Blocks the current thread or process.
          In some implementations, the associated mutex
          is released as part of the operation.
    \item \texttt{signal}: Wakes up one thread or process
          waiting on the condition variable.
          In some implementations, the associated mutex lock is immediately
          acquired by the signaled thread or process.
\end{itemize}

Condition variables are typically associated
with a boolean predicate (a condition) and a mutex.
The boolean predicate is the condition
on which the threads or processes are waiting for.
When it is set to a particular value (either true or false),
the thread or process should continue executing.
The mutex ensures that only one thread or process
may access the condition variable at a time.

Condition variables do not contain an actual value
accessible to the programmer inside of them.
Instead, they are implemented using a queue data structure, where
threads or processes are added to the queue when they enter the wait state.
When another thread or process signals the condition,
an element from the queue is selected to resume execution.
The specific scheduling policy may vary depending on the implementation.

Over the years, various implementations and optimizations have been developed
for condition variables to improve performance and reduce overhead.
For example, some implementations allow multiple threads to be awakened at once
(an operation called \emph{broadcast}),
while others use a priority queue
to ensure that the most important threads are awakened first.

Condition variables are part of the POSIX standard library for threads \cite{nichols1996pthreads}
and they are now widely used in concurrent programming languages and systems.
They are found among others in:

\begin{itemize}
    \item UNIX\footnote{\url{https://man7.org/linux/man-pages/man3/pthread_cond_init.3p.html}},
    \item Rust\footnote{\url{https://doc.rust-lang.org/std/sync/struct.Condvar.html}}
    \item Python\footnote{\url{https://docs.python.org/3/library/threading.html}}
    \item Go\footnote{\url{https://pkg.go.dev/sync}}
    \item Java\footnote{
              \url{https://docs.oracle.com/en/java/javase/20/docs/api/java.base/java/util/concurrent/locks/Condition.html}}
\end{itemize}

Despite their widespread use, condition variables can be tricky to use correctly,
and incorrect use can lead to subtle and
hard-to-debug errors such as missed signals or spurious wakeups.
We will look now at these errors in detail.

\subsection{Missed signals}

A missed signal occurs when a thread or process waiting on a condition variable
fails to receive a signal even though it has been emitted.
This can happen due to a race condition, where the signal is emitted
before the thread enters the wait state, causing the signal to be missed.

To illustrate the concept of a missed signal, we will look at an example.
Suppose we have two threads, T1 and T2,
and a shared integer variable called \texttt{flag}.
T1 sets \texttt{flag} to \texttt{true} and signals
a condition variable \texttt{cv} to wake up T2,
which is waiting on \texttt{cv} to know when \texttt{flag} has been set.
T2 waits on \texttt{cv} until it receives a signal from T1.
Listing \ref{lst:missed-signal-example} shows the corresponding pseudocode.

\begin{listing}
    \begin{minted}{C++}
            // T1
            lock.acquire()
            flag = true
            cv.signal()   // Signal T2 to wake up
            lock.release()
            
            // T2
            lock.acquire()
            while (flag == false)   // Wait until flag has changed
                cv.wait(lock)
            lock.release()
      \end{minted}
    \caption{Pseudocode for a missed signal example.}
    \label{lst:missed-signal-example}
\end{listing}

Now, suppose that T1 sets \texttt{flag} and signals \texttt{cv},
but T2 has not yet entered the wait state on \texttt{cv} due to some scheduling delay.
In this case, the signal emitted by T1 could be missed by T2,
as shown in the following sequence of events:

\begin{enumerate}
    \item T1 acquires the lock and sets \texttt{flag} to \texttt{true}.
    \item T1 signals \texttt{cv} to wake up T2.
    \item T1 releases the lock.
    \item T2 acquires the lock and checks if \texttt{flag} has changed.
          Since \texttt{flag} is still \texttt{false},
          T2 enters the wait state on \texttt{cv}.
    \item Due to scheduling delays or other factors,
          T2 does not receive the signal emitted by T1
          and remains stuck in the wait state forever.
\end{enumerate}

This scenario illustrates the concept of a missed signal,
where a thread waiting on a condition variable fails
to receive a signal even though it has been emitted.
To prevent missed signals, it is important to ensure that
threads waiting on condition variables are properly synchronized
with the threads emitting signals and
that there are no race conditions or timing issues
that could cause signals to be missed.

\subsection{Spurious wakeups}

A spurious wakeup happens when a thread waiting on a condition variable wakes up
without receiving a signal or notification from another thread.
Reasons for this are multiple: hardware or operating system interrupts,
internal implementation details of the condition variable
or other unpredictable factors.

Reusing the situation described in the previous section and
the pseudocode shown in Listing \ref{lst:missed-signal-example},
suppose now that T1 sets \texttt{flag} to \texttt{true} and signals \texttt{cv},
but T2 wakes up without receiving the signal emitted by T1.

This is precisely the spurious wakeup.
The following sequence of events leads to this unfortunate outcome:

\begin{enumerate}
    \item T1 acquires the lock and sets \texttt{flag} to \texttt{true}.
    \item T1 signals \texttt{cv} to wake up T2.
    \item T1 releases the lock.
    \item T2 acquires the lock and checks if \texttt{flag} is \texttt{true}.
          Since \texttt{flag} is still \texttt{false},
          T2 enters the wait state on \texttt{cv}.
    \item Due to some internal implementation detail
          of the condition variable or other unpredictable factors,
          T2 wakes up without receiving the signal emitted by T1
          and continues executing the next statement in its code.
\end{enumerate}

This example demonstrates the idea of a spurious wakeup,
in which a thread waiting on a condition variable wakes up
without receiving a signal or notification from another thread.
To prevent spurious wakeups, it is important to use a loop
to recheck the condition after waking up from a wait state,
as shown in the pseudocode for T2 (line 9).
This ensures that the thread does not proceed
until the condition it is waiting for has indeed occurred.
If the while loop were not there, a spurious wakeup would cause T2 to
continue executing after the call to \texttt{wait},
regardless of whether a signal was emitted by T1 or not.

\end{document}