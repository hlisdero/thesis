% !TeX root = ../Thesis.tex
\documentclass[../Thesis.tex]{subfiles}
\graphicspath{{\subfix{../images/}}}

\begin{document}

In \cite{rawson2022petri}, the authors propose a generalized model
based on colored Petri nets and implement an open-source middleware framework
in Rust\footnote{\url{https://github.com/MarshallRawson/nt-petri-net}}
to build, design, simulate and analyze the resulting Petri nets.

\acrfull{CPN} are a type of Petri net that can represent
more complex systems than traditional Petri nets.
In a CPN, tokens have a specific value associated with them,
which can represent various attributes or properties of the system being modeled.
This allows for more detailed and accurate modeling of real-world systems,
including those with complex data structures and behaviors.
In the visual representation, each token has a color
(analogous to a type in programming languages)
and the transitions expect tokens from a particular
color (type) and can generate tokens
of the same color or tokens of a different color.
As a short example, consider a transition
with two input places and one output place
representing the mixing of primary colors.
If the input token colors are red and blue, then the output token color is purple.
If the input token colors are yellow and blue, then the output token color is green.

The model proposed by the authors is an even more general type of Petri net,
named \acrfull{NT-PN}, which allows transitions to fire
without having all their input places marked with tokens,
while also allowing each transition to define
which output places should be marked depending on the input.
In other words, each transition defines arbitrary rules for its firing to take place.
They explain briefly how the Petri net could be analyzed
to solve for the maximal number of useful threads to execute the task modeled therein.
They also mention the modeling step as a tool for checking for erroneous states
before deploying an electronic or computer system.

In \cite{deboer2013petri}, a translation from a formal language to Petri nets
for deadlock detection in the context of active objects and futures is presented.
The formal language chosen is \acrfull{Creol}.
It is an object oriented modeling language designed for specifying distributed systems.
In this paper, the program is made of asynchronously communicating active objects
where futures are used to handle return values,
which can be retrieved via a lock detaining \texttt{get} primitive (blocking)
or a lock releasing \texttt{claim} primitive (non-blocking).
After translating the program to a Petri net,
reachability analysis is applied to detect deadlocks.
This paper shows that a traslation of asynchronous communication strategies to
Petri nets is also possible.

\end{document}