\section{Petri nets libraries in Rust}

As part of the development of the translation of the source code to a Petri net,
it is necessary to use a Petri net library for the Rust programming language.
A quick search of the packages available on
\emph{crates.io}\footnote{\url{https://crates.io/}}, GitHub, and GitLab
revealed that there is, unfortunately, no well-maintained library.

Some Petri net simulators were found such as:

\begin{itemize}
    \item pns\footnote{\url{https://gitlab.com/porky11/pns}}:
          Programmed in C. It does not offer the option
          to export the resulting network to a standard format.
    \item PetriSim\footnote{\url{https://staff.um.edu.mt/jskl1/petrisim/index.html}}:
          An old DOS/PC simulator programmed in Bordland Pascal.
    \item WOLFGANG\footnote{\url{https://github.com/iig-uni-freiburg/WOLFGANG}}:
          A Petri net editor in Java, maintained by the Department of Computer Science
          at the University of Freiburg, Germany.
\end{itemize}

Regrettably, none of them meet the requirements of the task.

Since a Petri net is a graph, the possibility
of using a graph library and modifying it to suit the objectives of this work was considered.
Two graph libraries were found in Rust:

\begin{itemize}
    \item petgraph\footnote{\url{https://docs.rs/petgraph/latest/petgraph/}}:
          The most widely used library for graphs in \textit{crates.io}.
          It offers an option to export to the DOT format.
    \item gamma\footnote{\url{https://github.com/metamolecular/gamma}}:
          Unstable and unchanged since 2021. It does not offer the ability to export the graph.
\end{itemize}

None of the possibilities satisfies the requirement
to export the resulting network to the \acrshort{PNML} format.
In addition, if a graph library is used,
the operations of a Petri net should be implemented as a \emph{wrapper} around a graph,
which reduces the possibility of optimizations for our use case
and hinders the long-term extensibility of the project.

In conclusion, it is imperative
to implement a Petri net library in Rust from scratch as a separate project.
This contributes one more tool to the community that could be reused in the future.