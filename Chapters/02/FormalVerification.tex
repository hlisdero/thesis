\section{Formal verification of Rust code}

There are numerous automatic verification tools available for Rust code.
A recommended first approximation to the topic is
the survey produced by Alastair Reid, a researcher at Intel.
It explicitly lists that most formal verification tools
do not support concurrency \cite{reid2021}.

The \emph{Miri}\footnote{\url{https://github.com/rust-lang/miri}} interpreter
developed by the Rust project on GitHub is an experimental interpreter
for the intermediate representation of the Rust language
(Mid-level Intermediate Representation, commonly known as ``\acrshort{MIR}'')
that allows executing standard cargo project binaries
in a granularized way, instruction by instruction,
to check for the absence of \acrfull{UB}
and other errors in memory handling.
It detects memory leaks, unaligned memory accesses, data races,
and precondition or invariant violations in code marked as \texttt{unsafe}.

\cite{toman2015crust} introduces a formal checker for Rust
that does not require modifications to the source code.
It was tested on past versions of modules from the Rust standard library.
As a result, errors were detected in the use of memory in unsafe Rust code
which in reality took months to be discovered manually by the development team.
This exemplifies the importance of using automatic verification tools
to complement manual code reviews.