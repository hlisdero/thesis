\section{Exchange file formats for Petri nets}

As observed in the preceding chapter,
Petri nets are a widely used tool for modeling software systems.
However, due to the different classes of Petri nets
(simple Petri nets, high-level Petri nets, timed Petri nets,
stochastic Petri nets, colored Petri nets, to name a few),
designing a standardized exchange file format
compatible with all applications has proven challenging.
One reason for this is that
Petri nets can be implemented and represented in multiple ways,
depending on the specific objectives, given that they are a type of graph.

In order to guarantee a certain degree of interoperability between the tool developed
as part of this thesis and other existing and future tools, it is crucial
to investigate which file formats would be more convenient to support.
The aim is to support file formats that are suited to analysis as well as visualization,
allowing for the possibility of extension to additional formats in the future,
via a well-defined API in the Petri net library.
A literature review led to three relevant file formats that are presented next.

\subsection{Petri Net Markup Language}

The \acrfull{PNML}\footnote{\url{https://www.pnml.org/}}
is a standard file format designed
for the exchange of Petri nets
among different tools and software applications.
Its development was initiated
at the `Meeting on XML/SGML based Interchange Formats for Petri Nets'
held in Aarhus in June 2000 \cite{jungel2000petri},
with the goal of providing
a standardized and widely accepted format for Petri net models.
PNML is an ISO standard consisting, as of 2023, of three parts:

\begin{itemize}
      \item ISO/IEC 15909-1:2004\footnote{\url{https://www.iso.org/standard/38225.html}}
            (and its latest revision ISO/IEC 15909-1:2019\footnote{\url{https://www.iso.org/standard/67235.html}})
            for concepts, definitions, and graphical notation.
      \item ISO/IEC 15909-2:2011\footnote{\url{https://www.iso.org/standard/43538.html}}
            for defining a transfer format based on XML.
      \item ISO/IEC 15909-3:2021\footnote{\url{https://www.iso.org/standard/81504.html}}
            for the extensions and structuring mechanisms.
\end{itemize}

It has become a de-facto standard
for exchanging Petri net models across different tools and systems.
It resulted from many years of hard work
to unify the notation as discussed in \cite{hillah:hal-01176335}.

\acrshort{PNML} has been designed to be a flexible and extensible format
that can represent different classes of Petri nets,
including simple Petri nets and high-level Petri nets.
It is based on the \acrfull{XML}
which makes it easy to read and parse by humans and machines alike.
Additionally, \acrshort{PNML} supports the use of metadata
to provide additional information about the Petri net models,
such as authorship, date of creation, and licensing information.

The development of \acrshort{PNML} has significantly improved
the interoperability and exchange of Petri net models
among different tools and systems.
Before the adoption of \acrshort{PNML},
exchanging Petri net models was a challenging task,
as different tools used proprietary formats
that were often incompatible with each other.
\acrshort{PNML} has greatly simplified this process,
enabling researchers and practitioners
to share and collaborate on Petri net models with ease.
Its use has also facilitated the development of new tools
and software applications for Petri nets,
as it provides a standard format
that can be easily parsed and processed by different systems.
For instance, it is the format used
in \cite{zhang2022deadlocks} and it is supported in \cite{meyer2020}.

\subsection{GraphViz DOT format}

The DOT format is a graph description language
used for creating visual representations of graphs and networks,
which is part of the open-source GraphViz suite\footnote{\url{https://graphviz.org/}}.
It was created in the early 1990s at AT\&T Labs Research
as a simple, concise, and human-readable language for describing graphs.
The GraphViz suite provides several tools for working with DOT files,
including the ability to automatically generate layouts for complex graphs and
to export visualizations in a range of formats, including PNG, PDF, and SVG.

DOT can be used to represent Petri nets in a graphical format, which makes it easy
to visualize the structure and behavior of the system being modeled.
It is particularly useful for visualizing large Petri nets,
as the user can navigate through the image
to gain an understanding of how the tokens flow through the net.

The DOT format is text-based and easy to use,
making it a popular choice for generating visual representations of graphs.
This simplicity also means that DOT files can be easily generated by programs
and can be read by a wide range of software tools,
which is essential for interoperability.
Additionally, DOT allows for the specification of various graph properties,
such as node shapes, colors, and styles \cite{dot2015},
which can be used to represent different aspects of a Petri net,
such as places, transitions, and arcs.
This flexibility in specifying visual properties also enables users
to customize the visualization to their needs and
to highlight particular features of the Petri net
that are important for their analysis.

\subsection{LoLA - Low Level Petri Net Analyzer}

\acrfull{LoLA} is a state-of-the-art model checker
developed at the University of Rostock in
Germany\footnote{\url{https://theo.informatik.uni-rostock.de/theo-forschung/tools/lola/}}
and published under the GNU Affero General Public License.
\acrshort{LoLA} is a tool that can check
if a system satisfies a given property expressed in \acrfull{CTL*}.
Its particular strength is the evaluation of simple properties
such as deadlock freedom or reachability as stated on the website.

This is the model checker used in \cite{meyer2020} and in this work.
Therefore, it is necessary to implement the file format required by the tool.
Examples are presented in Sec. \ref{sec:integration-tests}.