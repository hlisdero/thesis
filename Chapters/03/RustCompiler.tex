\section{Rust compiler: \emph{rustc}}
\label{sec:rustc}

The Rust compiler, \emph{rustc}, is responsible for translating Rust code into executable code.
However, \emph{rustc} is not a traditional compiler in the sense
that it performs multiple passes over the code,
as described in Sec. \ref{sec:compiler-architecture}.
Instead, \emph{rustc} is built on a query-based system that supports incremental compilation.

In \emph{rustc}'s query system, the compiler computes a dependency graph between code artifacts,
including source files, crates, and intermediate artifacts, such as object files.
The query system then uses this graph to efficiently recompile
only those artifacts that have changed since the last
compilation\footnote{\url{https://rustc-dev-guide.rust-lang.org/queries/incremental-compilation.html}}.
This incremental compilation can significantly
reduce the compilation time for large projects,
making it easier to develop and iterate on Rust code.

The query system also enables the Rust compiler to perform other optimizations,
such as memoization and caching of intermediate results.
For example, if a function's return value has been computed before,
the query system can return the cached result instead of recomputing it,
further reducing compilation time.

Another important design choice in \emph{rustc} is interning.
Interning is a technique to store strings and
other data structures in a memory-efficient way.
Instead of storing multiple copies of the same string or data structure,
the Rust compiler stores only one copy in a special allocator called an \emph{arena}.
References to values stored in the arena are passed around between different parts
of the compiler and they can be compared cheaply by comparing pointers.
This can reduce memory usage and speed up operations
that compare or manipulate strings and data structures.

\emph{rustc} uses the LLVM compiler infrastructure\footnote{\url{https://llvm.org/}}
to perform low-level code generation and optimization.
LLVM provides a flexible framework for compiling code
to a variety of targets, including native machine code and \acrfull{WASM}.
The Rust compiler uses LLVM to optimize code for performance and
to generate high-quality code for a variety of platforms.
Instead of generating machine code, it only needs
to generate the source code's LLVM \acrfull{IR}
and then instruct LLVM to transform this to the compilation target,
applying the desired optimizations.

\emph{rustc} is programmed in Rust.
In order to compile the newer version of the compiler
and the newer standard library version that goes with it,
a slightly older version of \emph{rustc} and the standard library is used.
This process is called \emph{bootstrapping} and it implies
that one of the major users of Rust is the Rust compiler itself.
Considering that a new stable version is released every six weeks,
bootstrapping involves a substantial amount of complexity and is described in detail
in the documentation\footnote{\url{https://rustc-dev-guide.rust-lang.org/building/bootstrapping.html}}
and in conferences \cite{nelson2022} and tutorials \cite{klock2022} by members of the Rust team.

\subsection{Compilation stages}

The existence of the query system does not imply
that \emph{rustc} does not have compilation phases at all.
On the contrary, several stages of compilation are required
to transform Rust source code into machine code that can be executed on a computer.
These stages involve multiple intermediate representations of the program,
each one optimized for a specific purpose.
We will now briefly describe these stages.
A more complete overview is found
in the documentation\footnote{\url{https://rustc-dev-guide.rust-lang.org/overview.html}}.

\subsubsection{Lexing and parsing}

First, the raw Rust source text is analyzed by a low-level lexer.
At this stage, the source text is turned
into a stream of atomic source code units known as tokens.

Then parsing takes place.
The stream of tokens is converted into an \acrshort{AST}.
Interning of string values occurs here.
Macro expansion, \acrshort{AST} validation,
name resolution, and early linting also take place during this stage.
The resulting intermediate representation from this step is thus the \acrshort{AST}.

\subsubsection{HIR lowering}

Next, the \acrshort{AST} is converted to \acrfull{HIR}.
This process is known as ``lowering''.
This representation looks like Rust code but with complex constructs desugared
to simpler versions.
For instance, all \texttt{while} and \texttt{for} loops are converted into simpler
\texttt{loop} loops.

The \acrshort{HIR} is used to perform some important steps:

\begin{enumerate}
      \item \emph{type inference}: The automatic detection of a type
            of an expression, e.g. when declaring variables with \texttt{let}.
      \item \emph{trait solving}: Ensuring that each implementation block (\texttt{impl})
            refers to a valid, existing trait.
      \item \emph{type checking}: This process converts the types written by the user
            into the internal representation used by the compiler.
            It is, in other words, where types are interned.
            Then, using this information, the type safety, correctness, and coherence are verified.
\end{enumerate}

\subsubsection{MIR lowering}

In this stage, the \acrshort{HIR} is lowered to \acrfull{MIR},
which is used for \emph{borrow checking}.
As part of the process, the \acrfull{THIR} is constructed,
which is a representation that is easier to convert to \acrshort{MIR} than the \acrshort{HIR}.

The \acrshort{THIR} is an even more desugared version of \acrshort{HIR}.
It is used for pattern and exhaustiveness matching.
It is similar to the \acrshort{HIR}, but with all types and method calls made explicit.
Furthermore, implicit dereferences are included where needed.

Many optimizations are performed on the \acrshort{MIR}
as it is still a very generic representation.
Optimizations are in some cases easier
to perform on the \acrshort{MIR} than on the subsequent LLVM \acrshort{IR}.

\subsubsection{Code generation}

This is the last stage when producing a binary.
It includes the call to LLVM for code generation
and the corresponding optimizations.
To this effect, the \acrshort{MIR} is converted to LLVM \acrshort{IR}.

LLVM \acrshort{IR} is the standard form of input for the LLVM compiler
that all compilers using LLVM, such as the \emph{clang} C compiler, utilize.
It is a type of assembly language that is well-annotated
and designed to be easy for other compilers to produce.
Additionally, it is designed to be rich enough to enable LLVM
to perform several optimizations on it.

LLVM transforms the LLVM \acrshort{IR} to machine code
and applies many more optimizations.
Finally, the object files with assembly code in them may be
linked together to form the binary.

\subsection{Rust nightly}
\label{sec:rust-nightly}

Understanding the release model of Rust is crucial
for the successful implementation of the tool proposed in this work.
The reason is that to use the crates of \emph{rustc} as a dependency in our project,
it must be compiled with the \emph{nightly} version.

The nightly Rust compiler refers to a specific build of \emph{rustc}
that is updated every night with the latest changes and improvements
but also includes experimental or unstable features that are not yet part of the stable release.
In Rust, the language and its standard library are versioned using a ``release train'' model,
where there are three main release channels:
stable, beta, and nightly\footnote{\url{https://forge.rust-lang.org/}}.

The stable release of the Rust compiler is
the most widely used and recommended version for production use.
It goes through a rigorous testing and stabilization process
to ensure that it provides a stable and reliable experience for developers.
The stable release only includes features and improvements
that have been thoroughly reviewed, tested, and deemed stable enough for production use.

On the other hand, the nightly Rust compiler is the most bleeding-edge version,
where new features, bug fixes, and experimental changes are introduced on a daily basis.
It is used by Rust language developers and contributors for testing and development purposes
but it is not recommended for production use
due to the potential instability and lack of long-term support.

Each feature exclusive to the nightly version is behind a so-called \emph{feature flag}.
They may only be used when compiling with the nightly toolchain.
Features flags may enable

\begin{itemize}
      \item syntactic constructs that are not available on the stable version,
      \item library functions exclusive to the nightly version,
      \item support for specific hardware instructions of a given \acrshort{ISA} or platform,
      \item additional compiler flags.
\end{itemize}

The full list of feature flags is found in \cite{unstable-book}
and contains more than 500 entries in total.
In a more concise manner, the Rust language used inside of \emph{rustc} is
a superset of the stable Rust language used outside of it.
These differences should be taken into account
when working on the compiler or building software that directly depends on the compiler.