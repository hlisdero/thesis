% !TeX root = ../../Thesis.tex
\documentclass[../../Thesis.tex]{subfiles}
\graphicspath{{\subfix{../../images/}}}

\begin{document}

\section{In search of a backend}

To put it succinctly, two approaches for translating Rust code to Petri nets exist.
The first option is to create a translator from scratch,
while the second option is to build upon an existing tool.

The first option may seem attractive at first,
considering that it gives the developer
the freedom to shape the tool according to his/her desires.
Features can be added as required and
data structures can be tailored to the specific purpose.
Nonetheless, this flexibility comes at a high price.
In order to support a reasonable subset of the Rust programming language,
substantial amounts of effort need to be invested into the task.
Complex language constructs, such as macros, generics, or the rich type system itself,
must be comprehended in their most intricate details to be translated effectively.
The result is, essentially, a new compiler for Rust code.
Noting that the Rust compiler was developed over many years and
with the support of a large community of contributors,
it becomes clear that this path is nothing more than work duplication.
It is indeed a Herculean labor that would require the full-time dedication of a whole team
to maintain and keep up-to-date with the newest changes in the Rust language and the compiler.

On the other hand, there is the possibility
of integrating with the existing Rust compiler,
which is available under an open-source license
and its documentation is extensive and regularly updated.
This frees the implementation partly from having to deal with the changes to the language,
giving more time to focus on the features that add value to the users.
Hence the compiler plays the role of a backend on which the static analysis relies.
Of course, this requires learning the compiler internals
but this is not the first time that a tool sets off to do this.
For instance, the official Rust linter,
\emph{clippy}\footnote{\url{https://github.com/rust-lang/rust-clippy}},
analyzes the Rust code for incorrect, inefficient, or non-idiomatic constructs.
It is an extremely valuable tool for developers that goes beyond the standard checks
performed during compilation.

Supporting all the language features from the beginning
and collaborating with the community is key to the success of the proposed solution.
Therefore, it is advisable to integrate with the existing ecosystem
and reuse as much work as possible.
Due to the above reasons, this project is based on \emph{rustc}.
We will now study the relevant parts of the Rust compiler in more detail.

\end{document}