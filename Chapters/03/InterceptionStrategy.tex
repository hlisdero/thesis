% !TeX root = ../../Thesis.tex
\documentclass[../../Thesis.tex]{subfiles}
\graphicspath{{\subfix{../../images/}}}

\begin{document}

\section[Interception strategy]
 {Interception strategy:\\Selecting a suitable starting point for the translation}
\label{sec:interception-strategy}

In this section, a rationale for selecting the \acrfull{MIR}
as the starting point for the translation to a Petri net is elucidated.
This architectural design choice is justified for several reasons.

\subsection{Benefits}

First, the \acrshort{MIR} is
the lowest machine-independent \acrshort{IR} used in \emph{rustc}.
It captures the semantics of Rust code
after it has undergone a series of optimization passes
without relying on the details of any particular machine.
By intercepting the translation at this stage,
the static analysis tool leverages the benefits of these optimizations,
such as constant folding, dead code elimination, and inlining,
which results in a more efficient and overall smaller Petri net representation.

Second, intercepting the compilation after the previous stages are completed
offers an advantage in terms of efficiency and code reuse.
By this stage, the Rust compiler has already performed crucial steps
such as borrow checking, type checking,
monomorphization of generic code, and macro expansion, among others.
These steps are resource-intensive
and involve complex analysis of the Rust code to ensure correctness and safety.
Re-implementing these steps in our tool from scratch
would be redundant and time-consuming.
It would require duplicating the efforts of the Rust compiler
and may introduce potential inconsistencies or errors.
By building on top of the existing \acrshort{MIR},
we take advantage of the work already done by \emph{rustc}.
This not only saves effort but also ensures that our static analysis tool
is aligned with the same level of correctness and safety as the Rust compiler.

Third, it simplifies the maintenance task of staying up-to-date
with the ongoing additions to the Rust language and its compiler.
Rust is a fast-evolving language and its compiler is
constantly updated with new features, bug fixes, and performance optimizations.
Repurposing the \acrshort{MIR} means our tool can benefit from these updates
without having to independently implement and maintain those changes.
This provides overall a more robust and reliable static analysis solution.

Furthermore, as it will be explained in the next section, the \acrshort{MIR}
is based on the concept of a \acrfull{CFG}, i.e. a type of graph found in compilers.
That means that \acrshort{MIR} and Petri nets are both graph representations,
which makes the \acrshort{MIR} particularly amenable to translation.
Both \acrshort{MIR} and Petri nets can be seen as graphical models
that capture the relationships and interactions between different entities.
The \acrshort{MIR} graph represents the underlying execution flow within a Rust program,
while a Petri net captures the state transitions and event occurrences in a system.
Therefore, it becomes easier to convert the \acrshort{MIR} to a Petri net,
as the graph structure and relationships are already present.
This allows for a more straightforward and efficient translation process
without having to create a graph structure from thin air,
resulting in better integration between the \acrshort{MIR}
and the Petri net model for deadlock detection.

Finally, working with the \acrshort{MIR} synergizes
with incremental compilation and modular analysis.
In fact, one of the reasons why \acrshort{MIR} was introduced
in the first place was incremental compilation \cite{matsakis2016mir}.
Even though it is not mandatory in the initial implementation,
the tool could profit from incremental compilation
and perform analysis on a per-crate/per-module basis,
allowing for faster and more efficient analysis of large Rust codebases.

\subsection{Limitations}

There are however some limitations to the approach of basing the translation
on the \acrfull{MIR}.

The most important one is that the \acrshort{MIR} is subject to change.
No stability guarantees are made in terms of
how the Rust code will be translated to \acrshort{MIR}
or which its constituent elements \acrshort{MIR} are.
These are internal details that the compiler developers reserve for themselves.
In short, \acrshort{MIR} as an interface is not stable.
As work on the compiler continues,
the \acrshort{MIR} undergoes modifications
to incorporate new language features, optimizations, or bug fixes,
which may require frequent updates and adjustments to the translation process,
increasing the maintenance cost.

In the course of this project, this situation occurred numerous times.
As an example, in the period between mid-February 2023 and mid-April 2023,
the code was modified 7 times to accommodate these changes.
They were always a few lines of code in size and detected by tests.
We will discuss how tests play a significant role
in coping with these changes in Sec. \ref{sec:integration-tests}.

On the same note, \cite{meyer2020} also relied on the \acrshort{MIR}
but did not incorporate tests to deal with the newer nightly versions.
As a result, the toolchain was pinned to an exact nightly
version\footnote{\url{https://github.com/Skasselbard/Granite/blob/master/rust-toolchain}}
to prevent the implementation from breaking before the publication of the thesis.

Another drawback worth mentioning is that, in some instances,
generic code could take the form of a function whose behavior can be modeled
by the same Petri net in all cases.
Under these circumstances, the \acrshort{MIR} could be ``condensed'' further
before translating it to a Petri net.
Similarly, some parts of the \acrshort{MIR} may be superfluous
to the deadlock detection analysis and its translation may enlarge the output,
which slows down the reachability analysis done by the model checker.
This can be countered with careful optimizations,
which will be proposed in Chapter \ref{chap:future-work}.

\subsection{Synthesis}

In conclusion, despite the drawbacks mentioned earlier,
intercepting the translation at the \acrshort{MIR} level offers significant advantages,
including maximizing the utilization of the existing compiler code,
reducing the implementation effort, and a more natural mapping to Petri nets.
These benefits outweigh the cons and make the \acrshort{MIR}
a compelling starting point for the translation
in the context of building a static analysis tool
for detecting deadlocks and missed signals in Rust code.

Both \cite{meyer2020} and \cite{zhang2022deadlocks} base their translations
on the \acrshort{MIR} as well and to the best knowledge of this author, there is
no analogous tool that performed a translation to Petri nets
starting from a higher-level \acrshort{IR}.

\end{document}