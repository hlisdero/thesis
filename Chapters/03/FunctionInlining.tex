\section{Function inlining in the translation to Petri nets}
\label{sec:function-inlining}

In this section, a thorough analysis and motivation for the third design decision
listed at the beginning of the chapter, namely inlining function calls, is presented.

Modeling functions in \acrshort{PN} is a crucial aspect of the translation
because it is the basic unit of the MIR.
By representing the functions in the MIR as \acrshort{PN} and connecting them accordingly,
the control flow and data shared between the threads in the program
can be captured in a formal framework.
Afterward, the Petri net is analyzed by a model checker
in order to identify potential deadlocks or lost signals.
This approach is especially useful when working with large and complex systems
that may have many interrelated threads and functions,
where the deadlock situation may not be evident even to an experienced code reviewer.

When translating MIR functions to \acrshort{PN}, one key question that arises is
whether to reuse the same representation for every call to a specific function or
to ``inline'' the corresponding representation every time the function is called.
Expressed differently, each function maps to a subnet
in the final \acrshort{PN} obtained after the translation, i.e.
a connected subgraph formed by the places and transitions
that model the behavior of the specific function.
This smaller part of the net can either be present only once in the \acrshort{PN}
and all calls to this function connect to it,
or be repeated for every instance of a call to the function in the Rust code.

Reusing the same model for every function seems at first glance more efficient,
as the \acrshort{PN} obtained is smaller.
However, this approach can also lead to invalid states
that were not present in the original Rust program.
These can be the source of false positives during deadlock detection,
as these extraneous states may violate
the safety guarantees offered by the compiler.

On the other hand, inlining the model every time a function is called results in
a larger \acrshort{PN}, which requires more memory and \acrshort{CPU} time to be analyzed,
but it can also improve the accuracy of the analysis by ensuring
that each function call is represented by a separate Petri net structure
that captures its specific data dependencies in the context
in which the function call occurs in the code.

\subsection{The basic case}

The impact of these subtle details can only
be fully comprehended with an appropriate example.
Therefore, consider first the most simple abstraction of a function call in
the language of Petri nets, formed by a single transition and
two places representing the start and end of the function.
This is seen in Fig. \ref{fig:simplest-function}.
The function call is treated as a black box,
all details are abstracted away in the transition.
We care only about where the function starts and where it ends.

\begin{figure}[!htb]
    \centering
    \includesvg[width=0.3\linewidth]{simplest-function.svg}
    \caption{The simplest Petri net model for a function call.}
    \label{fig:simplest-function}
\end{figure}

Observe now such a function in the context of a Rust program.
Listing \ref{lst:repeated-function-call} provides a simple example
in which one function is called
five times consecutively in a \Rustinline{for} loop.
A possible \acrshort{PN} that models the program
is found in Fig. \ref{fig:repeated-function-call}.
It should be emphasized that
this net and the subsequent ones in this section
do \emph{not} result from a translation of the MIR.
They are simplifications to showcase the difficulties
of dealing with functions called in various places in the code.

\begin{listing}
    \begin{minted}{Rust}
        fn simple_function() {}

        pub fn main() {
            for n in 0..5 {
                simple_function();
            }
        }
    \end{minted}
    \caption{A simple Rust program with a repeated function call.}
    \label{lst:repeated-function-call}
\end{listing}

\begin{figure}[!htb]
    \centering
    \includesvg[width=0.7\linewidth]{repeated-function-call.svg}
    \caption{A possible \acrshort{PN} for the code in Listing \ref{lst:repeated-function-call}
        applying the model of Fig. \ref{fig:simplest-function}.}
    \label{fig:repeated-function-call}
\end{figure}

\subsection{A characterization of the problem}

The troublesome scenario has not emerged so far.
It manifests only when a function is called
in at least two different places in the code or,
in simpler terms, the expression \Rustinline{simple_function()} appears twice or more.
Listing \ref{lst:two-simple-function-calls} satisfies this condition
and is designed to exhibit
the extraneous behavior described at the beginning of the section.

\begin{listing}
    \begin{minted}{Rust}
        fn simple_function() {}

        pub fn main() {
            let mut second_call = false;
            simple_function();
            if second_call {
                panic!()
            }
            second_call = true;
            simple_function();
        }
    \end{minted}
    \caption{A simple Rust program that calls a function in two different places.}
    \label{lst:two-simple-function-calls}
\end{listing}

As stated before, the first approach to modeling the program consists
in reusing the function model for both calls.
This is shown in Fig. \ref{fig:two-function-calls-incorrect-1}.

\begin{figure}[!htbp]
    \centering
    \includesvg[width=0.88\linewidth]{two-function-calls-incorrect-1.svg}
    \caption{A first (incorrect) \acrshort{PN} for the code
        in Listing \ref{lst:two-simple-function-calls}.}
    \label{fig:two-function-calls-incorrect-1}
\end{figure}

It is evident to the reader that
the program in Listing \ref{lst:two-simple-function-calls}
never calls the \Rustinline{panic!()} macro and always terminates successfully,
given that the variable \Rustinline{second_call}
is never \Rustinline{true} before line 9.

Yet, the \acrshort{PN} depicted in Fig. \ref{fig:two-function-calls-incorrect-1}.
is conspicuously flawed, making it unsuitable as a model for the program.
The reason is that after firing the transition labeled \texttt{RETURN\_simple\_function}
a token is placed in \texttt{check\_flag} but \emph{also} in \texttt{main\_end\_place}.
The token in \texttt{main\_end\_place} will eventually appear in \texttt{PROGRAM\_END},
which indicates a normal termination of the program.
This is technically correct since we know that the program terminates successfully.

Nonetheless, there are concerning issues regarding the second token.
The token in \texttt{check\_flag} could be consumed either
by the transition \texttt{flag\_is\_false} or \texttt{flag\_is\_true}.
If it is consumed by the latter, a token will be placed in \texttt{PROGRAM\_PANIC},
signaling an erroneous termination of the program.
This is absurd because it means that the program could panic
but also \emph{always} ends normally, as seen in the previous paragraph.

The situation becomes worse if we follow the path of firing \texttt{flag\_is\_false}.
In that case, the token triggers another function call, which is in principle correct,
but nothing prevents it from doing this over and over again.
The conclusion is that an infinite amount of tokens could accumulate
in \texttt{main\_end\_place} or \texttt{PROGRAM\_END}
in the circumstance that, by pure chance,
the transition \texttt{flag\_is\_true} does not fire.

It has become clear that we must discard this model and look for a better solution.
One possibility is to split the transition labeled \texttt{RETURN\_simple\_function}
in two separate transitions depending on the function call order
as illustrated in Fig. \ref{fig:two-function-calls-incorrect-2}.

\begin{figure}[!htbp]
    \centering
    \includesvg[width=0.94\linewidth]{two-function-calls-incorrect-2.svg}
    \caption{A second (also incorrect) \acrshort{PN} for the code
        in Listing \ref{lst:two-simple-function-calls}.}
    \label{fig:two-function-calls-incorrect-2}
\end{figure}

This second attempt unfortunately comes with its own set of extraneous states.
First, the program may now exit after calling the function only once.
Nothing prevents the transition \texttt{RETURN\_simple\_function\_2} from firing first.
This is equivalent to saying that the execution flow jumps
from line 5 to line 11 in Listing \ref{lst:two-simple-function-calls},
which is obviously not a property present in the original Rust code.

On the other hand, the problem of the infinite loop persists.
The \acrshort{PN} may continue firing indefinitely as long as
\texttt{flag\_is\_true} and \texttt{RETURN\_simple\_function\_2} do not fire.
There is no guarantee that the transitions fire in a specific order.
As seen in Sec. \ref{sec:transition-firing},
the transition firing is non-deterministic.

\clearpage
\subsection{A feasible solution}

Having observed the difficulties of modeling function calls,
we turn our attention to the other approach to modeling function calls:
Inlining the \acrshort{PN} representation.
Some of the lessons learned from the preceding subsection are:

\begin{itemize}
    \item Creating a loop in the net where there is no loop in the original program
          opens the door to infinite sequences of transition firings.
          This could in turn break the \emph{safety} property of the \acrshort{PN}.
    \item As the token symbolizes the program counter,
          there must be only one token in the \acrshort{PN} at any given time.
    \item The program state may change between function calls.
          Accordingly, separate places should model these states.
          Put differently, the state when calling a function the first time
          may not be the same as when calling the function a second time.
\end{itemize}

Fig. \ref{fig:two-function-calls-correct} introduces
the inlining approach implemented in the tool.
The \acrshort{PN} therein is correct.
It matches the structure of the Rust code more closely.
It does not contain any loops nor
it creates additional tokens when firing transitions,
i.e. none of the transitions has two outputs.
It is worth mentioning that the resulting \acrshort{PN} is a state machine
(Definition \ref{definition:state-machines})
as expected for a single-threaded program.
This was not the case for Fig. \ref{fig:two-function-calls-incorrect-1}
and \ref{fig:two-function-calls-incorrect-2}.

\begin{figure}[!htbp]
    \centering
    \includesvg[width=\linewidth]{two-function-calls-correct.svg}
    \caption{A correct \acrshort{PN} for the code
        in Listing \ref{lst:two-simple-function-calls} using inlining.}
    \label{fig:two-function-calls-correct}
\end{figure}

A significant advantage of the inlining approach is
that every function call is unequivocally identified.
This proves helpful when interpreting the output of the model checker or
error messages during the translation of a given program.
The use of an incremental non-negative id is arbitrary but convenient.
Moreover, the accuracy of deadlock detection is increased because
certain classes of extraneous states
such as those in the \acrshort{PN} shown in the previous section
are not present.
Minimizing the number of false positives plays an important role
when considering which approach to implement
for a tool that aims to be user-friendly and easy to set up.

One disadvantage mentioned earlier is that the size of the resulting net is larger.
The exact penalty in the number of additional places and transitions depends
on the frequency with which functions are reused on average in the codebase.
It is reasonable to assume that functions are called from several places.
However, certain optimizations can be applied,
which can reduce the size of the net considerably,
thus compensating for the effect of using inlining.
These optimizations are discussed in detail in Chapter \ref{chap:future-work}.

Lastly, an attentive reader may notice that
the analysis of the \acrshort{PN} in Fig. \ref{fig:two-function-calls-correct}
leads to the conclusion that
the program may call \Rustinline{panic!()} and terminate abruptly,
which does not match the execution of the Rust program.
This is correct but it is a limitation of low-level Petri nets
that cannot be solved in the framework of the model and
goes beyond the scope of this work.
Chapter \ref{chap:future-work} explores
the consequences of this restriction and proposes potential remedies.

Armed with new insights and knowledge about the design choices,
we are now able to fully describe the implementation.