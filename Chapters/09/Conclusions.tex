This thesis has explored the translation of Rust programs into Petri net models
for the purpose of deadlock and missed signal detection.
Throughout the study, various aspects of the translation process have been examined,
including the handling of function calls, threads, mutexes, and condition variables.
The translator we developed has demonstrated its capability
to accurately capture the concurrency and
synchronization behavior of rather simple Rust programs.

The translation approach presented in this thesis has shown promising results,
successfully modeling and detecting deadlocks in a range of test programs,
comprising even two classical problems of concurrent programming.
By harnessing the expressive power of Petri nets,
the translator provides a visual representation of program behavior,
facilitating the identification of potential synchronization issues.
Most importantly, the translation produces a model that can be analyzed
by a myriad of model checking tools,
leveraging the existing academic work to bring solutions to industry problems.
The incorporation of a succinct model for condition variables
enhances the modeling capabilities and enables the detection of missed signals,
which are a more intricate class of deadlock in concurrent systems.

Moving forward, there are several avenues for future research and improvement.
One potential direction is the exploration of more complex programs and real-world applications
to evaluate the scalability and effectiveness of the translation approach.
Additionally, further refinement and optimization of the translation algorithms
could enhance the efficiency of the analysis,
specially higher-level models that would allow modeling the memory more effectively.

Overall, this thesis has made a significant contribution by developing a translator
that bridges the gap between Rust programs and Petri nets.
The insights gained from this research have shed light on the challenges
and opportunities in modeling and analyzing concurrent systems at compile-time.
Ideally, a programming language whose compiler detects concurrency problems
would be a godsend for many applications.
Building on the strengths of Petri nets, this possibility could be advanced
further in the Rust programming language.

On a different note, the contribution of this thesis extends
beyond the immediate benefits of the proposed translator and its capabilities.
By providing a solid, well-documented base for the translation of Rust programs into Petri nets,
this work aims to make a meaningful contribution to the Rust community as a whole.
It serves as a stepping stone for future endeavors,
offering a reliable foundation upon which other tools and research projects can be built.
It opens up new possibilities for exploring the analysis and verification
of concurrent Rust programs using Petri nets.
This, in turn, has the potential to drive further advancements in the field,
stimulating innovation and promoting a deeper understanding of concurrent programming in Rust.
With its comprehensive documentation and clear implementation,
the translator not only facilitates immediate use but also serves as a valuable resource
for those interested in studying or extending the translation techniques employed.
Ultimately, this work aspires to ignite curiosity and
inspire further contributions to the Rust ecosystem,
fostering collaboration and growth in the community.