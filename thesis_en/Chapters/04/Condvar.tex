\section{Condition variable (\texttt{std::sync::Condvar})}

A condition variable is a synchronization primitive used in concurrent programming
to enable threads to wait for a certain condition before proceeding with their execution.
Threads wait until they are notified by another thread
that the desired condition has been met.

Condition variables are typically associated with a mutex,
which ensures exclusive access to the shared data that the condition depends on.
When a thread waits on a condition variable, it releases the associated mutex,
allowing other threads to make progress.
When the condition becomes true or some event occurs,
a notifying thread signals the condition variable,
allowing one or more waiting threads to resume their execution.

The semantics of condition variables, as well as examples in pseudocode,
were introduced in Sec. \ref{sec:condition-variables}.
The understanding of the precise behavior of condition variables in all circumstances
is a prerequisite for this section.

This section provides an elaborate explanation of the \acrshort{PN} model
used to represent condition variables from the Rust standard library.
It is followed by a practical example that aims to enhance the clarity of the concepts.
Finally, the algorithms for the translation of condition variable functions are outlined.

\subsection{Petri net model}

In this particular case, the \acrshort{PN} model must be examined carefully,
as it involves not only the condition variable itself
but also the variable that holds the condition
on which the blocked thread is waiting \emph{and} the mutex
that synchronizes access to that condition.

This interaction can be extremely complex in general.
For instance, the same condition variable could be used
to signal an arbitrary number of distinct conditions.
Accordingly, different mutexes may be passed
as an argument to the \Rustinline{wait} call.
Furthermore, an arbitrary number of threads may block on a condition variable and
Rust supports the broadcast operation to awaken all waiting threads at once
through the method \Rustinline{notify_all}\footnote{\url{https://doc.rust-lang.org/std/sync/struct.Condvar.html\#method.notify_all}}
(see Sec. \ref{sec:condition-variables}).
Most importantly, the condition itself could be of any type
and could take a long sequence of values during the execution,
depending on which the waiting threads could act in diverse ways for every scenario.

For the above reasons, it is unavoidable to make assumptions
regarding the supported use cases of condition variables
in order to reduce the complexity of the task.
Embracing and dealing with every possibility is beyond the scope of this thesis.

\subsubsection{Assumptions}

\begin{enumerate}
      \item \emph{Single call}: There is only one call to \Rustinline{wait} per condition variable, i.e.,
            \Rustinline{condvar.wait()} appears in a single place in the source code for a given \Rustinline{condvar}.
            For example, it can be inside of a loop but it cannot be in two different functions.
      \item \emph{Single-element queue}: There is at most one waiting thread per condition variable.
      \item \emph{Boolean condition}: The condition is a boolean flag. It is either set or not set.
            Waiting on a condition that may take 3 or more values is \emph{not} supported by this model.
      \item \emph{Mandatory set-condition / No ``false notify''}: If a thread locks the mutex and accesses the shared condition mutably,
            then it always sets it to a different value.
            In simpler terms, threads that look at the value, do not change it
            and immediately notify the condition variable are \emph{not} supported.
      \item \emph{Broadcast exclusion}: The method \Rustinline{std::sync::Condvar::notify_all} is out of scope.
\end{enumerate}

Support for multiple calls to \Rustinline{wait} and multiple waiting threads could be implemented
but a considerable implementation effort is required.
Therefore, assumptions 1 and 2 can be overcome with the proposed model.

Supporting non-boolean conditions
and detecting which value is set
necessitates a thorough reconsideration of the modeling approach
for representing concrete data values in simple Petri nets.
Consequently, assumptions 3 and 4 are particularly challenging and
could be the subject of future research into higher-level models.
See Sec. \ref{sec:future-work-higher-level-models} for some thoughts to this effect.

\subsubsection{Analysis of the proposed model}

Fig. \ref{fig:condition-variable-model} depicts
the \acrshort{PN} model used in the implementation.
The same diagram in DOT, PNG, and SVG format
can be found in the repository as documentation.

\begin{figure}[!htbp]
      \centering
      \includesvg[width=\linewidth]{condition-variable-model.svg}
      \caption{The Petri net model for condition variables.}
      \label{fig:condition-variable-model}
\end{figure}

The input places are:

\begin{itemize}
      \item \Rustinline{input}: The start place of the wait function.
            The model supports the methods from the standard library \Rustinline{std::sync::Condvar::wait}
            and its variation \Rustinline{std::sync::Condvar::wait_while}.
      \item \Rustinline{condition_not_set}: The place is marked when the condition is \Rustinline{false}.
      \item \Rustinline{condition_set}: The place is marked when the condition is \Rustinline{true}.
      \item \Rustinline{notify}: The place where the notifying thread places a token to wake up the waiting thread.
\end{itemize}

The \Rustinline{output} place is the end place
of the function call to \Rustinline{wait} or \Rustinline{wait_while}.
The execution of the thread continues from there.

Two possible ways exist to go
from \Rustinline{input} to \Rustinline{output}, represented by two transitions:

\begin{itemize}
      \item \Rustinline{wait_start}: This is the ``common case'', the thread blocks and waits for the signal.
      \item \Rustinline{wait_skip}: This is the alternative path that the token takes when the condition was already set.
            The thread does not wait, instead, it skips the wait and reaches \Rustinline{output} in a single jump.
\end{itemize}

It is essential to notice that the greyed-out part
on the left of Fig. \ref{fig:condition-variable-model}
controls which transition is enabled and which transition is disabled.
As soon as a token is set in \Rustinline{condition_set}, \Rustinline{wait_start} is disabled.
Before that, the opposite is true: \Rustinline{wait_start} may fire but \Rustinline{wait_skip} may not.

Note the arcs between \Rustinline{condition_not_set} and \Rustinline{wait_start}.
The token is regenerated every time that \Rustinline{wait_start} fires.
The same is true for \Rustinline{condition_set} and \Rustinline{wait_skip}.
These arcs restore the condition places to their previous state.
Modeling more than 2 values as a \acrshort{PN} would entail a more convoluted net,
Besides, it would be impossible to know at compile time,
how many possible values the condition takes to generate the correct number of places.
This limitation is the justification for Assumptions 3 and 4.

Now focus on the right side of Fig. \ref{fig:condition-variable-model}.
In the middle of the condition variable, we find the place for the mutex.
As expected, it is unlocked when the wait starts
and it is locked when the notify is received.

The place labeled \Rustinline{wait_enabled} plays an important role.
On the one hand, it consumes the token from \Rustinline{notify}
if \Rustinline{wait_start} was not fired.
This is the archetypical missed signal case that we would like to detect.
On the other hand, the token in \Rustinline{wait_enabled}
is consumed when \Rustinline{wait_start} fires,
This prevents the condition variable from ``accepting'' other threads (Assumption 2)
and preserves the token in \Rustinline{notify}, ensuring that the missed signal cannot occur.

Finally, the \Rustinline{notify_received} transition combines the requisites
for the thread to leave the wait: The mutex must be unlocked and \Rustinline{notify_one} was called.
To restore the initial state of the condition variable,
it regenerates the token in \Rustinline{wait_enabled}.

\subsubsection{Global translation requirements of the Petri net model}

A fundamental challenge that surfaces
during the implementation of the model in Fig. \ref{fig:condition-variable-model}
is that connections across the blue frontier in the diagram cannot be established
in general when processing the call to \Rustinline{wait}.
We will analyze the problem and explain how the solution copes with it.

The transition where the condition is set, named \Rustinline{set_condition} on the diagram,
is the next candidate.
In the current implementation, the transition selected to fulfill this role
is the call to \Rustinline{std::ops::DerefMut::deref_mut}
when a mutex or mutex guard is being dereferenced.

Consider Listing \ref{lst:set-condition-example},
yet another test program from the repository.
In line 9, the mutex guard is dereferenced to write the value \Rustinline{true} to it,
setting the condition for the condition variable.
In the \acrshort{MIR}, this maps to a call to \Rustinline{std::ops::DerefMut::deref_mut}.
Although it is not the exact place where the value is written
(which would actually be a statement in \acrshort{BB}),
it is close enough and it satisfies our needs.

\begin{listing}[!htbp]
      \begin{minted}{Rust}
      fn main() {
            let pair = std::sync::Arc::new((std::sync::Mutex::new(false), std::sync::Condvar::new()));
            let pair2 = std::sync::Arc::clone(&pair);
        
            // Inside of our lock, spawn a new thread, and then wait for it to start.
            std::thread::spawn(move || {
                let (lock, cvar) = &*pair2;
                let mut started = lock.lock().unwrap();
                *started = true;
                // We notify the condvar that the value has changed.
                cvar.notify_one();
            });
        
            // Wait for the thread to start up.
            let (lock, cvar) = &*pair;
            let mut started = lock.lock().unwrap();
            while !*started {
                started = cvar.wait(started).unwrap();
            }
      }              
      \end{minted}
      \caption{A program that requires global Petri net information to be translated.}
      \label{lst:set-condition-example}
\end{listing}

Unfortunately, the connections to the condition variable cannot be established
when processing the call to \Rustinline{deref_mut} either.
The reason is that there is no guarantee that the condition variable was already seen.
It could still be ahead in the translation path.
In Listing \ref{lst:set-condition-example}, the main thread is translated first,
so the condition variable is discovered first.
But if the roles of the threads are swapped, then the translation cannot be performed.

We reach thus an unpleasing conclusion.
In order to connect the model of the condition variable
to the places that model the condition and the transitions where it is set,
we need the whole \acrshort{PN}, i.e., we need \emph{global} information
to translate the synchronization primitive effectively.

As a result, it is unavoidable to incorporate some sort of postprocessing step
to the translation.
The tasks must also be performed in a specific order.
The mutex must have been discovered first.
Later it may be linked to a condition variable
if such a condition variable is found (the source code may as well not use any).
Hence, it is advisable to introduce a notion of ``priority'' to the postprocessing tasks.

The \Rustinline{Translator} relies on a \Rustinline{std::collections::BinaryHeap} to
implement a priority queue of
\Rustinline{PostprocessingTask}\footnote{\url{https://github.com/hlisdero/cargo-check-deadlock/blob/main/src/translator/function.rs\#L92}}.
The tasks are returned by the methods that translate synchronization primitives if needed.
After all the threads are translated, the \Rustinline{Translator}
addresses the postprocessing tasks.
By completing them according to their priority,
we guarantee that the information is available in the required order.

\subsubsection{Table of possible inputs and expected outputs}

As a complement to the explanation in the previous subsections,
here is a table summarizing the expected output for a given input.
The reader may compare Fig. \ref{fig:condition-variable-model} to verify that
the model produces the correct output for each scenario.

\begin{table}[!htb]
      \centering
      \begin{tabular}{ |c|c|c|c|l| }
            \hline
            Row \# & \multicolumn{3}{|c|}{Input} & Output                                                                                                            \\
            \hline
                   & \Rustinline{condition_set}  & \Rustinline{wait_enabled} & \Rustinline{notify} & \emph{where the initial token at \Rustinline{input} ends}       \\
            \hline
            R1     & False                       & False                     & False               & \Rustinline{waiting} (waiting for a \Rustinline{notify})        \\
            R2     & False                       & False                     & True                & \Rustinline{output} (correct wait end condition)                \\
            R3     & False                       & True                      & False               & \Rustinline{input} (initial state)                              \\
            R4     & False                       & True                      & True                & \emph{lost signal (transient state, goes to R1)}                \\
            R5     & True                        & False                     & False               & \Rustinline{waiting} (condition set, needs \Rustinline{notify}) \\
            R6     & True                        & False                     & True                & \Rustinline{waiting} (correct wait end condition)               \\
            R7     & True                        & True                      & False               & \Rustinline{output} (skip the wait)                             \\
            R8     & True                        & True                      & True                & \Rustinline{output} (skip the wait, with lost signal)           \\
            \hline
      \end{tabular}
      \caption{A summary of the possible states of the Petri net model for condition variables.}
      \label{table:condition-variable-model-states}
\end{table}

\subsection{A practical example}

Due to the size constraints of the resulting \acrshort{PN},
we are compelled to select a small-scale program for demonstration purposes.
It would be unfeasible to embed within a single page
the complete \acrshort{PN} of a realistic program with condition variables and multiple threads.
For more comprehensive examples, readers are encouraged to explore the repository,
which contains a collection of more intricate programs included as part of the integration tests.

Despite the space limitations, the example in Listing \ref{lst:condition-variable-example}
comprises the core elements of the model presented before.
The complete \acrshort{PN} can be seen in Fig. \ref{fig:condition-variable-example}.

Observe the following sequence of transitions:

\begin{enumerate}
      \item The mutex is locked in \Rustinline{std_sync_Mutex_T_lock_0_CALL}.
      \item \Rustinline{std_sync_Condvar_notify_one_0_CALL} sets a token in \Rustinline{CONDVAR_0_NOTIFY}.
      \item The token flow continues to \Rustinline{main_BB5} right before \Rustinline{CONDVAR_0_WAIT_START}.
\end{enumerate}

It should be emphasized that no deadlock arises
if the transition \Rustinline{CONDVAR_0_WAIT_START} fires
\emph{before} \Rustinline{CONDVAR_0_LOST_SIGNAL}.
In short, a conflict exists between \Rustinline{CONDVAR_0_WAIT_START}
and \Rustinline{CONDVAR_0_LOST_SIGNAL} for the token in \Rustinline{CONDVAR_0_NOTIFY}.
Nevertheless, the model checker verifies \emph{all} possible firings
and it will uncover the missed signal case without difficulties.

Another noteworthy observation is that this program illustrates
the effect that the cleanup paths would have on missed signal detection.
If there were a second transition
at the same level as \Rustinline{CONDVAR_0_WAIT_START} or \Rustinline{std_sync_Condvar_notify_one_0_CALL},
the token could ``escape'' to the \Rustinline{PROGRAM_PANIC} place and the deadlock would remain undetected.

It is indispensable for the translation to ``force'' the Petri net to stay blocked
and not open alternative paths that could be used by the model checker
to come to the conclusion that the \acrshort{PN} never deadlocks.

\begin{listing}[!htbp]
      \begin{minted}{Rust}
      fn main() {
            let mutex = std::sync::Mutex::new(false);
            let cvar = std::sync::Condvar::new();
            let mutex_guard = mutex.lock().unwrap();
            cvar.notify_one();
            let _result = cvar.wait(mutex_guard);
      }     
      \end{minted}
      \caption{A basic program to showcase condition variable translation.}
      \label{lst:condition-variable-example}
\end{listing}

\begin{figure}[!htbp]
      \centering
      \includesvg[width=\linewidth]{condition-variable-example.svg}
      \caption{The Petri net model for the program in Listing \ref{lst:condition-variable-example}.}
      \label{fig:condition-variable-example}
\end{figure}

\subsection{Algorithms for condition variable translation}
\label{sec:condvar-algorithms}

To wrap up this section, we will present a concise summary
of the algorithms utilized in the translation of condition variables.
The additions required to the mutex algorithms are included afterward as well.

When a call to \Rustinline{std::sync::Condvar::new} is encountered:

\begin{enumerate}
      \item Translate the function call using the model seen in Fig. \ref{fig:function-with-cleanup}.
      \item Create a new \Rustinline{Condvar}\footnote{\url{https://github.com/hlisdero/cargo-check-deadlock/blob/main/src/translator/sync/condvar.rs}}
            structure with an index to identify it unequivocally across the \acrshort{PN}.
      \item Link the return value of \Rustinline{std::sync::Condvar::new},
            the new condition variable, to the \Rustinline{Condvar} structure.
\end{enumerate}

When a call to \Rustinline{std::sync::Condvar::notify_one} is encountered:

\begin{enumerate}
      \item Translate the function call using the model seen in Fig. \ref{fig:simplest-function}.
            Ignore the cleanup place because, otherwise, any call may fail,
            which amounts to the notify operation not being present in the program,
            leading to a false lost signal.
            This is equivalent to assuming that the \Rustinline{notify_one} function never fails.
      \item Retrieve the \Rustinline{self} reference to the condition variable
            on which the function is called.
      \item Add an arc from the transition representing the function call
            to the \Rustinline{notify} place of the underlying condition variable.
\end{enumerate}

When a call to \Rustinline{std::sync::Condvar::wait}
or \Rustinline{std::sync::Condvar::wait_while} is encountered:

\begin{enumerate}
      \item Ignore the cleanup place because, otherwise, any call may fail,
            which amounts to the wait operation not being present in the program,
            leading to an incorrect result.
            We must force the \acrshort{PN}
            to ``wait'' for the notifying signal to be sent.
            This is equivalent to assuming that the \Rustinline{wait}
            or \Rustinline{wait_while} function never fails.
      \item Retrieve the \Rustinline{self} reference to the condition variable
            on which the function is called.
      \item Extract the mutex guard passed into the function.
      \item If the condition variable was already connected to a function call,
            then the translation fails.
            This enforces Assumptions 1 and 2 seen at the beginning of the section.
      \item Otherwise, connect the start and end places to the
            \Rustinline{wait_start} and \Rustinline{notify_received} transitions respectively.
      \item Link the return value, the same mutex guard that was passed as an argument,
            to the \Rustinline{MutexGuard} structure.
      \item Notify the translator that the mutex received must be linked to this \Rustinline{Condvar}.
            For this purpose, use the enum variant \Rustinline{PostprocessingTask::LinkMutexToCondvar}.
            This task will be processed after translating all the threads.
\end{enumerate}

When all the threads finished translating, that is,
when the queue of threads to process is empty,
the \Rustinline{Translator} enters a loop
to complete the postprocessing tasks by priority order:

\begin{enumerate}
      \item Create at the beginning of the loop an empty vector of mutex references.
      \item Pop from the \Rustinline{std::collections::BinaryHeap} the task with the lowest priority.
            This will be by design a \Rustinline{PostprocessingTask::NewMutex}.
            Add the mutex reference to the vector.
      \item After processing all the lower priority tasks,
            the \Rustinline{Translator} has references to all the mutexes in the code.
            Continue popping tasks from the priority queue.
      \item Eventually, a \Rustinline{PostprocessingTask::LinkMutexToCondvar} is extracted.
            Link each mutex to the condition variable,
            which creates the places \Rustinline{condition_set}
            and \Rustinline{condition_not_set} for the condition.
            It also connects the \Rustinline{deref_mut} transitions
            to these places to set the condition.
            Lastly, it connects the condition places to the condition variable transitions
            to disable the \Rustinline{wait}.
\end{enumerate}

\subsubsection{Modifications to the mutex algorithms}

As stated before, the mutex algorithms require some additions to successfully
perform missed signal detection.

Add the following to the handler of the \Rustinline{std::sync::Mutex::new} function:

\begin{enumerate}
      \item Notify the translator that a new mutex has been created.
            For this purpose, use the enum variant \Rustinline{PostprocessingTask::NewMutex}.
            This task will be processed after translating all the threads.
\end{enumerate}

When a call to \Rustinline{std::result::Result::<T, E>::unwrap} is encountered:

\begin{enumerate}
      \item Check that the \Rustinline{self} reference is a mutex or a mutex guard.
      \item Translate the function call using the model seen in Fig. \ref{fig:simplest-function}.
            Ignore the cleanup place because, otherwise, any call may fail,
            as if the mutex lock operation
            were not present in the program, leading to a false lost signal.
            This is equivalent to assuming that the \Rustinline{unwrap} function never fails
            when applied to a variable linked to a mutex or a mutex guard.
\end{enumerate}

When a call to \Rustinline{std::ops::Deref::deref}
or \Rustinline{std::ops::DerefMut::deref_mut} is encountered:

\begin{enumerate}
      \item Check that the \Rustinline{self} reference is a mutex or a mutex guard.
      \item Translate the function call using the model seen in Fig. \ref{fig:simplest-function}.
            Ignore the cleanup place because, otherwise, any call may fail,
            as if the mutex lock operation
            were not present in the program, leading to a false lost signal.
            This is equivalent to assuming that the \Rustinline{deref}
            and \Rustinline{deref_mut} functions never fail
            when dereferencing a variable linked to a mutex or a mutex guard.
      \item If the value is being dereferenced mutably (\Rustinline{deref_mut}),
            extract the first argument passed to the function: The mutex or mutex guard.
            Add the \Rustinline{deref_mut} transition to the mutex
            to set the condition for a condition variable in the postprocessing step.
      \item Otherwise do nothing.
            The immutable case does not need to be added to the mutex.
\end{enumerate}

It should now be clear to the reader that the algorithms for missed signal detection are
fundamentally of higher complexity and ought to handle more border cases than
those for detecting simple deadlocks caused by incorrect usage of mutexes or
calling \Rustinline{join} on threads that never terminate.

It is worth mentioning that some border cases arise
due to the inclusion of the cleanup logic
from the \acrshort{MIR} in the \acrshort{PN} model.
If the implementation instead skipped this,
under the hypothesis that Rust standard library functions may never panic,
then the algorithms would become simpler.
Sec. \ref{sec:future-work-no-cleanup} is concerned with
the question of not modeling the cleanup paths.