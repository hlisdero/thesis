\section{Deadlock detection using Petri nets}

Deadlock prevention is one of the classic strategies
to address this fundamental problem in concurrent programming,
as discussed in Sec. \ref{sec:deadlock-strategies}.
The main problem with the approach of detecting deadlocks before they occur
is proving that the desired type of deadlock is detected in all cases
and that no false negatives are produced in the process.
The Petri net-based approach, being a formal method, satisfies these conditions.
However, the difficulty of adoption lies mainly in the practicability of the solution
due to the large number of possible states in a real software project.

In \cite{karatkevich2014deadlock}, a method is proposed
to reduce the number of explored states
during the detection of deadlocks using reachability analysis.
These heuristics help improve the performance of the Petri net-based approach.
Another optimization is presented in \cite{kungas2005petri}.
The author proposes a very promising polynomial order method to avoid the problem
of the state explosion that underlies the naïve deadlock detection algorithm.
Through an algorithm that abstracts a given Petri net to a simpler representation,
a hierarchy of networks of increasing size is obtained
for which the verification of the absence of deadlocks is substantially faster.
It is, crudely put, a ``divide and conquer'' strategy
that checks for the absence of deadlocks in parts of the network
to later build the verification of the final whole
by adding parts to the initial small network.

Despite the previously mentioned caveats,
the use of Petri nets as a formal software verification method
has been established since the late 1980s.
Petri nets allow for intuitive modeling of synchronization primitives,
such as sending a message or waiting for the reception of a message.
Examples of these simple nets with correspondingly simple behavior
are found in \cite{heiner1992petri}.
These nets are construction blocks that can be combined to form a more complex system.

To put these models to use, there are two possibilities:

\begin{itemize}
      \item One is designing the system in terms of Petri nets
            and then translating the Petri nets to the source code.
      \item The other one is to translate the existing source code to a Petri net representation
            and then verify that the Petri net model satisfies the desired properties.
\end{itemize}

For the purposes of this work, we are interested in the latter.
This approach is not novel.
It has been implemented for other programming languages like C and Rust already,
as seen in the literature.

In \cite{kavi2002modeling} and \cite{moshtaghi2001},
a translation of some synchronization primitives available as part of
the POSIX library of threads (\texttt{pthread}) in C to Petri nets is described.
In particular, the translation supports:

\begin{itemize}
      \item The creation of threads with the function \texttt{pthread\_create}
            and the handling of the variable of type \texttt{pthread\_t}.
      \item The thread join operation with the \texttt{pthread\_join} function.
      \item The operation of acquiring a mutex with \texttt{pthread\_mutex\_lock}
            and its eventual manual release with \texttt{pthread\_mutex\_unlock}.
      \item The \texttt{pthread\_cond\_wait} and \texttt{pthread\_cond\_signal} functions
            for working with condition variables.
\end{itemize}

The source code for this library named ``C2Petri'' is disappointingly not found online,
as the publications are fairly old.

In a more recent master's thesis, \cite{meyer2020} establishes
the bases for a Petri net semantic for the Rust programming language.
He focuses his efforts however on single-threaded code,
limiting himself to the detection of deadlocks caused by
executing the \texttt{lock} operation twice on the same mutex in the main thread.
Unfortunately, the code available on GitHub\footnote{\url{https://github.com/Skasselbard/Granite}}
as part of the thesis is no longer valid for the new version of \emph{rustc}
since the internals of the compiler changed significantly in the last three years.

In a late 2022 pre-print, \cite{zhang2022deadlocks} implement a translation
of Rust source code to Petri nets for checking deadlocks.
The translation focuses on deadlocks caused by two types of locks
in the standard library: \texttt{std::sync::Mutex} and \texttt{std::sync::RwLock}.
The resulting Petri net is expressed in the \acrfull{PNML}
and fed into the model checker \acrfull{PIPE2}\footnote{\url{https://pipe2.sourceforge.net/}}
to perform reachability analysis.
Function calls are handled in a very different way compared to this work and
missed signals are not modeled at all.
The source code of their tool, named TRustPN, is not publicly available as of this writing.
Despite these limitations, the authors offer a very detailed and up-to-date survey
of static analysis tools for checking Rust code,
which could be appealing to the interested reader.
Moreover, they list several papers dedicated to
formalizing the semantics of the Rust programming language,
which are out of the scope of this work.